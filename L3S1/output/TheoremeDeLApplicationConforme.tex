\documentclass[14pt,usepdftitle=false,aspectratio=169]{beamer}
\usepackage{preambule}
\setbeamercolor{structure}{fg=black}
\usepackage{complexes}\usepackage[nopar]{analyse}\usepackage{footnotes}
\hypersetup{pdftitle=Analyse complexe -- Théorème de l'application conforme}
\title{Analyse complexe\\\emph{Théorème de l'application conforme}}
\author{}
\date{}
\begin{document}
\begin{frame}
    \titlepage
\end{frame}
\slideq{Formule de Cauchy}{1/4}
\slider{Si $f\in H\l U\r$, $\gamma$ est un lacet $\mathcal C^1$ par morceaux et $z\in U\setminus\gamma\l\left[0,1\right]\r$ alors $\frac1{2\i\pi}\int[w][\gamma][]{\frac{f\l w\r}{w-z}}=I\l z,\gamma\r f\l z\r$}{1/4}
\slideq{Logarithme d'une fonction holomorhe}{2/4}
\slider{Si $U$ est simplement connexe\footnote{Tout lacet est homotope au lacet constant} et $f\!:\!U\to\mathbb C^*$ est holomorphe alors $f$ admet un logarithme holomorphe, ie, il existe $F\in H\l U\r$ telle que $f=\e^F$}{2/4}
\slideq{CNS pour que l'ouvert $U$ soir simplement connexe}{3/4}
\slider{$U=\mathbb C$ ou $U$ est biholomorphe à $D\l0,1\r$}{3/4}
\slideq{Racine carrée d'une fonction holomorhe}{4/4}
\slider{Si $U$ est simplement connexe\footnote{Tout lacet est homotope au lacet constant} et $f\!:\!U\to\mathbb C^*$ est holomorphe alors $f$ admet une racine carrée, ie, il existe $F\in H\l U\r$ telle que $f=F^2$\footnote{L'existence d'une racine carrée est équivalent au caractère simplement connexe}}{4/4}
\end{document}