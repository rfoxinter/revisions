\section{Tris}
\begin{df*}{Tri en place}
Tri qui déplace les éléments à l'intérieur du tableau au lieu d'en créer un nouveau.
\end{df*}
\begin{df*}{Tri stable}
Tri qui préserve l'ordre des éléments (si \mintinline{c}|a| apparaît avant \mintinline{c}|b| dans le tableau initial, alors \mintinline{c}|a| apparaît avant \mintinline{c}|b| dans le tableau trié).
\end{df*}
%%%%%%%%%%%%%%%%%%%%%%%%%%%%%%%%%%%%%%%%%%%%%%%%%%%%%%%%%%%%%%%%%%%%%%%%%%%%%%%%%%%%%%%%%%%%%%%%%%%
\subsection{Différents algorithmes de tris}
\begin{df*}{Tri par sélection}
Tri qui sélectionne le plus petit élement du tableau non trié et l'insère au début de ce dernier.
\end{df*}
\c
\begin{imp*}{échange de deux élements dans un tableau}
\begin{minted}{c}
void array_swap(int* tab, int i, int j){
    int tmp = tab[i];
    tab[i] = tab[j];
    tab[j] = tmp;
}
\end{minted}
\info{}{Échange les éléments \mintinline{c}|i| et \mintinline{c}|j| du tableau \mintinline{c}|tab|}{\Th{1}}
\end{imp*}
\begin{imp*}{tri sélection}
\begin{minted}{c}
void tri_selection(int* tab, int n) {
    for (int i = 0; i < n-1; ++i) {
        int min = tab[i];
        int idx_min = i;
        for(int j = i+1; j < n; ++j) {
            if(tab[j] < min) {
                min = tab[j];
                idx_min = j;
            }
        }
        array_swap(tab, i, idx_min);
    }
}
\end{minted}
\info{}{Tri le tableau \mintinline{c}|tab| avec un tri sélection}{\Th{n^2}}
\end{imp*}