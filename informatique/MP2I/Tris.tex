\section{Tris}
\begin{df*}{Tri en place}
Tri qui déplace les éléments à l'intérieur du tableau au lieu d'en créer un nouveau.
\end{df*}
\begin{df*}{Tri stable}
Tri qui préserve l'ordre des éléments (si \mintinline{c}|a| apparaît avant \mintinline{c}|b| dans le tableau initial, alors \mintinline{c}|a| apparaît avant \mintinline{c}|b| dans le tableau trié).
\end{df*}
%%%%%%%%%%%%%%%%%%%%%%%%%%%%%%%%%%%%%%%%%%%%%%%%%%%%%%%%%%%%%%%%%%%%%%%%%%%%%%%%%%%%%%%%%%%%%%%%%%%
\subsection{Différents algorithmes de tris}
\begin{df*}{Tri par sélection}
Tri qui sélectionne le plus petit élement du tableau non trié et l'insère au début de ce dernier.
\end{df*}
\begin{df*}{Tri par insertion}
Tri qui parcourt le tableau de gauche à droite, insérant chaque élément à sa place dans la partie triée du tableau.
\end{df*}
\begin{df*}{Tri à bulles}
Tri qui parcourt le tableau de gauche à droite en comparant chaque élément avec son voisin, et échange les deux éléments si le voisin est plus petit.
\end{df*}
\begin{pt*}{Tri à bulles}
Le tri à bulles est stable.
\end{pt*}
\begin{df*}{Tri rapide}
Tri qui sélectionne un pivot et partitionne le tableau en deux sous-tableaux: les éléments plus petits que le pivot et ceux plus grands. Cette opération est répétée récursivement sur chaque sous-tableau.
\end{df*}
\begin{df*}{Tri mixte}
Tri qui utilise plusieurs algorithmes de tri pour avoir une meilleure complexité.
\end{df*}
\c
\begin{imp*}{échange de deux élements dans un tableau}
\begin{minted}{c}
void array_swap(int* tab, int i, int j){
    int tmp = tab[i];
    tab[i] = tab[j];
    tab[j] = tmp;
}
\end{minted}
\info{}{Échange les éléments \mintinline{c}|i| et \mintinline{c}|j| du tableau \mintinline{c}|tab|}{\Th{1}}
\end{imp*}
\begin{imp*}{tri sélection}
\begin{minted}{c}
void tri_selection(int* tab, int n) {
    for (int i = 0; i < n-1; ++i) {
        int min = tab[i];
        int idx_min = i;
        for(int j = i+1; j < n; ++j) {
            if(tab[j] < min) {
                min = tab[j];
                idx_min = j;
            }
        }
        array_swap(tab, i, idx_min);
    }
}
\end{minted}
\info{}{Tri le tableau \mintinline{c}|tab| avec un tri sélection}{\Th{n^2}}
\end{imp*}
\begin{imp*}{tri insertion}
\begin{minted}{c}
void tri_insertion(int* tab, int n) {
    for(int i = 1; i < n; ++i) {
        for(int j = i; j > 0 && tab[j] < tab[j-1]; --j) {
            array_swap(tab, j, j-1);
        }
    }
}
\end{minted}
\info{}{Tri le tableau \mintinline{c}|tab| avec un tri insertion}{\Th{n^2}}
\end{imp*}
\begin{imp*}{tri à bulles}
\begin{minted}{c}
void tri_bulle(int* tab, int n) {
    for(int i = 0; i < n; ++i) {
        for(int j = n-1; j > i; --j) {
            if(tab[j] < tab[j-1]) {
                array_swap(tab, j, j-1);
            }
        }
    }
}
\end{minted}
\info{}{Tri le tableau \mintinline{c}|tab| avec un tri à bulles}{\Th{n^2}}
\end{imp*}
\begin{imp*}{partitionnement d'un tableau pour le tri rapide}
\begin{minted}{c}
int partition(int* tab, int n) {
    int pivot = tab[0];
    int d = n-1;
    int g = 1;
    while(d > g-1) {
        if(tab[g] < pivot) {++g;}
        else {array_swap(tab, g, d); --d;}
    }
    array_swap(tab, 0, g-1);
    return g;
}
\end{minted}
\info{int* tab, int n}{Partitionne le tableau \mintinline{c}|tab| de taille \mintinline{c}|n| pour le tri rapide}{\Th{n}}
\end{imp*}
\begin{imp*}{tri rapide}
\begin{minted}{c}
void tri_rapide(int* tab, int n) {
    int p = partition(tab, n);
    if(p > 1) {tri_rapide(tab, p);}
    if(n-p > 1) {tri_rapide(&tab[p], n-p);}
}
\end{minted}
\info{}{Trie le tableau \mintinline{c}|tab| avec un tri rapide}{\Th{n \log n} (en moyenne), \Th{n^2} (dans le pire des cas)}
\end{imp*}
\begin{imp*}{tri rapide}
\begin{minted}{c}
void tri_mixte(int* tab, int n) {
    if(n < 10) {
        tri_bulle(tab,n);
    } else {
        int p = partition(tab, n);
        if(p > 1) {tri_mixte(tab, p);}
        if(n-p > 1) {tri_mixte(&tab[p], n-p);}
    }
}
\end{minted}
\info{}{Trie le tableau \mintinline{c}|tab| avec un tri rapide ou un tri à bulles si la liste est petite}{\Th{n \log n} (en moyenne), \Th{n^2} (dans le pire des cas)}
\end{imp*}
\ocaml
\begin{imp*}{minimum d'une liste}
\begin{minted}{ocaml}
let min = function
    | [] -> failwith "Liste vide"
    | t::q ->
        let rec lst_min t = function
            | [] -> t;
            | t2::q2 when t < t2 -> lst_min t q2
            | t2::q2 -> lst_min t2 q2
        in lst_min t q;;
\end{minted}
\info{'a list -> 'a}{Renvoie le minimum de la liste fournie en paramètre}{\Th{n}}
\end{imp*}
\begin{imp*}{retire un élément d'une liste}
\begin{minted}{ocaml}
let rec retire elem = function
    | [] -> failwith "Liste vide"
    | t::q when t = elem -> q
    | t::q -> t::retire elem q;;
\end{minted}
\info{'a list -> 'a list}{Retire l'élément \mintinline{ocaml}|elem| de la liste fournie en paramètre}{\Th{n}}
\end{imp*}
\begin{imp*}{tri sélection}
\begin{minted}{ocaml}
let rec tri_selection = function
    | [] -> []
    | lst -> let m = min lst in m::tri_selection (retire m lst);;
\end{minted}
\info{'a list -> 'a list}{Trie le tableau avec un tri sélection}{\Th{n^2}}
\end{imp*}
\begin{imp*}{Tri insertion}
\begin{minted}{ocaml}
let rec tri_insertion = function
    | [] -> []
    | t::q ->
        let rec insere elem = function
            | [] -> [elem]
            | t::q when t < elem -> t::insere elem q
            | lst -> elem::lst
        in insere t (tri_insertion q);;
\end{minted}
\info{'a list -> 'a list}{Trie le tableau avec un tri insertion}{\Th{n^2}}
\end{imp*}
\begin{imp*}{tri bulle}
\begin{minted}{ocaml}
let rec tri_bulle = function
    | [] -> []
    | lst ->
        let rec bulle = function
            | [] -> []
            | t::[] -> [t]
            | t::q ->
                let b = bulle q in
                    if List.hd b < t
                    then List.hd b::t::(List.tl b)
                    else t::b
        in let b = bulle lst
        in List.hd b::tri_bulle (List.tl b);;
\end{minted}
\info{'a list -> 'a list}{Trie le tableau avec un tri bulle}{\Th{n^2}}
\end{imp*}
\begin{imp*}{joint deux listes}
\begin{minted}{ocaml}
let rec join lst = function
    | [] -> lst
    | t::q -> t::join lst q
\end{minted}
\info{'a list -> 'a list -> 'a list}{Joint une liste devant la liste \mintinline{ocaml}|lst|}{\Th{n} ($n$ est la longueur de la deuxième liste)}
\end{imp*}
\begin{imp*}{partitionnement d'une liste}
\begin{minted}{ocaml}
let rec partition p left right = function
    | [] -> (left,right)
    | t::q when t < p -> partition p (t::left) right q
    | t::q -> partition p left (t::right) q;;
\end{minted}
\info{'a -> 'a list -> 'a list -> 'a list|\newline\mintinline{ocaml}|    -> 'a list * 'a list}{Partitionne les éléments de la liste donnée en paramètre\newline\mintinline{ocaml}|left| et \mintinline{ocaml}|right| servent d'accumulateurs pour les éléments plus petits et plus grand que le pivot \mintinline{ocaml}|p|}{\Th{n}}
\end{imp*}
\begin{imp*}{tri rapide}
\begin{minted}{ocaml}
let rec tri_rapide = function
    | [] -> []
    | t::q ->
        let (l,r) = partition t [] [] q in
        join (t::tri_rapide r) (tri_rapide l);;
\end{minted}
\info{'a list -> 'a list}{Trie le tableau avec un tri rapide}{\Th{n \log n} (en moyenne), \Th{n^2} (dans le pire des cas)}
\end{imp*}