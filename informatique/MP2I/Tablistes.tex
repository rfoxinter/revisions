\section{\emph{Tablistes}}
\begin{df*}{\emph{Tabliste}}
Structure de données qui permet de stocker un sous-ensemble $\mathcal{N}$ fini de $\mathbb{N}$ (de la forme $\left[\!\left[0,n-1\right]\!\right]$), et d'effectuer des opérations sur celle-ci en \Th{1}.
\end{df*}
%%%%%%%%%%%%%%%%%%%%%%%%%%%%%%%%%%%%%%%%%%%%%%%%%%%%%%%%%%%%%%%%%%%%%%%%%%%%%%%%%%%%%%%%%%%%%%%%%%%
\c
\begin{tp*}{}
\begin{minted}{c}
typedef struct {int* pos; int* values; int size;} Tablist;
\end{minted}
La champ \mintinline{c}|values| correspond à une liste des entiers de $\mathcal{N}$ sous la forme d'un tableau de taille $n$ dont les \mintinline{c}|size| premières cases contiennent les éléments présents dans $\mathcal{N}$.\\Le champ \mintinline{c}|pos| est un tableau de taille $n$ tel que: si $\vr{k}\in\mathcal{N}$, \mintinline{c}|pos[k]| contient la position de \mintinline{c}|k| dans la liste \mintinline{c}|values| et une valeur quelconque sinon.
\end{tp*}
\begin{fnc*}{création d'une \emph{tabliste}}
\begin{minted}{c}
Tablist init(int n) {
    Tablist t = {
        .pos = malloc(n*sizeof(int)),
        .values = malloc(n*sizeof(int)),
        .size = 0
    };
    for (int i = 0; i < n; ++i) {
        t.pos[i] = 0; // nécessaire à cause des nouvelles normes
        t.values[i] = 0; // facultatif
    }
    return t;
}
\end{minted}
\info{}{Créé une \emph{tabliste} vide}{\Th{n}}
\end{fnc*}
\begin{fnc*}{appartenance à une \emph{tabliste}}
\begin{minted}{c}
bool mem(Tablist t, int k) {
    int p = t.pos[k];
    return (p > 0 && p <= t.size && t.values[p]==k);
}
\end{minted}
\info{}{Vérifie si un élement \mintinline{c}|k| appartient à la \emph{tabliste} \mintinline{c}|t|}{\Th{1}}
\end{fnc*}
\begin{fnc*}{ajout à une \emph{tabliste}}
\begin{minted}{c}
void add(Tablist* ptr_t, int k) {
    if (!mem(*ptr_t, k)) {
        ptr_t->values[ptr_t->size] = k;
        ptr_t->pos[k] = ptr_t->size;
        ++ptr_t->size;
    };
}
\end{minted}
\info{}{Ajoute un élément \mintinline{c}|k| à la \emph{tabliste} pointée par \mintinline{c}|ptr_t|}{\Th{1}}
\end{fnc*}
\begin{fnc*}{supression d'une \emph{tabliste}}
\begin{minted}{c}
void t_remove(Tablist* ptr_t, int k) {
    if (mem(*ptr_t,k)) {
        int i = ptr_t->values[ptr_t->size-1];
        int p = ptr_t->pos[k];
        ptr_t->values[p] = i;
        ptr_t->pos[i] = p;
        --ptr_t->size;
    }
}
\end{minted}
\info{}{Supprime l'élément \mintinline{c}|k| de la \emph{tabliste} pointée par \mintinline{c}|ptr_t|}{\Th{1}}
\end{fnc*}
\begin{fnc*}{affichage d'une \emph{tabliste}}
\begin{minted}{c}
void print(Tablist t) {
    for (int i = 0; i < t.size-1; ++i) {
        printf("%d, ",t.values[i]);
    }
    printf("%d\n",t.values[t.size-1]);
}
\end{minted}
\info{}{Affiche les éléments de la \emph{tabliste} \mintinline{c}|t| dans laquelle ils ont été ajoutés à celle-ci}{\Th{\left|\mathcal{N}\right|}}
\end{fnc*}
\begin{fnc*}{vidage d'une \emph{tabliste}}
\begin{minted}{c}
void empty(Tablist* ptr_t) {
    ptr_t->size = 0;
}
\end{minted}
\info{}{Supprime les éléments de la \emph{tabliste} pointée par \mintinline{c}|ptr_t| ($\mathcal{N}=\varnothing$)}{\Th{1}}
\end{fnc*}
%%%%%%%%%%%%%%%%%%%%%%%%%%%%%%%%%%%%%%%%%%%%%%%%%%%%%%%%%%%%%%%%%%%%%%%%%%%%%%%%%%%%%%%%%%%%%%%%%%%
\ocaml
\begin{tp*}{}
\begin{minted}{ocaml}
type tablist = {
    pos : int array;
    values : int array;
    mutable size : int
};;
\end{minted}
La champ \mintinline{c}|values| correspond à une liste des entiers de $\mathcal{N}$ sous la forme d'un tableau de taille $n$ dont les \mintinline{c}|size| premières cases contiennent les éléments présents dans $\mathcal{N}$.\\Le champ \mintinline{c}|pos| est un tableau de taille $n$ tel que: si $\vr{k}\in\mathcal{N}$, \mintinline{c}|pos[k]| contient la position de \mintinline{c}|k| dans la liste \mintinline{c}|values| et une valeur quelconque sinon.
\end{tp*}
\begin{fnc*}{création d'une \emph{tabliste}}
\begin{minted}{ocaml}
let init n = {
    pos = Array.make n 0;
    values = Array.make n 0;
    size = 0
};;
\end{minted}
\info{int -> tablist}{Créé une \emph{tabliste} vide}{\Th{n}}
\end{fnc*}
\begin{fnc*}{appartenance à une \emph{tabliste}}
\begin{minted}{ocaml}
let mem t k =
    let p = t.pos.(k) in
    p > 0 && p < t.size && t.values.(p) = k;;
\end{minted}
\info{tablist -> int -> bool}{Vérifie si un élement \mintinline{c}|k| appartient à la \emph{tabliste} \mintinline{c}|t|}{\Th{1}}
\end{fnc*}
\begin{fnc*}{ajout à une \emph{tabliste}}
\begin{minted}{ocaml}
let add t k =
    if not (mem t k) then
        begin
            t.values.(t.size) <- k;
            t.pos.(k) <- t.size;
            t.size <- t.size + 1;
        end;
    ;;
\end{minted}
\info{tablist -> int -> unit}{Ajoute un élément \mintinline{c}|k| à la \emph{tabliste} pointée par \mintinline{c}|ptr_t|}{\Th{1}}
\end{fnc*}
\begin{fnc*}{supression d'une \emph{tabliste}}
\begin{minted}{ocaml}
let remove t k =
    if mem t k then
        begin
            let i = t.values.(t.size - 1)
            and p = t.pos.(k) in
            t.values.(p) <- i;
            t.pos.(i) <- p;
            t.size <- t.size - 1;
        end;
    ;;
\end{minted}
\info{tablist -> int -> unit}{Supprime l'élément \mintinline{c}|k| de la \emph{tabliste} pointée par \mintinline{c}|ptr_t|}{\Th{1}}
\end{fnc*}
\begin{fnc*}{affichage d'une \emph{tabliste}}
\begin{minted}{ocaml}
let print t =
    for i = 0 to t.size - 2 do
        print_int t.values.(i);
        print_string ", ";
    done;
    print_int t.values.(t.size - 1);
    print_newline ();;
\end{minted}
\info{tablist -> unit}{Affiche les éléments de la \emph{tabliste} \mintinline{c}|t| dans laquelle ils ont été ajoutés à celle-ci}{\Th{\left|\mathcal{N}\right|}}
\end{fnc*}
\begin{fnc*}{vidage d'une \emph{tabliste}}
\begin{minted}{ocaml}
let empty t = t.size <- 0;;
\end{minted}
\info{tablist -> unit}{Supprime les éléments de la \emph{tabliste} pointée par \mintinline{c}|ptr_t| ($\mathcal{N}=\varnothing$)}{\Th{1}}
\end{fnc*}