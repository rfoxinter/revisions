\documentclass[a4paper,12pt,titlepage]{article}
\usepackage[utf8]{inputenc}
\usepackage[french]{babel}
\usepackage[T1]{fontenc}
\usepackage[margin=2.5cm]{geometry}
\usepackage{amsmath, amsfonts, amssymb}
\usepackage{xcolor}
\usepackage[most]{tcolorbox}
\usepackage{minted}
\usepackage{tabularx}
\usepackage{hyperref}
\usepackage{graphics}
\usemintedstyle{perso}
\setminted{breaklines,breaksymbolleft=\qquad,breaksymbolright=}
\renewcommand{\tabularxcolumn}[1]{>{\arraybackslash}m{#1}}

\renewcommand{\c}{\subsection[C]{C\hfill\protect\includegraphics[height=11pt]{C}}}
\newcommand{\ocaml}{\subsection[OCaml]{OCaml\hfill\protect\includegraphics[height=11pt]{Ocaml}}}

\let\oldsection\section
\renewcommand\section{\clearpage\oldsection}
\newcolumntype{A}{@{}>{\raggedright\arraybackslash\hspace{0pt}}p{.2\hsize}@{}}
\newcolumntype{B}{@{}>{\raggedright\arraybackslash\hspace{0pt}}p{.8\hsize}X@{}}
\let\oldleft\left
\renewcommand{\left}{\mathopen{}\mathclose\bgroup\oldleft}
\let\oldright\right
\renewcommand{\right}{\aftergroup\egroup\oldright}
\renewcommand{\O}[1]{O\left(#1\right)}
\newcommand{\Th}[1]{\varTheta\left(#1\right)}

\setlength{\parindent}{0pt}
\newtcbtheorem{fun}{Fonction disponible }{breakable,colback=green!5,colframe=green!75!black,fonttitle=\bfseries}{func}
\newtcbtheorem{imp}{Implémentation }{breakable,colback=orange!5,colframe=orange!50!red,fonttitle=\bfseries}{impl}
\newtcbtheorem{tp}{Définition d'un type }{breakable,colback=violet!5,colframe=violet!75!black,fonttitle=\bfseries}{type}
\newtcbtheorem{fnc}{Fonction à implémenter }{breakable,colback=blue!5,colframe=blue!75!white,fonttitle=\bfseries}{func}

\newcommand{\info}[3][]{\begin{tabularx}{\textwidth}{AB}Description&#2\\Complexité&{\ensuremath{#3}}#1\end{tabularx}}
\newcommand{\fn}[4]{\begin{fun*}{}\info[\\Signature&\mintinline{#1}|#2|]{#3}{#4}\end{fun*}}

\hypersetup{hidelinks,linktoc=all}
\setcounter{secnumdepth}{5}
\setcounter{tocdepth}{5}
\hypersetup{pdftitle=Informatique MP2I}
\title{Informatique MP2I}
\author{}
\date{}

\begin{document}
\pagenumbering{gobble}
\maketitle
\null\newpage
\tableofcontents
\null\newpage
\pagenumbering{arabic}
\section{Listes chaînées}
\c
\subsubsection{Type complexe}
\begin{tp*}{}
\begin{minted}{c}
struct int_list {
    int value;
    struct int_list* next;
};
typedef struct int_list int_list;
// Cette définition de listes chaînées s'adapte également pour les autres types
\end{minted}
\end{tp*}
\begin{fnc*}{Longueur d'une liste}
\begin{minted}{c}
int length(int_list* lst) {
    if (lst == NULL) {
        return 0;
    }
    return 1 + length(int_list->next);
}
\end{minted}
\info{Renvoie la longueur d'une liste}{\Th{n}}
\end{fnc*}
\ocaml
\subsubsection{\mintinline{ocaml}|List.length|}
\fn{ocaml}{'a list -> int}{Renvoie la longueur de la liste}{\Th{n}}
\begin{imp*}{}
\begin{minted}{ocaml}
let rec length = function
    | [] -> 0
    | h::t -> 1 + length t;;
\end{minted}
\end{imp*}
\subsubsection{\mintinline{ocaml}|List.iter|}
\fn{ocaml}{('a -> unit) -> 'a list -> unit}{Applique une fonction à tous les éléments de la liste}{\Th{n}}
\begin{imp*}{}
\begin{minted}{ocaml}
let rec iter f = function
    | [] -> ()
    | h::t -> f h; iter f t;;
\end{minted}
\end{imp*}
\subsubsection{\mintinline{ocaml}|List.fold_left|}
\fn{ocaml}{('a -> 'b -> 'a) -> 'a -> 'b list -> 'a}{Applique une fonction successivement à un élément de la liste et au résultat de l'itération précédente en partant de la fin de la liste}{\Th{n}}
\begin{imp*}{}
\begin{minted}{ocaml}
let rec fold_left f acc = function
    | [] -> acc
    | h::t -> fold_left f (f acc h) t;;
\end{minted}
\end{imp*}

\end{document}