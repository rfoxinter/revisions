% !TEX root = InformatiqueMP2I.tex

\documentclass[a4paper,12pt,titlepage]{article}
\usepackage[utf8]{inputenc}
\usepackage[french]{babel}
\usepackage[T1]{fontenc}
\usepackage[margin=2.5cm]{geometry}
\usepackage{amsmath, amsfonts, amssymb}
\usepackage{xcolor}
\usepackage[most]{tcolorbox}
\usepackage[finalizecache=true]{minted}
\usepackage{ragged2e}
\usepackage{tabularx}
\usepackage{hyperref}
\usepackage{graphics}
\usemintedstyle{perso}
\renewcommand{\tabularxcolumn}[1]{>{\justifying\arraybackslash}m{#1}}

\renewcommand{\c}{\subsection[C]{C\hfill\protect\includegraphics[height=11pt]{C}}}
\newcommand{\ocaml}{\subsection[OCaml]{OCaml\hfill\protect\includegraphics[height=11pt]{Ocaml}}}

\let\oldsection\section
\renewcommand\section{\clearpage\oldsection}

\newcolumntype{A}{@{}>{\raggedright\arraybackslash\hspace{0pt}}p{.2\hsize}@{}}
\newcolumntype{B}{@{}>{\raggedright\arraybackslash\hspace{0pt}}p{.8\hsize}X@{}}

\let\oldleft\left
\renewcommand{\left}{\mathopen{}\mathclose\bgroup\oldleft}
\let\oldright\right
\renewcommand{\right}{\aftergroup\egroup\oldright}
\renewcommand{\O}[1]{\ensuremath{O\left(#1\right)}}
\newcommand{\Th}[1]{\ensuremath{\varTheta\left(#1\right)}}
\let\oldlog\log
\renewcommand{\log}[1]{\oldlog\left(#1\right)}

\definecolor{kywd}{RGB}{17,96,132}
\definecolor{fnc}{RGB}{160,160,0}
\definecolor{var}{RGB}{0,40,186}
\newcommand{\kywd}[1]{\texttt{\color{kywd}#1}}
\newcommand{\vr}[1]{\mathtt{\color{var}#1}}

\setlength{\parindent}{0pt}
\newtcbtheorem{fun}{Fonction disponible}{colback=green!5,colframe=green!75!black,fonttitle=\bfseries,separator sign={~:}}{func}
\newtcbtheorem{imp}{Implémentation}{colback=orange!5,colframe=orange!50!red,fonttitle=\bfseries,separator sign={~:}}{impl}
\newtcbtheorem{tp}{Définition d'un type}{colback=violet!5,colframe=violet!75!black,fonttitle=\bfseries,separator sign={~:}}{type}
\newtcbtheorem{fnc}{Fonction à implémenter}{colback=blue!5,colframe=blue!75!white,fonttitle=\bfseries,separator sign={~:}}{func}
\newtcbtheorem{df}{Définition}{colback=teal!5,colframe=teal!75!white,fonttitle=\bfseries,separator sign={~:}}{defi}
\newtcbtheorem{pt}{Propriété/Théorème}{colback=lime!5,colframe=lime!75!black,fonttitle=\bfseries,separator sign={~:}}{prop}

\newcommand{\info}[4][]{\def\argI{}\ifx&#2&\else\def\argI{Signature&\mintinline{ocaml}|#2|\\}\fi\def\argII{}\ifx&#3&\else\def\argII{Description&#3\\}\fi\begin{tabularx}{\textwidth}{AB}\argI\argII Complexité&#4#1\end{tabularx}}
\newcommand{\fn}[5]{\begin{fun*}{\mintinline{#1}|#2|}\info{#3}{#4}{#5}\end{fun*}}

\hypersetup{hidelinks,linktoc=all}
\setcounter{secnumdepth}{5}
\setcounter{tocdepth}{5}
\hypersetup{pdftitle=Informatique MP2I}
\title{Informatique MP2I}
\author{}
\date{}

\begin{document}
\pagenumbering{gobble}
\maketitle
\null\newpage
\tableofcontents
\null\newpage
\null\newpage
\pagenumbering{arabic}
\section{Listes chaînées}
\c
\begin{tp*}{}
\begin{minted}{c}
struct cell {
    int value;
    struct cell* next;
};
typedef struct cell int_list;
/* Cette définition de listes chaînées ainsi que la majorité des
    fonctions à suivre s'adapte également pour les autres types */
\end{minted}
\end{tp*}
\begin{fnc*}{longueur d'une liste}
\begin{minted}{c}
int length(int_list* lst) {
    if (lst == NULL) {
        return 0;
    }
    return 1 + length(int_list->next);
}
\end{minted}
\info{}{Renvoie la longueur de la liste}{\Th{n}}
\end{fnc*}
\begin{fnc*}{ajoute un élément à gauche d'une liste}
\begin{minted}{c}
void add_l(int elem, int_list* lst) {
    list_int* new_p = (list_int*)malloc(n*sizeof(list_int));
    new_p->next = lst;
    new_p->value = elem;
    return new_p
}
\end{minted}
\info{}{Ajoute un élément à gauche de la liste}{\Th{1}}
\end{fnc*}
\ocaml
\subsubsection{Type complexe}
\subsubsection{Le module \kywd{List}}
\fn{ocaml}{List.length}{'a list -> int}{Renvoie la longueur de la liste}{\Th{n}}
\begin{imp*}{}
\begin{minted}{ocaml}
let rec length = function
    | [] -> 0
    | h::t -> 1 + length t;;
\end{minted}
\info{'a list -> int}{Renvoie la longueur de la liste}{\Th{n}}
\end{imp*}
\fn{ocaml}{List.iter}{('a -> unit) -> 'a list -> unit}{Applique une fonction à tous les éléments de la liste}{\Th{n}}
\begin{imp*}{}
\begin{minted}{ocaml}
let rec iter f = function
    | [] -> ()
    | h::t -> f h; iter f t;;
\end{minted}
\info{('a -> unit) -> 'a list -> unit}{Applique une fonction à tous les éléments de la liste}{\Th{n}}
\end{imp*}
\fn{ocaml}{List.fold_left}{('a -> 'b -> 'a) -> 'a -> 'b list -> 'a}{Applique une fonction successivement à un élément de la liste et au résultat de l'itération précédente en partant de la fin de la liste}{\Th{n}}
\begin{imp*}{}
\begin{minted}{ocaml}
let rec fold_left f acc = function
    | [] -> acc
    | h::t -> fold_left f (f acc h) t;;
\end{minted}
\info{('a -> 'b -> 'a) -> 'a -> 'b list -> 'a}{Applique une fonction successivement à un élément de la liste et au résultat de l'itération précédente en partant de la fin de la liste}{\Th{n}}
\end{imp*}
\section{Tableaux}
%%%%%%%%%%%%%%%%%%%%%%%%%%%%%%%%%%%%%%%%%%%%%%%%%%%%%%%%%%%%%%%%%%%%%%%%%%%%%%%%%%%%%%%%%%%%%%%%%%%
\c
%%%%%%%%%%%%%%%%%%%%%%%%%%%%%%%%%%%%%%%%%%%%%%%%%%%%%%%%%%%%%%%%%%%%%%%%%%%%%%%%%%%%%%%%%%%%%%%%%%%
\ocaml[Le module \kywd{Array}]
\fn{ocaml}{Array.length}{'a array -> int}{Renvoie la longueur du tableau}{\Th{1}}
\begin{imp*}{}
\begin{minted}{ocaml}
let length arr =
    let l = ref 0 and count = ref true in
    while !count do
        try
            let _ = arr.(!l) in incr l
        with
            | Invalid_argument _ -> count := false
    done;
    !l;;
\end{minted}
\info{'a array -> int}{Renvoie la longueur du tableau}{\Th{n}}
\end{imp*}
\fn{ocaml}{Array.iter}{('a -> unit) -> 'a array -> unit}{Applique une fonction à tous les éléments du tableau}{\Th{n}}
\begin{imp*}{}
\begin{minted}{ocaml}
let iter f arr =
    for i = 0 to length arr-1 do
        arr.(i) <- (f arr.(i))
    done;;
\end{minted}
\info{('a -> unit) -> 'a array -> unit}{Applique une fonction à tous les éléments du tableau}{\Th{n}}
\end{imp*}
\fn{ocaml}{Array.fold_left}{('a -> 'b -> 'a) -> 'a -> 'b array -> 'a}{Applique une fonction successivement à un élément du tableau et au résultat de l'itération précédente en partant du début du tableau}{\Th{n}}
\begin{imp*}{}
\begin{minted}{ocaml}
let fold_left f def arr =
    let acc = ref def in
    for i = 0 to length arr-1 do
        acc := f !acc arr.(i)
    done;
    !acc;;
\end{minted}
\info{('a -> 'b -> 'a) -> 'a -> 'b array -> 'a}{Applique une fonction successivement à un élément du tableau et au résultat de l'itération précédente en partant du début du tableau}{\Th{n}}
\end{imp*}
\fn{ocaml}{Array.fold_right}{('a -> 'b -> 'b) -> 'a array -> 'b -> 'b}{Applique une fonction binaire de pliage à tous les éléments du tableau, en partant de la droite, et renvoie un résultat plié}{\Th{n}}
\begin{imp*}{}
\begin{minted}{ocaml}
let fold_right f arr def =
    let acc = ref def in
    for i = length arr-1 downto 0 do
        acc := f arr.(i) !acc
    done;
    !acc;;
\end{minted}
\info{('a -> 'b -> 'b) -> 'a array -> 'b -> 'b}{Applique une fonction binaire de pliage à tous les éléments du tableau, en partant de la droite, et renvoie un résultat plié}{\Th{n}}
\end{imp*}
\fn{ocaml}{Array.map}{('a -> 'b) -> 'a array -> 'b array}{Applique une fonction à tous les éléments du tableau et renvoie un nouveau tableau avec les résultats}{\Th{n}}
\begin{imp*}{}
\begin{minted}{ocaml}
let map f arr =
    let new_arr = Array.make (Array.length arr) (f arr.(0)) in
    for i = 1 to length arr-1 do
        new_arr.(i) <- f arr.(i)
    done;
    new_arr;;
\end{minted}
\info{('a -> 'b) -> 'a array -> 'b array}{Applique une fonction à tous les éléments du tableau et renvoie un nouveau tableau avec les résultats}{\Th{n}}
\end{imp*}
\section{Tris}
\begin{df*}{Tri en place}
Tri qui déplace les éléments à l'intérieur du tableau au lieu d'en créer un nouveau.
\end{df*}
\begin{df*}{Tri stable}
Tri qui préserve l'ordre des éléments (si \mintinline{c}|a| apparaît avant \mintinline{c}|b| dans le tableau initial, alors \mintinline{c}|a| apparaît avant \mintinline{c}|b| dans le tableau trié).
\end{df*}
%%%%%%%%%%%%%%%%%%%%%%%%%%%%%%%%%%%%%%%%%%%%%%%%%%%%%%%%%%%%%%%%%%%%%%%%%%%%%%%%%%%%%%%%%%%%%%%%%%%
\subsection{Différents algorithmes de tris}
\begin{df*}{Tri par sélection}
Tri qui sélectionne le plus petit élement du tableau non trié et l'insère au début de ce dernier.
\end{df*}
\c
\begin{imp*}{échange de deux élements dans un tableau}
\begin{minted}{c}
void array_swap(int* tab, int i, int j){
    int tmp = tab[i];
    tab[i] = tab[j];
    tab[j] = tmp;
}
\end{minted}
\info{}{Échange les éléments \mintinline{c}|i| et \mintinline{c}|j| du tableau \mintinline{c}|tab|}{\Th{1}}
\end{imp*}
\begin{imp*}{tri sélection}
\begin{minted}{c}
void tri_selection(int* tab, int n) {
    for (int i = 0; i < n-1; ++i) {
        int min = tab[i];
        int idx_min = i;
        for(int j = i+1; j < n; ++j) {
            if(tab[j] < min) {
                min = tab[j];
                idx_min = j;
            }
        }
        array_swap(tab, i, idx_min);
    }
}
\end{minted}
\info{}{Tri le tableau \mintinline{c}|tab| avec un tri sélection}{\Th{n^2}}
\end{imp*}
\section{\emph{Tablistes}}
\begin{df*}{\emph{Tabliste}}
Structure de données qui permet de stocker un sous-ensemble $\mathcal{N}$ fini de $\mathbb{N}$ (de la forme $\left[\!\left[0,n-1\right]\!\right]$), et d'effectuer des opérations sur celle-ci en \Th{1}.
\end{df*}
%%%%%%%%%%%%%%%%%%%%%%%%%%%%%%%%%%%%%%%%%%%%%%%%%%%%%%%%%%%%%%%%%%%%%%%%%%%%%%%%%%%%%%%%%%%%%%%%%%%
\c
\begin{tp*}{}
\begin{minted}{c}
typedef struct {int* pos; int* values; int size;} Tablist;
\end{minted}
La champ \mintinline{c}|values| correspond à une liste des entiers de $\mathcal{N}$ sous la forme d'un tableau de taille $n$ dont les \mintinline{c}|size| premières cases contiennent les éléments présents dans $\mathcal{N}$.\\Le champ \mintinline{c}|pos| est un tableau de taille $n$ tel que: si $\vr{k}\in\mathcal{N}$, \mintinline{c}|pos[k]| contient la position de \mintinline{c}|k| dans la liste \mintinline{c}|values| et une valeur quelconque sinon.
\end{tp*}
\begin{fnc*}{création d'une \emph{tabliste}}
\begin{minted}{c}
Tablist init(int n) {
    Tablist t = {
        .pos = malloc(n*sizeof(int)),
        .values = malloc(n*sizeof(int)),
        .size = 0
    };
    for (int i = 0; i < n; ++i) {
        t.pos[i] = 0; // nécessaire à cause des nouvelles normes
        t.values[i] = 0; // facultatif
    }
    return t;
}
\end{minted}
\info{}{Créé une \emph{tabliste} vide}{\Th{n}}
\end{fnc*}
\begin{fnc*}{appartenance à une \emph{tabliste}}
\begin{minted}{c}
bool mem(Tablist t, int k) {
    int p = t.pos[k];
    return (p > 0 && p <= t.size && t.values[p]==k);
}
\end{minted}
\info{}{Vérifie si un élement \mintinline{c}|k| appartient à la \emph{tabliste} \mintinline{c}|t|}{\Th{1}}
\end{fnc*}
\begin{fnc*}{ajout à une \emph{tabliste}}
\begin{minted}{c}
void add(Tablist* ptr_t, int k) {
    if (!mem(*ptr_t, k)) {
        ptr_t->values[ptr_t->size] = k;
        ptr_t->pos[k] = ptr_t->size;
        ++ptr_t->size;
    };
}
\end{minted}
\info{}{Ajoute un élément \mintinline{c}|k| à la \emph{tabliste} pointée par \mintinline{c}|ptr_t|}{\Th{1}}
\end{fnc*}
\begin{fnc*}{supression d'une \emph{tabliste}}
\begin{minted}{c}
void t_remove(Tablist* ptr_t, int k) {
    if (mem(*ptr_t,k)) {
        int i = ptr_t->values[ptr_t->size-1];
        int p = ptr_t->pos[k];
        ptr_t->values[p] = i;
        ptr_t->pos[i] = p;
        --ptr_t->size;
    }
}
\end{minted}
\info{}{Supprime l'élément \mintinline{c}|k| de la \emph{tabliste} pointée par \mintinline{c}|ptr_t|}{\Th{1}}
\end{fnc*}
\begin{fnc*}{affichage d'une \emph{tabliste}}
\begin{minted}{c}
void print(Tablist t) {
    for (int i = 0; i < t.size-1; ++i) {
        printf("%d, ",t.values[i]);
    }
    printf("%d\n",t.values[t.size-1]);
}
\end{minted}
\info{}{Affiche les éléments de la \emph{tabliste} \mintinline{c}|t| dans laquelle ils ont été ajoutés à celle-ci}{\Th{\left|\mathcal{N}\right|}}
\end{fnc*}
\begin{fnc*}{vidage d'une \emph{tabliste}}
\begin{minted}{c}
void empty(Tablist* ptr_t) {
    ptr_t->size = 0;
}
\end{minted}
\info{}{Supprime les éléments de la \emph{tabliste} pointée par \mintinline{c}|ptr_t| ($\mathcal{N}=\varnothing$)}{\Th{1}}
\end{fnc*}
%%%%%%%%%%%%%%%%%%%%%%%%%%%%%%%%%%%%%%%%%%%%%%%%%%%%%%%%%%%%%%%%%%%%%%%%%%%%%%%%%%%%%%%%%%%%%%%%%%%
\ocaml
\begin{tp*}{}
\begin{minted}{ocaml}
type tablist = {
    pos : int array;
    values : int array;
    mutable size : int
};;
\end{minted}
La champ \mintinline{c}|values| correspond à une liste des entiers de $\mathcal{N}$ sous la forme d'un tableau de taille $n$ dont les \mintinline{c}|size| premières cases contiennent les éléments présents dans $\mathcal{N}$.\\Le champ \mintinline{c}|pos| est un tableau de taille $n$ tel que: si $\vr{k}\in\mathcal{N}$, \mintinline{c}|pos[k]| contient la position de \mintinline{c}|k| dans la liste \mintinline{c}|values| et une valeur quelconque sinon.
\end{tp*}
\begin{fnc*}{création d'une \emph{tabliste}}
\begin{minted}{ocaml}
let init n = {
    pos = Array.make n 0;
    values = Array.make n 0;
    size = 0
};;
\end{minted}
\info{int -> tablist}{Créé une \emph{tabliste} vide}{\Th{n}}
\end{fnc*}
\begin{fnc*}{appartenance à une \emph{tabliste}}
\begin{minted}{ocaml}
let mem t k =
    let p = t.pos.(k) in
    p > 0 && p < t.size && t.values.(p) = k;;
\end{minted}
\info{tablist -> int -> bool}{Vérifie si un élement \mintinline{c}|k| appartient à la \emph{tabliste} \mintinline{c}|t|}{\Th{1}}
\end{fnc*}
\begin{fnc*}{ajout à une \emph{tabliste}}
\begin{minted}{ocaml}
let add t k =
    if not (mem t k) then
        begin
            t.values.(t.size) <- k;
            t.pos.(k) <- t.size;
            t.size <- t.size + 1;
        end;
    ;;
\end{minted}
\info{tablist -> int -> unit}{Ajoute un élément \mintinline{c}|k| à la \emph{tabliste} pointée par \mintinline{c}|ptr_t|}{\Th{1}}
\end{fnc*}
\begin{fnc*}{supression d'une \emph{tabliste}}
\begin{minted}{ocaml}
let remove t k =
    if mem t k then
        begin
            let i = t.values.(t.size - 1)
            and p = t.pos.(k) in
            t.values.(p) <- i;
            t.pos.(i) <- p;
            t.size <- t.size - 1;
        end;
    ;;
\end{minted}
\info{tablist -> int -> unit}{Supprime l'élément \mintinline{c}|k| de la \emph{tabliste} pointée par \mintinline{c}|ptr_t|}{\Th{1}}
\end{fnc*}
\begin{fnc*}{affichage d'une \emph{tabliste}}
\begin{minted}{ocaml}
let print t =
    for i = 0 to t.size - 2 do
        print_int t.values.(i);
        print_string ", ";
    done;
    print_int t.values.(t.size - 1);
    print_newline ();;
\end{minted}
\info{tablist -> unit}{Affiche les éléments de la \emph{tabliste} \mintinline{c}|t| dans laquelle ils ont été ajoutés à celle-ci}{\Th{\left|\mathcal{N}\right|}}
\end{fnc*}
\begin{fnc*}{vidage d'une \emph{tabliste}}
\begin{minted}{ocaml}
let empty t = t.size <- 0;;
\end{minted}
\info{tablist -> unit}{Supprime les éléments de la \emph{tabliste} pointée par \mintinline{c}|ptr_t| ($\mathcal{N}=\varnothing$)}{\Th{1}}
\end{fnc*}
\pagenumbering{gobble}
\null\newpage
\end{document}