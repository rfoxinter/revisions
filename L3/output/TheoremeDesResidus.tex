\documentclass[14pt,usepdftitle=false,aspectratio=169]{beamer}
\usepackage{preambule}
\setbeamercolor{structure}{fg=black}
\DeclareMathOperator{\oldres}{Res}\newcommand{\res}[2]{\oldres\l#1;#2\r}\let\phi\varphi\let\Phi\varPhi\usepackage{analyse}\togglebigoppar\usepackage{complexes,usuelles}%\usepackage{analyse,trigo,complexes}\newcommand{\appl}[5]{\begin{array}[t]{@{}r@{}r@{}c@{}l@{}}#1\!:\!{}&#2&{}\longrightarrow{}&#3\\&#4&{}\longmapsto{}&#5\end{array}}
\hypersetup{pdftitle=Analyse complexe -- Théorème des résidus}
\title{Analyse complexe\\\emph{Théorème des résidus}}
\author{}
\date{}
\begin{document}
\begin{frame}
    \titlepage
\end{frame}
\slideq{CNS pour que $U$ soit élémentaire}{1/8}
\slider{$U$ est simplement connexe\linebreak$U=\mathbb C$ ou $U$ est biholomorphe à $D\l 0,1\r$}{1/8}
\slideq{Un ouvert $U$ de $\mathbb C$ est élémentaire}{2/8}
\slider{$U$ est non vide, connexe et toute fonction holomorphe sur $U$ admet une primitive sur $U$}{2/8}
\slideq{$I\l a,\gamma\r$ pour un lacet $\mathcal C^1$ par morceaux}{3/8}
\slider{$\frac{1}{2\i\pi}{\displaystyle\oldint_\gamma}\frac{\dd z}{z-a}$}{3/8}
\slideq{Transfert du caractère élémentaire par un holomorphisme}{4/8}
\slider{Si $\phi\!:\!U_1\to U_2$ est un biholomorphisme et $U_1$ est élémentaire alors $U_2$ est élémentaire}{4/8}
\slideq{$\res fz$}{5/8}
\slider{Coefficient $a_{-1}$ du développement en série de Laurent de $f$ en $z$}{5/8}
\slideq{Indice du lacet $\gamma\!:\!\left[\alpha,\beta\right]\to\mathbb C$ continu par rapport à $z\in\mathbb C$}{6/8}
\slider{$\Phi\l\beta\r-\Phi\l\alpha\r$ où $\Phi\l t\r$ est continue et vérifie $\phi\l t\r=\exp{2\i\pi\Phi\l t\r}$}{6/8}
\slideq{Stabilité du caractère ouvert par union}{7/8}
\slider{Si $U_1$ et $U_2$ sont élémentaires et $U_1\cap U_2$ est connexe alors $U_1\cup U_2$ est élémentaire\linebreak Si $\l U_n\r$ est une suite croissante d'ouverts élémentaires alors $\bigcup{n\in\mathbb N}{}{U_n}$ est élémentaire}{7/8}
\slideq{Théorème des résidus}{8/8}
\slider{Si $U$ est un ouvert élémentaire de $\mathbb C$, $F$ un ensemble fini de points de $U$, $f\in H\l U\setminus F\r$ et $\gamma$ un lacet $\mathcal C^1$ par morceaux dans $U\setminus F$ alors $\int[z][\gamma]{f\l z\r}=2\i\pi\sum{a\in F}{}{I\l a\gamma\r\res fa}$}{8/8}
\end{document}