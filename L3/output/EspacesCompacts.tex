\documentclass[14pt,usepdftitle=false,aspectratio=169]{beamer}
\usepackage{preambule}
\setbeamercolor{structure}{fg=black}
\newcounter{footnotemarkcounter}\setcounter{footnotemarkcounter}0\newcounter{footnotetextcounter}\setcounter{footnotetextcounter}0\renewcommand{\footnotemark}{\stepcounter{footnotemarkcounter}{\textsuperscript{\textit{\oldstylenums{\thefootnotemarkcounter}}}}}\let\oldfootnotetext\footnotetext\renewcommand{\footnotetext}[1]{{\stepcounter{footnotetextcounter}\def\thefootnote{\textit{\oldstylenums{\thefootnotetextcounter}}}\def\thempfootnote{\textit{\oldstylenums{\thefootnotetextcounter}}}\oldfootnotetext{#1}}}\renewcommand{\footnote}{\footnotemark\footnotetext}
\hypersetup{pdftitle=Topologie et calcul différentiel -- Espaces compacts}
\title{Topologie et calcul différentiel\\\emph{Espaces compacts}}
\author{}
\date{}
\begin{document}
\begin{frame}
    \titlepage
\end{frame}
\slideq{Description topologique des espaces compact}{1/10}
\slider{Si $E$ est un espace topologique, $E$ est compact si \emph{$E$ est séparé}\footnote{seulement en français} et tout recouvrement ouvert de $E$ a un sous-recouvrement fini}{1/10}
\slideq{Théorème de Riesz}{2/10}
\slider{La boule unité d'un ev $E$ est compacte si et seulement si $E$ est de dimension finie}{2/10}
\slideq{Théorème de Heine}{3/10}
\slider{Si $X$ est compact et $f\!:\!X\to Y$ est continue alors $f$ est uniformément continue}{3/10}
\slideq{Espace précompact}{4/10}
\slider{Espace $X$ pour lequel pour tout $\varepsilon>0$, $X$ est recouvrable par un nombre fini de boules de rayon $\varepsilon$}{4/10}
\slideq{Théorème de Borel-Lebesgue}{5/10}
\slider{L'espace métrique $\l E,d\r$ est compact si et seulement si de tout recouvrement par des ouverts, on peut extraire un sous-recouvrement fini}{5/10}
\slideq{Propriété du compact $\l S,d\r$\linebreak $S=\left\{0,1\right\}^{\mathbb N^*}$, $d\l x,y\r=2^{-\min\l\left\{ n\in\mathbb N,x_n\neq y_n\right\}\r}$}{6/10}
\slider{Si $\l K,d'\r$ est un espace compact alors il existe $f\!:\!S\to K$ une surjection continue}{6/10}
\slideq{Définition équivalente à la compacité}{7/10}
\slider{Complet et précompact}{7/10}
\slideq{Théorème de Stone}{8/10}
\slider{Si $\l X,d\r$ est un espace métrique compact et si $A$ est une $\mathbb R$-algèbre de fonctions continues qui sépare les points ($\forall x\neq y$, $\exists f\in A$, $f\l x\r\neq f\l y\r$) alors $A$ est dense dans $\mathcal C^0\l X,\mathbb R\r$}{8/10}
\slideq{Théorème de Weirstrass}{9/10}
\slider{Si $I=\left[a,b\right]$ alors toute fonction continue sur $I$ est limite uniforme de fonctions polynomiales}{9/10}
\slideq{Théorème de Poincaré}{10/10}
\slider{Si $X$ est compact et $f\!:\!X\to Y$ est bijective et continue alors $f^{-1}\!:\!Y\to X$ est continue}{10/10}
\end{document}