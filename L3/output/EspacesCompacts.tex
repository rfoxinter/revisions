\documentclass[14pt,usepdftitle=false,aspectratio=169]{beamer}
\usepackage{preambule}
\setbeamercolor{structure}{fg=black}
\newcounter{footnotemarkcounter}\setcounter{footnotemarkcounter}0\newcounter{footnotetextcounter}\setcounter{footnotetextcounter}0\renewcommand{\footnotemark}{\stepcounter{footnotemarkcounter}{\textsuperscript{\textit{\oldstylenums{\thefootnotemarkcounter}}}}}\let\oldfootnotetext\footnotetext\renewcommand{\footnotetext}[1]{{\stepcounter{footnotetextcounter}\def\thefootnote{\textit{\oldstylenums{\thefootnotetextcounter}}}\def\thempfootnote{\textit{\oldstylenums{\thefootnotetextcounter}}}\oldfootnotetext{#1}}}\renewcommand{\footnote}{\footnotemark\footnotetext}
\hypersetup{pdftitle=Topologie et calcul différentiel -- Espaces compacts}
\title{Topologie et calcul différentiel\\\emph{Espaces compacts}}
\author{}
\date{}
\begin{document}
\begin{frame}
    \titlepage
\end{frame}
\slideq{Description topologique des espaces compact}{1/7}
\slider{Si $E$ est un espace topologique, $E$ est compact si \emph{$E$ est séparé}\footnote{seulement en français} et tout recouvrement ouvert de $E$ a un sous-recouvrement fini}{1/7}
\slideq{Théorème de Borel-Lebesgue}{2/7}
\slider{L'espace métrique $\l E,d\r$ est compact si et seulement si de tout recouvrement par des ouverts, on peut extraire un sous-recouvrement fini}{2/7}
\slideq{Théorème de Poincaré}{3/7}
\slider{Si $X$ est compact et $f\!:\!X\to Y$ est bijective et continue alors $f^{-1}\!:\!Y\to X$ est continue}{3/7}
\slideq{Théorème de Heine}{4/7}
\slider{Si $X$ est compact et $f\!:\!X\to Y$ est continue alors $f$ est uniformément continue}{4/7}
\slideq{Définition équivalente à la compacité}{5/7}
\slider{Complet et précompact}{5/7}
\slideq{Théorème de Riesz}{6/7}
\slider{La boule unité d'un ev $E$ est compacte si et seulement si $E$ est de dimension finie}{6/7}
\slideq{Espace précompact}{7/7}
\slider{Espace $X$ pour lequel pour tout $\varepsilon>0$, $X$ est recouvrable par un nombre fini de boules de rayon $\varepsilon$}{7/7}
\end{document}