\documentclass[14pt,usepdftitle=false,aspectratio=169]{beamer}
\usepackage{preambule}
\setbeamercolor{structure}{fg=black}
\usepackage{al}\newcommand{\appl}[5]{\begin{array}[t]{@{}r@{}r@{}c@{}l@{}}#1\!:\!{}&#2&{}\longrightarrow{}&#3\\&#4&{}\longmapsto{}&#5\end{array}}\newcommand{\nappl}[4]{\begin{array}{@{}r@{}c@{}l@{}}#1&{}\longrightarrow{}&#2\\#3&{}\longmapsto{}&#4\end{array}}
\hypersetup{pdftitle=Algèbre 1 -- Forme linéaire et dualité}
\title{Algèbre 1\\\emph{Forme linéaire et dualité}}
\author{}
\date{}
\begin{document}
\begin{frame}
    \titlepage
\end{frame}
\slideq{Propriété de $\appl{\tau}{E}{E^{**}}{x}{\l\nappl{E^*}{\Bbbk}{l}{l\l x\r}\r}$}{1/5}
\slider{$\tau$ est un isomorphisme en dimension finie}{1/5}
\slideq{Théorème du rang}{2/5}
\slider{Si $S$ est un supplémentaire de $F$ dans $E$, alors $S$ est un système de représentants de $E/F$ et $\pi_S\!:\!S\to E/F$ est un isomorphisme}{2/5}
\slideq{Forme linéaire sur $\Bbbk$ où $E$ est un $\Bbbk$-ev}{3/5}
\slider{Application linéaire $l\!:\!E\to\Bbbk$\linebreak L'ensemble des formes linéaires est le dual de $E$ noté $E^*$}{3/5}
\slideq{Première forme coordonnée}{4/5}
\slider{Forme linéaire sur $E$ de base $\l e_1,\cdots,e_n\r$ vérifiant $e_i^*\l e_j\r=\delta_{i,j}$}{4/5}
\slideq{Propriétés sur les bases de $E^*$}{5/5}
\slider{Pour toute base $\mathcal B'$ de $E^*$, il existe une base $\mathcal B$ de $E$ telle que $\mathcal B'=\mathcal B^*$}{5/5}
\end{document}