\documentclass[14pt,usepdftitle=false,aspectratio=169]{beamer}
\usepackage{preambule}
\setbeamercolor{structure}{fg=black}
\let\Pi\varPi\usepackage{tikz-cd}\tikzcdset{every arrow/.append style={line cap=round}}\DeclareMathOperator{\pr}{pr}\let\phi\varphi\DeclareMathOperator{\oldgal}{Gal}\newcommand{\gal}[1]{\oldgal\l#1\r}\newcounter{footnotemarkcounter}\setcounter{footnotemarkcounter}0\newcounter{footnotetextcounter}\setcounter{footnotetextcounter}0\renewcommand{\footnotemark}{\stepcounter{footnotemarkcounter}{\textsuperscript{\textit{\oldstylenums{\thefootnotemarkcounter}}}}}\let\oldfootnotetext\footnotetext\renewcommand{\footnotetext}[1]{{\stepcounter{footnotetextcounter}\def\thefootnote{\textit{\oldstylenums{\thefootnotetextcounter}}}\def\thempfootnote{\textit{\oldstylenums{\thefootnotetextcounter}}}\oldfootnotetext{#1}}}\renewcommand{\footnote}{\footnotemark\footnotetext}\let\tilde\widetilde\let\ge\geqslant\let\le\leqslant
\hypersetup{pdftitle=Groupe fondamental et revêtement -- Revêtements}
\title{Groupe fondamental et revêtement\\\emph{Revêtements}}
\author{}
\date{}
\begin{document}
\begin{frame}
    \titlepage
\end{frame}
\slideq{CNS pour que $G\curvearrowright X$ soit proprement discontinue}{1/18}
\slider{$G$ est discret, $X$ est localement compact et $G\curvearrowright X$ est libre et propre}{1/18}
\slideq{Propriétés de $\gal p$ si $p\!:\!X\to B$ est un revêtement galoisien et $X$ est connexe}{2/18}
\slider{$\gal p\curvearrowright X$ est proprement discontinue et $\gal p\backslash p$ et $B$ sont homéomorphes}{2/18}
\slideq{Revêtement associé à une action proprement discontinue}{3/18}
\slider{Si $G\curvearrowright X$ par homéomorphismes est proprement discontinue, $G$ est discret et $X$ est un espace topologique alors $\pi\!:\!X\to G\backslash X$\footnote{$G\backslash X$ désigne les classes à gauche par l'action $G\curvearrowright X$} est un revêtement dont le groupe de Galois contient $G$}{3/18}
\slideq{Factorisation par un revêtement}{4/18}
\slider{Si on a le diagramme suivant qui commute\linebreak\begin{tikzcd}[column sep=1em,row sep=0.8em,ampersand replacement=\&]\&X\arrow[dl]\arrow[dr,"p"]\&\\X/p\arrow[rr,"\phi"]\&\&B\end{tikzcd}\linebreak Alors $\phi$ est un homéomorphisme}{4/18}
\slideq{Espace simplement connexe}{5/18}
\slider{Un espace connexe est simplement connexe si son seul revêtement connexe est l'identité\linebreak Le seul revêtement simplement connexe est le revêtement universel}{5/18}
\slideq{Action proprement discontinue\linebreak $G\curvearrowright X$}{6/18}
\slider{Pour tout $x\in X$, il existe un voisinage ouvert $U$ de $x$ tel que pour tout $g\neq1$, $g\l U\r\cap U=\varnothing$}{6/18}
\slideq{Revêtement universel}{7/18}
\slider{Si $X$ est tel que tout $x\in X$ admet un voisinage trivialisant universel alors il eiste un unique revêtement universel $\tilde p$ à homéomorphisme près tel que, pour tout revêtement $p$, $\tilde p\ge p$}{7/18}
\slideq{Morphisme de revêtement}{8/18}
\slider{$\phi\!:\!X\to X'$ tel que le diagramme suivant commute\linebreak\begin{tikzcd}[column sep=1em,row sep=0.8em,ampersand replacement=\&]X\arrow[dr,"p",swap]\arrow[rr,"\phi"]\&\&X'\arrow[dl,"p'"]\\\&B\&\end{tikzcd}}{8/18}
\slideq{$p\ge q$ pour $p$ et $q$ des revêtements au dessus de $\l B,b\r$}{9/18}
\slider{Il existe un homéomorphisme $\phi$ tel qye le diagramme suivant commute\linebreak\begin{tikzcd}[column sep=1em,row sep=0.8em,ampersand replacement=\&]\l X,x\r\arrow[dr,"p",swap]\arrow[rr,"\phi"]\&\&\l Y,y\r\arrow[dl,"q"]\\\&B\&\end{tikzcd}\linebreak\linebreak$\ge$ est un ordre total}{9/18}
\slideq{Propriété locale d'un revêtement}{10/18}
\slider{Un revêtement est un homéomorphisme local}{10/18}
\slideq{Revêtement trivial}{11/18}
\slider{Un revêtement pour lequel $V=B$ convient}{11/18}
\slideq{Action propre\linebreak $G\curvearrowright X$}{12/18}
\slider{Pour tout compact de $X$, $\left\{g\in G,g\l K\r\cap K\neq\varnothing\right\}$ est fini}{12/18}
\slideq{Terminologie associée aux revêtements}{13/18}
\slider{$X$: espace total\linebreak$B$: base\linebreak$p$: revêtement\linebreak$V$: voisinage distingé (de $y$) ou assiette\linebreak$h$: trivialisation locale\linebreak$p^{-1}\l y\r$: fibre de $y$ ou pile d'assiettes}{13/18}
\slideq{Groupe des automorphismes de $p$}{14/18}
\slider{$\gal p:=\left\{\phi\text{ homéomorphismes},p\circ\phi=p\right\}$}{14/18}
\slideq{Revêtement d'un espace topologique $B$\linebreak Définition avec l'espace discret}{15/18}
\slider{$p\!:\!X\to B$ continue avec $X$ un espace topologique tel que pour tout $y\in B$, il existe un voisinage ouvert $V$ de $y$, un espace discret $D$ et $h\!:\!V\times D\to p^{-1}\l V\r$ tels que le diagramme suivant commute\linebreak\begin{tikzcd}[column sep=1em,row sep=0.8em,ampersand replacement=\&]V\times D\arrow[dr,"\pr_1",swap]\arrow[rr,"h"]\&\&p^{-1}\l V\r\arrow[dl,"p"]\\\&V\&\end{tikzcd}}{15/18}
\slideq{$\Pi_1\l B,b\r$}{16/18}
\slider{L'ensemble des groupes d'automorphismes non pointés du revêtement universel de $\l B,b\r$}{16/18}
\slideq{Revêtement d'un espace topologique $B$\linebreak Définition avec l'espace discret}{17/18}
\slider{$p\!:\!X\to B$ continue avec $X$ un espace topologique tel que pour tout $y\in B$, il existe un voisinage ouvert $V$ de $y$, tel que $p^{-1}\l V\r$ est une réunion disjointe d'ouverts de $X$ qui s'envoient chacun homéomorphiquement sur $V$ via $p$}{17/18}
\slideq{Revêtement galoisien}{18/18}
\slider{$p$ est galoisien si pour tout $x\in B$, $\gal p\curvearrowright p^{-1}\l x\r$ transitivement\linebreak Si cette propriété est vérifiée pour un $x\in B$ alors elle est vérifiée pour tout $x\in B$}{18/18}
\end{document}