\documentclass[14pt,usepdftitle=false,aspectratio=169]{beamer}
\usepackage{preambule}
\setbeamercolor{structure}{fg=black}
\usepackage{analyse}\usepackage{complexes}\usepackage{usuelles}
\hypersetup{pdftitle=Analyse complexe -- Fonctions holomorphes}
\title{Analyse complexe\\\emph{Fonctions holomorphes}}
\author{}
\date{}
\begin{document}
\begin{frame}
    \titlepage
\end{frame}
\slideq{$f$ est analytique sur $U$ un ouvert de $\mathbb C$}{1/7}
\slider{Pour tout $z\in U$, $f$ est développable en série entière en $z$}{1/7}
\slideq{$f$ est holomorphe sur $U$ un ouvert de $\mathbb C$}{2/7}
\slider{$\forall z_0\in U$, $\lim[z\to z_0]{\frac{f\l z\r-f\l z_0\r}{z-z_0}}=f'\l z_0\r$ existe}{2/7}
\slideq{Lemme de Hadamard pour les séries entières}{3/7}
\slider{$R^{-1}=\limsup\limits_{n\in\mathbb N}\l\left|a_n\right|^{\oldfrac1n}\r$}{3/7}
\slideq{Condition de Cauchy-Riemann}{4/7}
\slider{$f\l x+\i y\r=P\l x,y\r+\i Q\l x,y\r$ est holomorphe si et seulement si $\pder{P}{}=\pder[][y]{Q}{}$ et $\pder[][y]{P}{}=-\pder{Q}{}$}{4/7}
\slideq{$\pder[][\bar z]{f\l x+\i y\r}{}$}{5/7}
\slider{$\frac12\l\pder{f\l x+\i y\r}{}+\i\pder[][y]{f\l x+\i y\r}{}\r$}{5/7}
\slideq{$\pder[][z]{f\l x+\i y\r}{}$}{6/7}
\slider{$\frac12\l\pder{f\l x+\i y\r}{}-\i\pder[][y]{f\l x+\i y\r}{}\r$}{6/7}
\slideq{Théorème d'inversion locale pour une fonction holomorphe}{7/7}
\slider{Si $f$ est holomorphe et de classe $\mathcal C^1$ sur $U$ tel que $f\l z_0\r\neq 0$ alors il existe $V\in\mathcal V\l z_0\r$ et $W\in\mathcal V\l f\l z_0\r\r$ tels que $f$ induit un biholomorphisme de $V$ dans $W$ (ie bijectif, holomorphe de réciproque holomorphe)}{7/7}
\end{document}