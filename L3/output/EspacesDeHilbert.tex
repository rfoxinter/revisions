\documentclass[14pt,usepdftitle=false,aspectratio=169]{beamer}
\usepackage{preambule}
\setbeamercolor{structure}{fg=black}
\usepackage{topologie}\DeclareMathOperator{\oldpr}{pr}\newcommand{\pr}[2]{\oldpr_{#1}\l#2\r}\usepackage{bigoperators}
\hypersetup{pdftitle=Topologie et calcul différentiel -- Espaces de Hilbert}
\title{Topologie et calcul différentiel\\\emph{Espaces de Hilbert}}
\author{}
\date{}
\begin{document}
\begin{frame}
    \titlepage
\end{frame}
\slideq{Lien entre $V$ et $E$ dans le cas où $V$ est un sev fermé de $E$}{1/5}
\slider{$E=V\varoplus V^\bot$}{1/5}
\slideq{Espace de Hilbert sur $\mathbb R$}{2/5}
\slider{Espace de Banach dont la norme provient d'un produit scalaire}{2/5}
\slideq{Projection orthogonale sur un convexe fermé}{3/5}
\slider{Si $C$ est un convexe fermé alors il existe un unique $c\in C$ tel que $d\l x,C\r=\nrm{x-c}$}{3/5}
\slideq{Espace en bijection isomorphique avec un espace de Hilbert}{4/5}
\slider{Si l'espace est de dimension finie alors in est en bijection isomorphe avec $\mathbb R^n$ muni de la norme euclidienne\linebreak Si l'espace est de dimension infinie alors in est en bijection isomorphe avec $l^2\l\mathbb R\r$ muni de la norme euclidienne}{4/5}
\slideq{$\pr Vx$ pour $V$ de dimension finie}{5/5}
\slider{Si $\l v_1,\cdots,v_n\r$ est une bon de $V$\linebreak$\pr Vx=\sum{k=1}{n}{\psc x{v_i}v_i}$}{5/5}
\end{document}