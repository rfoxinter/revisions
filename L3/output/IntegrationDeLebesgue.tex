\documentclass[14pt,usepdftitle=false,aspectratio=169]{beamer}
\usepackage{preambule}
\setbeamercolor{structure}{fg=black}
\usepackage{bigoperators,analyse}\usepackage{dsft}\usepackage{structures}
\hypersetup{pdftitle=Intégration et théorie de la mesure -- Intégration de Lebesgue}
\title{Intégration et théorie de la mesure\\\emph{Intégration de Lebesgue}}
\author{}
\date{}
\begin{document}
\begin{frame}
    \titlepage
\end{frame}
\slideq{Intégrale d'une fonction étagée}{1/3}
\slider{\toggleanalysepar$\int[\mu]{f}=\sum{\lambda\in\im f}{}{\lambda\mu\l\left\{f=\lambda_i\right\}\r}$\toggleanalysepar}{1/3}
\slideq{Lien entre fonction mesurable et fonction étagée}{2/3}
\slider{Toute fonction mesurable est limite simple de fonctions mesurables croissantes}{2/3}
\slideq{$f$ est étagée}{3/3}
\slider{$f=\sum{i=1}{n}{\lambda_i\1{\left\{f=\lambda_i\right\}}}$}{3/3}
\end{document}