\documentclass[14pt,usepdftitle=false,aspectratio=169]{beamer}
\usepackage{preambule}
\setbeamercolor{structure}{fg=black}
\usepackage{topologie}\usepackage{usuelles}
\hypersetup{pdftitle=Topologie et calcul différentiel -- Complétude}
\title{Topologie et calcul différentiel\\\emph{Complétude}}
\author{}
\date{}
\begin{document}
\begin{frame}
    \titlepage
\end{frame}
\slideq{Prolongement de fonctions à image dans un espace complet}{1/6}
\slider{Si $P\subset X$, $Y$ est complet et $f\!:\!P\to Y$ est une fonction uniformément continue sur $P$ alors $f$ se prolonge en une fonction $\widetilde f\!:\!\overline P\to Y$ uniformément continue}{1/6}
\slideq{Completude de l'espace des fonctions continues}{2/6}
\slider{Si $\l X,d\r$ est un espace métrique et $E$ l'espace des fonctions continues bornées sur $X$ à valeurs dans $\mathbb R$, muni de la norme $\anrm[X]{f}=\oldsup\limits_{x\in X}\l\left|f\l x\r\right|\r$ est complet}{2/6}
\slideq{Théorème du point fixe de Banach}{3/6}
\slider{Si $\l E,d\r$ est un espace métrique complet et $f\!:\!E\to E$ est une fonction $\lambda$-contractante, $\lambda<1$ alors $f$ admet un unique point fixe qui est la limite des suites définies par $x_0\in E$ et $x_{n+1}=f\l x_n\r$}{3/6}
\slideq{Complété d'un espace métrique}{4/6}
\slider{Si $\l X,d\r$ est un espace métrique alors in existe un espace complet $\l\widehat X,d\r$ (unique à isométrie près) et une isométrie $i\!:\!X\to\widehat X$ tels que $i\l X\r$ est dense dans $\widehat X$\linebreak Toute fonction $f\!:\!X\to  Y$ uniformément continue se prolonge alorsen une unique fonction uniformément continue $\widetilde f\!:\!\widehat X\to Y$}{4/6}
\slideq{$\l E,d\r$ est complet}{5/6}
\slider{Toute suite de Cauchy de $E$ coverge}{5/6}
\slideq{Espace de Banach}{6/6}
\slider{Espace vectoriel normé complet}{6/6}
\end{document}