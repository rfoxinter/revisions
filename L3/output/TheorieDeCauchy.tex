\documentclass[14pt,usepdftitle=false,aspectratio=169]{beamer}
\usepackage{preambule}
\setbeamercolor{structure}{fg=black}
\usepackage{analyse}\toggleanalysepar\usepackage{complexes}\let\phi\varphi\usepackage{usuelles}
\hypersetup{pdftitle=Analyse complexe -- Théorie de Cauchy}
\title{Analyse complexe\\\emph{Théorie de Cauchy}}
\author{}
\date{}
\begin{document}
\begin{frame}
    \titlepage
\end{frame}
\slideq{Primitive de fonctions holomorphe sur un ouvert simplement connexe $U$ (ie connexe par arcs et tout lacet est homotope à un lacet constant)}{1/21}
\slider{Toute fonction $f$ holomorphe de classe $\mathcal C^1$ admet une primitive holomorphe sur $U$\linebreak Ce résultat est en particulier vrai si $U$ est étoilé}{1/21}
\slideq{Formule de Cauchy}{2/21}
\slider{Si $U$ est un ouvert de $\mathbb C$, $f$ holomorphe sur $U$ et $z_0\in U$ alors pour tout $z\in D\l z_0,r\r$, $f\l z\r=\frac{1}{2\i\pi}\int[w][C\l z_0,r\r]{\frac{f\l w\r}{w-r}}$}{2/21}
\slideq{Invariance par homotopie des intégrales}{3/21}
\slider{Si $f$ est holomorphe et $\Gamma$ de classe $\mathcal C^2$, en posant $\gamma_s=\Gamma\l s,\cdot\r$, si $\gamma_s$ vérifie l'une des deux conditions suivantes, $\forall s\in\left[0,1\right]$, $\gamma_s$ est fermé ou $\gamma_s\l0\r$ et $\gamma_s\l1\r$ sont indépendant de $s$ alors $\int[z][\gamma_0]{f\l z\r}=\int[z][\gamma_1]{f\l z\r}$}{3/21}
\slideq{Propriétés équivalentes à $f=0$ sur $U$ ou $f$ est holomorphe}{4/21}
\slider{L'ensemble des zéros de $f$ possède un point d'accumulation dans $U$\linebreak$\exists z_0\i  U$, $\forall n\in\mathbb N$, $f^{\l n\r}\l z_0\r=0$}{4/21}
\slideq{Théorème de Cauchy}{5/21}
\slider{Pour $U$ un ouvert de $\mathbb C$ et $f\!:\!U\to\mathbb C$, $f$ est holomorphe sur $U$ si et seulement si $f$ est analytique sur $U$}{5/21}
\slideq{Majoration d'une intégrale}{6/21}
\slider{$\left|\int[z][\gamma]{f\l z\r}\right|\leqslant\oldmax\limits_{z\in\gamma\l\left[a,b\right]\r}{\left|f\l z\r\right|}\times\underbrace{\int[t][a][b]{\left|\gamma'\l t\r\right|}}_{\text{longueur de }\gamma}$}{6/21}
\slideq{Intégrale sur $\partial\Gamma$ où $\Gamma\!:\!\left[0,1\right]^2\to U$}{7/21}
\slider{Si $f$ est holomorphe et $\Gamma$ de classe $\mathcal C^2$ alors $\int[z][\partial\Gamma]{f\l z\r}=0$}{7/21}
\slideq{Intgrale de $f$ sur $C\l0,r\r$ holomorphe sur couronne}{8/21}
\slider{$\int[z][C\l0,r\r]{f\l z\r}$ est indépendante de $r$}{8/21}
\slideq{Intégrales sur un lacet}{9/21}
\slider{Si $f$ est holomorphe sur $U$ et $\gamma\l a\r=\gamma\l b\r$, $\int[z][\gamma]{f\l z\r}=0$}{9/21}
\slideq{Concaténation d'intégrales}{10/21}
\slider{Si $c\in\left[a,b\right]$, $\gamma_1=\gamma_{|\left[a,c\right]}$ et $\gamma_2=\gamma_{|\left[c,b\right]}$, $\int[z][\gamma_1]{f\l z\r}+\int[z][\gamma_2]{f\l z\r}=\int[z][\gamma]{f\l z\r}$}{10/21}
\slideq{Intégration sur le chemin opposé}{11/21}
\slider{Si $\gamma^*\l t\r=\gamma\l a+b-t\r$, $\int[z][\gamma^*]{f\l z\r}=-\int[z][\gamma]{f\l z\r}$}{11/21}
\slideq{Théorème de Liouville}{12/21}
\slider{Toute fonction holomorphe bornée sur $\mathbb C$ est constante}{12/21}
\slideq{Principe du prolongement analytique}{13/21}
\slider{Si $f$ et $g$ sont holomorphe sur $U$ et coïncident sur un ouvert de $U$ alors $f=g$ sur $U$\linebreak\linebreak Soit $f$ analytique non identiquement nulle sur $U$, l'ensemble des zéros de $f$ sont isolés}{13/21}
\slideq{Invariance par paramétrage de l'intégrale}{14/21}
\slider{Si $\phi\!:\!\left[a',b'\right]\to\left[a,b\right]$ est croissante et $\mathcal C^1$, en posant $\gamma_0=\gamma\circ\phi$, $\int[z][\gamma_0]{f\l z\r}=\int[z][\gamma]{f\l z\r}$}{14/21}
\slideq{Principe du maximum}{15/21}
\slider{Si $f$ est holomorphe sur $U$ et $\left|f\right|$ admet un maximum sur $U$ alors $f$ est constante dur $U$}{15/21}
\slideq{$\int[z][\gamma]{f\l z\r}$\linebreak$\gamma\!:\!\left[a,b\right]\to\mathbb C$ de classe $\mathcal C^1$ par morceaux}{16/21}
\slider{$\int[t][a][b]{f\l\gamma\l t\r\r\times\gamma'\l t\r}$}{16/21}
\slideq{Coefficients de la série de Taylor d'une fonction holomorphe}{17/21}
\slider{$a_n=\frac{1}{2\i\pi}\int[w][C\l z_0,r\r]{\frac{f\l w\r}{\l w-z_0\r^{n+1}}}$}{17/21}
\slideq{Lien entre fonction holomorphe sur la couronne $A\l R_1,R_2\r$ et série de Laurent}{18/21}
\slider{\togglebigopdisplay\toggleanalysedisplay Les application $\l a_n\r_{n\in\mathbb Z}\mapsto\l z\mapsto\sum{n\in\mathbb Z}{}{a_nz^n}\r$ et $f\mapsto \l\tfrac{1}{2\i\pi}\int[z][C\l0,r\r]{\tfrac{f\l z\r}{z^{n+1}}}\r_{n\in\mathbb N}$ où $\sum{n\in\mathbb N}{}{a_nz^n}$ est holomorphe sur $D\l0,R_1\r$ et $\sum{n\in-\mathbb N^*}{}{a_nz^n}$ est holomorphe sur $D\l0,R_2\r$ sont deux bijections réciproques\togglebigopdisplay\toggleanalysedisplay}{18/21}
\slideq{Théorème de Morera}{19/21}
\slider{Si $U$ est un ouvert de $\mathbb C$ et $f\!:\!U\to\mathbb C$, il y a équivalence entre $f$ est analytique sur $U$ et $f$ est continue et vérifie $\forall\l a,b,c\r\in U^3$ tels que $\Delta\l a,b,c\r\subset U$, $\int[z][{\left[a,b\right]}]{f\l z\r}+\int[z][{\left[b,c\right]}]{f\l z\r}+\int[z][{\left[c,a\right]}]{f\l z\r}=0$}{19/21}
\slideq{Inégalité de Cauchy}{20/21}
\slider{$\left|\frac{f^{\l n\r}\l z_n\r}{n!}\right|\leqslant r^{-n}\oldmax\limits_{\theta\in\left[0,2\pi\right]}\left|f\l z_0+r\e^{\i\theta}\r\right|$}{20/21}
\slideq{Maximum de $f$ continue sur $\bar U$ et holomorphe sur $U$ un ouvert connexe de $\mathbb C$}{21/21}
\slider{$\oldmax\limits_{z\in\bar U}\left|f\l z\r\right|=\oldmax\limits_{z\in\partial U}\left|f\l z\r\right|$}{21/21}
\end{document}