\documentclass[14pt,usepdftitle=false,aspectratio=169]{beamer}
\usepackage{preambule}
\setbeamercolor{structure}{fg=black}
\usepackage{analyse,topologie}\togglebigoppar\toggleanalysepar\newcommand{\lnrm}[2][p]{\anrm[L^{#1}]{#2}}\let\phi\varphi\let\otimes\varotimes
\hypersetup{pdftitle=Intégration et théorie de la mesure -- Espaces \textit{L\raisebox{0.5ex}{\textsuperscript{p}}}}
\title{Intégration et théorie de la mesure\\\emph{Espaces \textit{L\raisebox{0.5ex}{\textsuperscript{p}}}}}
\author{}
\date{}
\begin{document}
\begin{frame}
    \titlepage
\end{frame}
\slideq{$L^p\l\mu\r$}{1/10}
\slider{$\mathcal L^p/{\sim}$ où $f\sim g$ ssi $f=g$ $\mu$-pp}{1/10}
\slideq{$\mathcal L^p\l\mu\r$}{2/10}
\slider{$\left\{f\!:\!X\to\mathbb R\text{ mesurables},\int[\mu]{\left|f\right|^p}<+\infty\right\}$\linebreak$\mathcal L^\infty=\left\{f\!:\!X\to\mathbb R\text{ mesurables},\anrm f<+\infty\right\}$}{2/10}
\slideq{Inégalité de Jensen}{3/10}
\slider{Soit $I\subset\mathbb R$ un intervalle, $\mu$ une mesure de probabilités ($\mu\l X\r=1$ et $\mu\geqslant0$), si $\phi\!:\!I\to\mathbb R$ est convexe et $f\!:\!X\to\mathbb R$ mesurable alors $\phi\l\int[\mu][X]f\r\leqslant\int[\mu][X]{\phi\circ f}$}{3/10}
\slideq{Densité des fonctions en escalier dans $L^p$}{4/10}
\slider{Si $p\in\left[1,+\infty\right]$ alors les fonctions en escalier sont dense dans $L^p$}{4/10}
\slideq{Inégalité de Hölder}{5/10}
\slider{Si $p\in\left[1,+\infty\right]$ et $p'=\frac{p}{p-1}$ alors pour $f$ et $g$ mesurables, $\int{\left|fg\right|}\leqslant\lnrm f\lnrm[p']g$}{5/10}
\slideq{Structures des $L^p$}{6/10}
\slider{$L^p$ sont des espaces de Banach\linebreak$L^2$ est un expace de Hilbert avec $\psc fg=\int{fg}$}{6/10}
\slideq{Densité des fonctions lipschitziennes à support compact dans $L^p$}{7/10}
\slider{Si $\l X,d\r$ est un espace métrique localement compact (pour tout $x\in X$, il existe $O$ ouvert tel que $x\in O$ et $\overline O$ est compact), séparable et $\mu$ de Radon (finie sur tout compact) et $p\in\left[1,+\infty\right[$ alors les fonctions lipschitziennes à support compact de $L^p$ sont dense dans $L^p$}{7/10}
\slideq{$\lnrm f$}{8/10}
\slider{$\l\int[\mu][X]{\left|f\right|^p}\r^{\oldfrac1p}$ si $p\in\mathbb N^*$\linebreak$\lnrm[\infty]f=\inf{\left\{c\geqslant0,\left|f\right|\leqslant c\text{ $\mu$-pp}\right\}}$}{8/10}
\slideq{Inégalité de Minkowski}{9/10}
\slider{Si $f,g\!:\!X\to\mathbb R$ sont mesurables et $p\in\left[1,+\infty\right]$, alors $\lnrm{f+g}\leqslant\lnrm f+\lnrm g$}{9/10}
\slideq{Densité des fonctions lipschitziennes dans $L^p$}{10/10}
\slider{Si $p\in\left[1,+\infty\right[$  et $\mu$ est extérieurement régulière alors les fonctions lipschitziennes de $L^p$ sont dense dans $L^p$}{10/10}
\end{document}