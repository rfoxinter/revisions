\documentclass[14pt,usepdftitle=false,aspectratio=169]{beamer}
\usepackage{preambule}
\setbeamercolor{structure}{fg=black}
\PassOptionsToPackage{nopar}{bigoperators}\usepackage{geometrie,al,analyse}
\hypersetup{pdftitle={Géométrie riemannienne -- Métrique et distance riemannienne, connexion de Levi--Civita}}
\title{Géométrie riemannienne\\\emph{Métrique et distance riemannienne, connexion de Levi--Civita}}
\author{}
\date{}
\begin{document}
\begin{frame}
    \titlepage
\end{frame}
\slideq{Métrique riemannienne sur une variété différentielle $M$}{1/1}
\slider{$g\in\cv M{\symalg{\tg^*M}}$ tel que, pour tout $x\in M$, $g_x$ soit une forme bilinéaire sur $\tg_xM$ est un produit scalaire sur cet espace\linebreak En coordonnées locales, $g=\sum{i=1}{\dim M}{g_{i,j}\dd x_i\dd x_j}$ où $g_{i,j}=g\l\partial_i,\partial_j\r$}{1/1}
\end{document}