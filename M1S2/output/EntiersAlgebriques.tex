\documentclass[14pt,usepdftitle=false,aspectratio=169]{beamer}
\usepackage{preambule}
\setbeamercolor{structure}{fg=black}
\DeclareMathOperator{\olddisc}{disc}\newcommand{\disc}[1]{\olddisc\l#1\r}
\hypersetup{pdftitle={Théorie algébrique des nombres -- Entiers algébriques}}
\title{Théorie algébrique des nombres\\\emph{Entiers algébriques}}
\author{}
\date{}
\begin{document}
\begin{frame}
    \titlepage
\end{frame}
\slideq{Porpriété de $x\in\calO_K$ tel que, pour tout plongement $\sigma$ de $K$ dans $\bbC$, on ait $\left|\sigma\l x\r\right|\leqslant1$}{1/4}
\slider{$x$ est une racine de l'unité}{1/4}
\slideq{CS pour avoir une base entière}{2/4}
\slider{$\disc{\alpha_1,\cdots,\alpha_d}$ est sans facteur carré}{2/4}
\slideq{Base entière de $K$}{3/4}
\slider{Base de $\calO_K$ comme $\bbZ$-module}{3/4}
\slideq{Discriminant de $K$}{4/4}
\slider{Discriminant d'une base entière de $K$}{4/4}
\end{document}