\documentclass[14pt,usepdftitle=false,aspectratio=169]{beamer}
\usepackage{preambule}
\setbeamercolor{structure}{fg=black}
\usepackage{topologie,structures,usuelles}\togglebornelimits
\hypersetup{pdftitle={Théorie spectrale -- Opérateurs autoadjoints}}
\title{Théorie spectrale\\\emph{Opérateurs autoadjoints}}
\author{}
\date{}
\begin{document}
\begin{frame}
    \titlepage
\end{frame}
\slideq{CNS pour que $T$ autoadjoint soit positif}{1/3}
\slider{Un opérateur autoadjoint $T$ est positif si et seulement si $\sigma\l T\r\subset\bbR_+$}{1/3}
\slideq{Propriétés des bornes du spectre d'un opérateur autoadjoint}{2/3}
\slider{Si $H$ est un Hilbert et $T\in\calB\l H\r$ est autoadjoint alors $\sigma\l T\r\neq\varnothing$, $\min{\sigma\l T\r}=\inf[\substack{x\in H\\\nrm x=1}]{\psc{Tx}x}$, $\max{\sigma\l T\r}=\sup[\substack{x\in H\\\nrm x=1}]{\psc{Tx}x}$ et $\max[\lambda\in\sigma\l T\r]{\vala\lambda}=\nrm T$}{2/3}
\slideq{Inégalité de Cauchy-Schwarz pour les opérateurs positifs}{3/3}
\slider{Si $T\in\calB\l H\r$ est un opérateur positif alors pour tout $\l x,y\r\in H^2$, $\vala{\psc{Tx}y}^2\leqslant\psc{Tx}x\psc{Ty}y$}{3/3}
\end{document}