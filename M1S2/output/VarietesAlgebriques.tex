\documentclass[14pt,usepdftitle=false,aspectratio=169]{beamer}
\usepackage{preambule}
\setbeamercolor{structure}{fg=black}
\usepackage[nopar]{bigoperators}\usepackage{polynomes}
\hypersetup{pdftitle={Introduction à la géométrie algébrique -- Variétés algébriques}}
\title{Introduction à la géométrie algébrique\\\emph{Variétés algébriques}}
\author{}
\date{}
\begin{document}
\begin{frame}
    \titlepage
\end{frame}
\slideq{$I\l X\r$\linebreak$X\subset\bbA^n_\bbk$}{1/13}
\slider{$\set{f\in\pol k{X_1,\cdots,X_n},\forall P\in X,f\l P\r=0}$}{1/13}
\slideq{Lien entre $I$ et $V$}{2/13}
\slider{$X\subset V\l I\l X\r\r$ avec égalité si et seulement si $X$ est un ensemble algébrique\linebreak$J\subset I\l V\l J\r\r$}{2/13}
\slideq{$V\l I\r$\linebreak$I\subset\bbk\lc X_1,\cdots,X_n\rc$}{3/13}
\slider{$\set{P\in\bbA^n_\bbk,\forall f\in I,f\l P\r=0}$}{3/13}
\slideq{Propriétés des ensembles algébriques irréductibles}{4/13}
\slider{$X$ est algébrique si et seulement si $I\l X\r$ est premier\linebreak Tout ensemble algébrique se décompose de manière unique en irréductibles}{4/13}
\slideq{$I$ est un idéal radical}{5/13}
\slider{$I=\sqrt I$}{5/13}
\slideq{Correspondances induites par $I$ et $V$}{6/13}
\slider{$\set{J\leqslant\pol k{X_1,\cdots,X_n}}\leftrightarrow\set{X\subset\bbA^n_k}$\linebreak$\set{J\text{ radicaux}}\overset{\scriptscriptstyle\sim}{\leftrightarrow}\set{X\text{ algébriques}}$\linebreak$\set{J\text{ premiers}}\overset{\scriptscriptstyle\sim}{\leftrightarrow}\set{X\text{ algébriques irréductibles}}$}{6/13}
\slideq{Un ensemble algébrique $X\subset\bbA^n_\bbk$ est irréductible}{7/13}
\slider{Il n'existe pas d'ensembles algébriques $X_1\varsubsetneq X$ et $X_2\varsubsetneq X$ tels que $X=X_1\cup X_2$}{7/13}
\slideq{Propriétés de $I$}{8/13}
\slider{$X\subset Y\Rightarrow I\l X\r\supseteq I\l Y\r$}{8/13}
\slideq{Radical d'un idéal $I$}{9/13}
\slider{$\operatorname{rad}\l I\r=\sqrt I=\set{x\in A,\exists n\in\bbN,x^n\in A}$}{9/13}
\slideq{Nullstellensatz pour $I$ et $V$}{10/13}
\slider{Soit $\bbk$ un corps algébriquement clos\linebreak Si $J$ est un idéal de $\pol k{X_1,\cdots,X_n}$ tel que $J\neq\l1\r$ alors $V\l J\r\neq\varnothing$\linebreak Pour tout idéal $J$ de $\pol k{X_1,\cdots,X_n}$, on a $\sqrt J=I\l V\l J\r\r$}{10/13}
\slideq{Topologie de Zariski}{11/13}
\slider{La topologie de Zariski sur $\bbA^n_\bbk$ est la topologie qui a pour fermés les ensembles algébriques}{11/13}
\slideq{Propriétés de $V$}{12/13}
\slider{$V\l\l0\r\r=\bbA^n_\bbk$\linebreak$V\l\bbk\lc X_1,\cdots,X_n\rc\r=\varnothing$\linebreak$I\subset J\Rightarrow V\l Y\r\supseteq V\l Y\r$\linebreak$V\l I_1\cap I_2\r=V\l I_1\r\cup V\l I_2\r$\linebreak$V\l\sum{\lambda\in\varLambda}{}{I_\lambda}\r=\bigcap{\lambda\in\varLambda}{}{V\l I_\lambda\r}$}{12/13}
\slideq{$X\subset\bbA^n_\bbk$ est algébrique}{13/13}
\slider{Il existe un idéal $I$ de $\bbk\lc X_1,\cdots,X_n\rc$ tel que $X=V\l I\r$}{13/13}
\end{document}