\documentclass[14pt,usepdftitle=false,aspectratio=169]{beamer}
\usepackage{preambule}
\setbeamercolor{structure}{fg=black}
\usepackage{polynomes,galois}\usepackage[nopar]{bigoperators}\DeclareMathOperator{\olddisc}{disc}\newcommand{\disc}[1]{\olddisc\l#1\r}
\hypersetup{pdftitle={Théorie algébrique des nombres -- Anneaux d’entiers des corps de nombres}}
\title{Théorie algébrique des nombres\\\emph{Anneaux d'entiers des corps de nombres}}
\author{}
\date{}
\begin{document}
\begin{frame}
    \titlepage
\end{frame}
\slideq{Propriété de $p$ pour un corps de nombres $K$ d'anneaux des entiers $\calO_K$ monogène, engendré par $\alpha$}{1/12}
\slider{Si dans $\bbF_p\lc X\rc$, $\overline{\varPi_\alpha}=\overline{Q_1}^{e_1}\cdots\overline{Q_r}^{e_r}$ avec les $\overline{Q_i}$ irréductibles unitaires distincts et $Q_i$ un relevé unitaire de $\overline{Q_i}$ dans $\bbZ\lc X\rc$ alors $\frakp_i=\l p,Q_i\l\alpha\r\r$ est premier, les $\frakp_i$ sont dictincts et $\l p\r=\frakp_1^{e_1}\cdots\frakp_r^{e_r}$, et le degré d'inertie de $\frakp_i$ est $f_i=\deg{Q_i}$}{1/12}
\slideq{$\frakp\cap\bbZ$ pour $p$ un idéal premier de $\calO_K$}{2/12}
\slider{$p\bbZ$ pour $p$ premier}{2/12}
\slideq{$p\in\bbZ$ premier se ramifie dans $K$}{3/12}
\slider{Il existe $\frakp$ au dessus de $p$ avec $e>1$}{3/12}
\slideq{Indice de ramification de $\frakp$ dans $p$}{4/12}
\slider{Puissance $e$ telle que $\l p\r=\frakp\frakq$ où $\nu_\frakp\l\frakq\r=0$}{4/12}
\slideq{$\calO_K$ est monogène}{5/12}
\slider{$\calO_K=\bbZ\lc\alpha\rc$ pour un certain $\alpha\in\calO_K$}{5/12}
\slideq{$p\in\bbZ$ premier est complètement décomposé dans $K$}{6/12}
\slider{Pour tout $\frakp$ au dessus de $p$, $e=1$}{6/12}
\slideq{Degré d'inertie de $\frakp$}{7/12}
\slider{$f\in\bbN$ tel que $\left|\calO_K/\frakp\right|=p^f$}{7/12}
\slideq{$\frakp$ est au dessus de $p$}{8/12}
\slider{$\frakp\cap\bbZ=p\bbZ$}{8/12}
\slideq{Décomposition du degré d'une extension en fonction de la décomposition d'un premier de $\bbZ$}{9/12}
\slider{Si $\l p\r=\frakp_1^{e_1}\cdots\frakp_r^{e_r}$ alors $\sum{i=1}r{e_if_i}=\edeg K\bbQ$}{9/12}
\slideq{CNS pour que $p$ se ramifie dans $K$}{10/12}
\slider{$p\mid\disc K$}{10/12}
\slideq{Propriétés de $\calO_K/\frakp^i$}{11/12}
\slider{$\left|\calO_K/\frakp^i\right|=p^{fi}$\linebreak Les idéaux de $\calO_K/\frakp^i$ sont les $\frakp_j/\frakp^i$, $j\leqslant i$\linebreak et $\frakp/\frakp^i$ est principal}{11/12}
\slideq{$p\in\bbZ$ premier est inerte dans $K$}{12/12}
\slider{$\frakp=\l p\r$ est le seul idéal au dessus de $p$, ie $e=1$, $r=1$ et $f=d$}{12/12}
\end{document}