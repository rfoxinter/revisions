\documentclass[14pt,usepdftitle=false,aspectratio=169]{beamer}
\usepackage{preambule}
\setbeamercolor{structure}{fg=black}
\usepackage{probas}
\hypersetup{pdftitle={Processus stochastiques -- Mouvement brownien}}
\title{Processus stochastiques\\\emph{Mouvement brownien}}
\author{}
\date{}
\begin{document}
\begin{frame}
    \titlepage
\end{frame}
\slideq{Processus stochastique gaussien}{1/3}
\slider{$\l X_t\r_{t\in T}$ est un processus stochastique gaussien si $X_t$ est à valeurs dans $\bbR^d$ muni des boréliens et tel que, pour tout $p\in\bbN^*$, tout $\l t_1,\cdots,t_p\r\in T^p$ deux à deux distincts, on a que $\l X_{t_1},\cdots,X_{t_p}\r$ est un vecteur gaussien}{1/3}
\slideq{CNS pour que $\l B_t\r_{t\in I}$ soit un mouvement prébrownien}{2/3}
\slider{$\l B_t\r_{t\in I}$ est un processus stochastique gaussien centré tel que $\esp{B_sB_t}=s\wedge t$}{2/3}
\slideq{$\l B_t\r_{t\in I}$ est un mouvement prébrownien\linebreak$I\subset\bbR_+$ un intevalle}{3/3}
\slider{Pour tout $p\in\bbN^*$ et tout $t_1<\cdots<t_p\in I$, $B_{t_1}$, $B_{t_2}-B_{t_1}$, \dots, $B_{t_p}-B_{t_{p-1}}$ sont des incréments indépendants avec $B_{t_1}\sim\normal0{t_1}$, $B_{t_i}-B_{t_{i-1}}\sim\normal0{t_i-t_{i-1}}$}{3/3}
\end{document}