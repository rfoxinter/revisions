\documentclass[14pt,usepdftitle=false,aspectratio=169]{beamer}
\usepackage{preambule}
\setbeamercolor{structure}{fg=black}
\PassOptionsToPackage{nopar}{bigoperators}\usepackage[nopar]{analyse}\usepackage{geometrie,al,topologie,usuelles}\let\oldnabla\nabla\renewcommand{\nabla}{{\slantbox[.25]{$\oldnabla$}}\!}
\hypersetup{pdftitle={Géométrie riemannienne -- Métrique et distance riemannienne, connexion de Levi--Civita}}
\title{Géométrie riemannienne\\\emph{Métrique et distance riemannienne, connexion de Levi--Civita}}
\author{}
\date{}
\begin{document}
\begin{frame}
    \titlepage
\end{frame}
\slideq{Métrique riemannienne sur une variété différentielle $M$}{1/6}
\slider{$g\in\cv M{\symalg{\tg^*M}}$ tel que, pour tout $x\in M$, $g_x$ soit une forme bilinéaire sur $\tg_xM$ est un produit scalaire sur cet espace\linebreak En coordonnées locales, $g=\sum{i=1}{\dim M}{g_{i,j}\dd x_i\dd x_j}$ où $g_{i,j}=g\l\partial_i,\partial_j\r$}{1/6}
\slideq{CNS pour que le difféomorphisme $\phi\!:\!\l M,g\r\to\l N,h\r$ soit une isométrie riemannienne}{2/6}
\slider{$\phi$ est une isométrie $\l M,d_g\r\to\l N,d_g\r$}{2/6}
\slideq{Distance riemannienne associée à $g$}{3/6}
\slider{$d_g\!:\!M\times M\to\bbR$ définir par $d_g\l x,y\r=\inf[\gamma:a\rightsquigarrow b]{l_g\l\gamma\r}$\linebreak C'est une distance qui induit la topologie de $M$ et $d_g$ détermine $g$}{3/6}
\slideq{$\phi\!:\!\l M,g\r\to\l N,h\r$ est une isométrie riemannienne}{4/6}
\slider{$\phi^*h=g$}{4/6}
\slideq{Longueur d'un chemin $\gamma\!:\!\lc a,b\rc\to M$}{5/6}
\slider{$\int[t][a][b]{\nrm[g]{\gamma'\l t\r}}$\linebreak$\gamma'\l t\r=\tg_t\gamma\l\partial_t\r$}{5/6}
\slideq{Connexion de Levi--Civita}{6/6}
\slider{Il existe une unique application $\bbR$-bilinéaire $\nabla\!:\!\calX\l M\r\times\calX\l M\r\to\calX M=\cv M{\tg M}$ telle que\linebreak$X_*g\l Y,Z\r=g\l\nabla_XY,Z\r+g\l Y,\nabla_XZ\r$\linebreak$\nabla_XY-\nabla_YX=\lc X,Y\rc$\linebreak Ceci vérifie alors $\nabla_{fX}Y=f\nabla_XY$ et $\nabla_X\l fY\r=f\nabla_XY+\l X_*f\r Y$}{6/6}
\end{document}