\documentclass[14pt,usepdftitle=false,aspectratio=169]{beamer}
\usepackage{preambule}
\setbeamercolor{structure}{fg=black}
\usepackage{topologie,structures}
\hypersetup{pdftitle={Théorie spectrale -- Opérateurs normaux}}
\title{Théorie spectrale\\\emph{Opérateurs normaux}}
\author{}
\date{}
\begin{document}
\begin{frame}
    \titlepage
\end{frame}
\slideq{$T\in\calB\l H\r$ est unitaire}{1/9}
\slider{$T^*=T^{-1}$}{1/9}
\slideq{$T\in\calB\l H\r$ est normal}{2/9}
\slider{$TT^*=T^*T$}{2/9}
\slideq{Théorèmes des valeurs propres approchées}{3/9}
\slider{Soientt $H$ un Hilbert et $T\in\calB\l H\r$ un opérateur normal, soit $\lambda\in\bbK$, alors $\lambda\in\sigma\l T\r$ si et seulement s'il existe une suite de vecteurs unitaires $\l x_n\r$ telle que $\nrm{Tx_n-\lambda x_n}\xrightarrow[n\to+\infty]{}0$\linebreak Si $T^*=T$ alors $\sigma\l T\r\subset\bbR$\linebreak Si $T$ est positif, $\sigma\l T\r\subset\bbR_+$\linebreak Si $T$ est unitaire alors $\sigma\l T\r\subset\bbU$}{3/9}
\slideq{Lien entre adjoint et orthogonal de $T\in\calB\l H\r$}{4/9}
\slider{Soit $K\subset H$ fermé\linebreak Si $K$ est stable par $T$, alors $K^\bot$ est stable par $T^*$\linebreak Si $K$ et $K^\bot$ sont stables par $T$ alors ils le sont aussi par $T^*$ et $T^*_{|K}=\l T_{|K}\r^*$\linebreak Si $T=T^*$ et $K$ est stable par $T$ alors $K^\bot$ est stable par $T$ et $T_{|K}$ et $T_{|K^\bot}$ sont autoadjoints}{4/9}
\slideq{Propriétés sur les normes de $T\in\calB\l H\r$}{5/9}
\slider{$\nrm{T^*}=\nrm T$\linebreak$\nrm{T^*T}=\nrm T^2$\linebreak Si $T$ est normal, $\nrm{T^n}=\nrm T^n$}{5/9}
\slideq{Lien entre le noyau et l'image d'un opérateur normal, et l'inversibilité de $T\in\calB\l H\r$ normal}{6/9}
\slider{$\ker T=\ker{T^*}=\im T^\perp=\im{T^*}^\perp$\linebreak De plus $T$ est inversible si et seulement si $T$ est borné inférieurement}{6/9}
\slideq{$T\in\calB\l H\r$ est positif}{7/9}
\slider{$T=T^*$ et $\psc{Tx}x\ge0$}{7/9}
\slideq{Orthogonalité des sous-espaces propres}{8/9}
\slider{Soient $H$ un Hilbert et $T\in\calB\l H\r$ un opérateur normal, alors les sous-espaces propres $\ker{T-\lambda\operatorname{id}}$ sont deux à deux orthogonaux\linebreak En particulier, si $H$ est séparable alors $\sigma_p\l T\r$ est au plus dénombrable}{8/9}
\slideq{$T\in\calB\l H\r$ est autoadjoint}{9/9}
\slider{$T=T^*$}{9/9}
\end{document}