\documentclass[14pt,usepdftitle=false,aspectratio=169]{beamer}
\usepackage{preambule}
\setbeamercolor{structure}{fg=black}
\usepackage{topologie,structures}
\hypersetup{pdftitle={Théorie spectrale -- Opérateurs normaux}}
\title{Théorie spectrale\\\emph{Opérateurs normaux}}
\author{}
\date{}
\begin{document}
\begin{frame}
    \titlepage
\end{frame}
\slideq{Lien entre le noyau et l'image d'un opérateur normal, et inversibilité de $T\in\calB\l H\r$ normal}{1/6}
\slider{$\ker T=\ker{T^*}=\im T^\perp=\im{T^*}^\perp$\linebreak De plus $T$ est inversible si et seulement si $T$ est borné inférieurement}{1/6}
\slideq{$T\in\calB\l H\r$ est autoadjoint}{2/6}
\slider{$T=T^*$}{2/6}
\slideq{$T\in\calB\l H\r$ est normal}{3/6}
\slider{$TT^*=T^*T$}{3/6}
\slideq{Propriétés sur les normes de $T\in\calB\l H\r$}{4/6}
\slider{$\nrm{T^*}=\nrm T$\linebreak$\nrm{T^*T}=\nrm T^2$\linebreak Si $T$ est normal, $\nrm{T^n}=\nrm T^n$}{4/6}
\slideq{$T\in\calB\l H\r$ est unitaire}{5/6}
\slider{$T^*=T^{-1}$}{5/6}
\slideq{$T\in\calB\l H\r$ est positif}{6/6}
\slider{$T=T^*$ et $\psc{Tx}x\ge0$}{6/6}
\end{document}