\documentclass[14pt,usepdftitle=false,aspectratio=169]{beamer}
\usepackage{preambule}
\setbeamercolor{structure}{fg=black}
\usepackage[nopar]{analyse}\usepackage{structures}\def\rmsigma{\operatorname{\sigma}}
\hypersetup{pdftitle={Théorie spectrale -- Spectre d’un opérateur}}
\title{Théorie spectrale\\\emph{Spectre d'un opérateur}}
\author{}
\date{}
\begin{document}
\begin{frame}
    \titlepage
\end{frame}
\slideq{Propriétés topologiques de $\calB\l E\r^\times$}{1/4}
\slider{C'est un ouvert de $\calB\l E\r$}{1/4}
\slideq{Propriétés topologiques de $\rmsigma\l T\r$}{2/4}
\slider{C'est un compact contenu dans la boule de rayon $\anrm[\null]T$}{2/4}
\slideq{$\lambda$ est une valeur spectrale de $T\in\calB\l E\r$}{3/4}
\slider{$T-\lambda\id\notin\calB\l E\r^\times$\linebreak L'ensemble des valeurs spectrales de $T$ est noté $\rmsigma\l T\r$}{3/4}
\slideq{$\lambda$ est une valeur propre de $T\in\calB\l E\r$}{4/4}
\slider{$\ker{T-\lambda\id}\neq\set0$\linebreak L'ensemble des valeurs spectrales de $T$ est noté $\rmsigma_p\l T\r$}{4/4}
\end{document}