\documentclass[14pt,usepdftitle=false,aspectratio=169]{beamer}
\usepackage{preambule}
\setbeamercolor{structure}{fg=black}
\DeclareMathOperator{\oldFrac}{Frac}\newcommand{\Frac}[1]{\oldFrac\l#1\r}
\hypersetup{pdftitle={Théorie algébrique des nombres -- Anneaux de Dedekind}}
\title{Théorie algébrique des nombres\\\emph{Anneaux de Dedekind}}
\author{}
\date{}
\begin{document}
\begin{frame}
    \titlepage
\end{frame}
\slideq{Propriétés algébriques de $\calO_K$}{1/5}
\slider{C'est un anneau de Dedekind}{1/5}
\slideq{CNS pour qu'un sous-$A$-module $M$ de $\Frac{A}$ soit de type fini}{2/5}
\slider{Il existe $x\in A$ tel que $xM\subset A$}{2/5}
\slideq{Structure de l'ensemble des idéaux fractionnaires}{3/5}
\slider{C'est un groupe de neutre $A$ pour la multiplication}{3/5}
\slideq{$A$ est un anneau de Dedekind}{4/5}
\slider{$A$ est commutatif, intègre et noethérien\linebreak Tout idéal premier non nul est maximal\linebreak $A$ est intégralement clos}{4/5}
\slideq{Idéal fractionnaire}{5/5}
\slider{Sous-$A$-module de $\Frac A$ non vide et de type fini}{5/5}
\end{document}