\documentclass[14pt,usepdftitle=false,aspectratio=169]{beamer}
\usepackage{preambule}
\setbeamercolor{structure}{fg=black}
\DeclareMathOperator{\oldFrac}{Frac}\newcommand{\Frac}[1]{\oldFrac\l#1\r}
\hypersetup{pdftitle={Théorie algébrique des nombres -- Anneaux de Dedekind}}
\title{Théorie algébrique des nombres\\\emph{Anneaux de Dedekind}}
\author{}
\date{}
\begin{document}
\begin{frame}
    \titlepage
\end{frame}
\slideq{$A$ est un anneau de Dedekind}{1/9}
\slider{$A$ est commutatif, intègre et noethérien\linebreak Tout idéal premier non nul est maximal\linebreak $A$ est intégralement clos}{1/9}
\slideq{CNS pour qu'un sous-$A$-module $M$ de $\Frac{A}$ soit de type fini}{2/9}
\slider{Il existe $x\in A$ tel que $xM\subset A$}{2/9}
\slideq{Structure de l'ensemble des idéaux fractionnaires}{3/9}
\slider{C'est un groupe de neutre $A$ pour la multiplication}{3/9}
\slideq{Propriétés des anneaux de Dedekind factoriel}{4/9}
\slider{Un anneau de Dedekind factoriel est principal}{4/9}
\slideq{Propriétés algébriques de $\calO_K$}{5/9}
\slider{C'est un anneau de Dedekind}{5/9}
\slideq{Théorème de factorisation des idéaux d'un anneau de Dedekind}{6/9}
\slider{Tout idéal de $A$ se factorise de manière unique en produit d'idéaux premiers}{6/9}
\slideq{Idéal fractionnaire}{7/9}
\slider{Sous-$A$-module de $\Frac A$ non vide et de type fini}{7/9}
\slideq{Système de générateurs des idéaux fractionnaires d'un anneau de Dedekind}{8/9}
\slider{Les idéaux premiers de $A$ engendrent le groupe des idéaux fractionnaires\linebreak Tout idéal fractionnaire de $A$ s'écrit de manière unique comme produit d'idéaux premiers de $A$ (avec des puissances négatives)}{8/9}
\slideq{Lemme de factorisation des idéaux d'un anneau de Dedekind}{9/9}
\slider{Si $I$ et $X$ sont deux idéaux de $A$ alors il existe un unique idéal $J$ de $A$ tel que $I=XJ$ si et seulement si $I\subset X$\linebreak De plus, $I=A$ si et seulement si $I=X$}{9/9}
\end{document}