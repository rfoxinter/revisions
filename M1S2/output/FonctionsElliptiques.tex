\documentclass[14pt,usepdftitle=false,aspectratio=169]{beamer}
\usepackage{preambule}
\setbeamercolor{structure}{fg=black}
\usepackage{analysecomplexe,analyse,polynomes,structures,usuelles}\usepackage[nopar]{bigoperators}\let\div\fdiv\let\Lambda\varLambda
\hypersetup{pdftitle={Surfaces de Riemann -- Fonctions elliptiques}}
\title{Surfaces de Riemann\\\emph{Fonctions elliptiques}}
\author{}
\date{}
\begin{document}
\begin{frame}
    \titlepage
\end{frame}
\slideq{Propriétés de $\olddeg\!:\!\bbZ\lc G\rc\to\bbZ$}{1/37}
\slider{$\olddeg\!:\!\sum{g\in G}{}{n_g\lc g\rc}\mapsto\sum{g\in G}{}{n_g}$ est un morphisme de groupes}{1/37}
\slideq{$f$ a un pôle (d'ordre $k$) en $P$}{2/37}
\slider{$f\circ\phi^{-1}$ a un pôle (d'ordre $k$) en $\phi\l P\r$}{2/37}
\slideq{Diviseurs principaux}{3/37}
\slider{$\fdiv{\fr C\Lambda^\times}\subset\pol Z{\bbC/\Lambda}$}{3/37}
\slideq{$f$ est méromorphe en $P$}{4/37}
\slider{$f\circ\phi^{-1}$ est méromorphe en $\phi\l P\r$}{4/37}
\slideq{$\pic[0]{\bbC/\Lambda}$}{5/37}
\slider{$I_\Lambda/\div{\fr C\Lambda^\times}$}{5/37}
\slideq{$f$ est holomorphe en $P$}{6/37}
\slider{$f\circ\phi^{-1}$ est holomorphe en $\phi\l P\r$}{6/37}
\slideq{$\calO_X\l U\r$ pour $U\subopn X$}{7/37}
\slider{Fonctions $f\!:\!U\to\bbC$ holomorphes}{7/37}
\slideq{Théorème d'Abel--Jacobi}{8/37}
\slider{$\pic[0]{\bbC/\Lambda}\cong\bbC/\Lambda$ via $\lc\sum{P\in\bbC/\Lambda}{}{n_P\lc P\rc}\rc\mapsto\sum{P\in\bbC/\Lambda}{}{n_PP}$ et $\lc P\rc-\lc0\rc\mapsfrom P$}{8/37}
\slideq{Pôles d'une fonction elliptiques}{9/37}
\slider{Une fonction elliptique a deux pôles, comptés avec multiplicité}{9/37}
\slideq{Propriétés des fonctions méromorphes sur une surface de Riemann connexe compacte}{10/37}
\slider{Il existe une fonction méromorphe $f$ non constante sur $X$, et pour toute telle fonction $f$, $\calM\l X\r/\bbC\l f\r$ est finie}{10/37}
\slideq{CNS pour avoir $f\!:\!\varOmega\setminus\set P\to\bbC$ holomorphe/méromorphe}{11/37}
\slider{$f$ s'étend en une fonction holomorphe $\widehat f\!:\!\varOmega\to\bbP^1\l\bbC\r$\linebreak$f$ a un pôle en $P$ si et seulement si $f\l P\r=\infty$}{11/37}
\slideq{$f\!:\!X\to Y$ est un isomorphisme (ou biholomorphisme) de surfaces de Riemann}{12/37}
\slider{Il existe une fonction holomorphe $g\!:\!Y\to X$ telle que $f\circ g=\id_X$ et $g\circ f=\id_Y$}{12/37}
\slideq{Structure de $\calM\l X\r$}{13/37}
\slider{Si $X$ est connexe, $\calM\l X\r$ est un corps contenant $\bbC$}{13/37}
\slideq{$f$ a un pôle en $P$}{14/37}
\slider{$f\circ\phi^{-1}$ a un pôle en $\phi\l P\r$}{14/37}
\slideq{$f$ a un zéro (d'ordre $k$) en $P$}{15/37}
\slider{$f\circ\phi^{-1}$ a un zéro (d'ordre $k$) en $\phi\l P\r$}{15/37}
\slideq{Sommes particulières pour les fonctions elliptiques}{16/37}
\slider{$\sum{P\in\bbC/\Lambda}{}{\res Pf}=0$\linebreak$\sum{P\in\bbC/\Lambda}{}{\ord[P]f}=0$\linebreak$\sum{P\in\bbC/\Lambda}{}{P\res[P]f}=\in\Lambda$}{16/37}
\slideq{$\pic{\bbC/\Lambda}$}{17/37}
\slider{$\pol Z{\bbC/\Lambda}/\div{\fr C\Lambda^\times}$}{17/37}
\slideq{Carte complexe d'un espace topologique $X$}{18/37}
\slider{$\l U,\phi\r$ avec $U\subset X$ ouvert et $\phi\!:\!U\to V$ un homéomorphisme sur un ouvert $V$ de $\bbC$}{18/37}
\slideq{Théorème de l'image ouverte\linebreak Conséquences selon les propriétés de $X$}{19/37}
\slider{Si $f\!:\!X\to Y$ est holomorphe non constante alors $f$ est ouverte\linebreak En particulier, si $X$ est compacte connexe et $f$ est non constante alors $Y$ est compacte et $f$ est surjective}{19/37}
\slideq{$f\!:\!X\to Y$ est une fonction holomorphe entre surfaces de Riemann\linebreak$\l\l U_i,\phi_i\r\r_{i\in I}$ et $\l\l V_j,\psi_j\r\r_{j\in J}$ atlas respectifs de $X$ et $Y$}{20/37}
\slider{$f$ est continue et, pour tout $j\in J$, $\psi_j\circ f\!:\!f^{-1}\l V_j\r\to\bbC$ est holomorphe}{20/37}
\slideq{Idéal d'augmentation de $\pol ZG$}{21/37}
\slider{$I_G=\ker{\olddeg}$}{21/37}
\slideq{Surface de Riemann}{22/37}
\slider{Esapce topologique séparé non vide muni d'un atlas complexe}{22/37}
\slideq{Ensemble en bijection avec $\calM\l X\r$ pour $X$ connexe}{23/37}
\slider{$\calM\l X\r$ est en bijection avec les applications holomorphes $X\to\bbP^1\l\bbC\r$ via $f\mapsto\widehat f$ avec $\widehat f$ qui vaut la limite de $f$ au voisinage des singularités illusoires et $\widehat f\l P\r=\infty$ aux pôles de $f$}{23/37}
\slideq{$\div f$}{24/37}
\slider{$\sum{P\in\bbC/\Lambda}{}{\ord[P]f\lc P\rc}\in\bbZ\lc\bbC/\Lambda\rc$}{24/37}
\slideq{CNS pour avoir des fonctions holomorphes entre surfaces de Riemann $f\!:\!X\to Y$\linebreak$\l\l U_i,\phi_i\r\r_{i\in I}$ et $\l\l V_j,\psi_j\r\r_{j\in J}$ atlas respectifs de $X$ et $Y$}{25/37}
\slider{$f$ est continue et pour tout $V\subopn X$, $g\in\calO_Y\l V\r$, $g\circ f\in\calO_X\l f^{-1}\l V\r\r$\linebreak$f_{|U_i}$ est holomorphe pour tout $i\in I$ avec $\l U_i\r_{i\in I}$ un recouvrement ouvert de $X$\linebreak$f$ est continue et $f_{|f^{-1}\l V_j\r}$ est holomorphe pour tout $j\in J$ avec $\l V_j\r_{j\in J}$ un recouvrement ouvert de $Y$}{25/37}
\slideq{Fonction méromorphe sur $X$}{26/37}
\slider{Donnée de $S\subset X$ fermée et discrète, ainsi que de $f\!:\!X\setminus S\to\bbC$ holomorphe et méromorphe en tout point de $S$}{26/37}
\slideq{$f\!:U\to\bbC$ est holomorphe pour $U$ un ouvert d'une surface de Riemann $X$}{27/37}
\slider{$f\circ\phi_i^{-1}\!:\!\phi_i\l U\cap U_i\r\to\bbC$ est holomorphe pour toute carte $\l U_i,\phi_i\r$}{27/37}
\slideq{Atlas complexe d'un espace topologique $X$}{28/37}
\slider{Famille de cartes compatibles $\l\l U_i,\phi_i\r\r_{i\in I}$ avec $X=\bigcup{i\in I}{}{U_i}$}{28/37}
\slideq{$f$ a une singularité essentielle en $P$}{29/37}
\slider{$f\circ\phi^{-1}$ a une singularité essentielle en $\phi\l P\r$}{29/37}
\slideq{Propriétés de $\ord[P]\cdot$}{30/37}
\slider{$\ord[P]f=0$ si et seulement si $f\equiv0$ pour $X$ connexe\linebreak$\ord[P]{fg}=\ord[P]f+\ord[P]g$\linebreak$\ord[P]{f+g}\geqslant\min{\ord[P]f\ord[P]g}$}{30/37}
\slideq{Propriétés des fibres de $f\!:\!X\to Y$ holomorphe}{31/37}
\slider{Si $X$ est connexe et $f$ est non constante alors $f^{-1}\l\set a\r$ est une partie discrète et fermée de $X$ pour tout $a\in Y$\linebreak Si $X$ est compact alors les fibres sont finies}{31/37}
\slideq{CNS pour que $D=\sum{P\in\bbC/\Lambda}{}{n_P\lc P\rc}$ soit un diviseur principal}{32/37}
\slider{$\sum{P\in\bbC/\Lambda}{}{n_P}=0$ et $\sum{P\in\bbC/\Lambda}{}{n_PP}\in\Lambda$\linebreak Ou bien $D\in I_\Lambda^2$}{32/37}
\slideq{$\ord[P]f$}{33/37}
\slider{$\ord[\phi\l P\r]{f\circ\phi^{-1}}$}{33/37}
\slideq{$\l U,\phi\r$ et $\l V,\psi\r$ sont compatibles}{34/37}
\slider{$\psi\circ\phi^{-1}\!:\!\phi\l U\cap V\r\to\psi\l U\cap V\r$ est un biholomorphisme\linebreak Comme $\psi\circ\phi^{-1}$ est bijective, cela revient à avoir $\psi\circ\phi^{-1}$ holomorphe}{34/37}
\slideq{Propriété topologique de $\pi\!:\!\bbC\to\bbC/\Lambda$}{35/37}
\slider{$\pi$ est une application ouverte\linebreak Pour $r>0$ tel que $D\l0,r\r\cap\Lambda=\set0$, $\pi_{|D\l z_0,\frac r2\r}$ est un homéomorphisme}{35/37}
\slideq{Principe du prolongement analytique}{36/37}
\slider{Si $f,g\!:\!X\to Y$ sont holomorphes et $X$ est connexe avec $f\neq g$, alors $\set{x\in X,f\l x\r=g\l x\r}$ est une partie discrète et fermée de $X$\linebreak En contraposant, si $\set{x\in X,f\l x\r=g\l x\r}$ a un point d'accumulation dans $X$ alors $f=g$}{36/37}
\slideq{CNS pour que $f\!:\!X\to Y$ soit un isomorphisme}{37/37}
\slider{$f$ est holomorphe et bijective}{37/37}
\end{document}