\documentclass[14pt,usepdftitle=false,aspectratio=169]{beamer}
\usepackage{preambule}
\setbeamercolor{structure}{fg=black}
\usepackage{dsfont}
\hypersetup{pdftitle=Fondements -- Ensembles}
\title{Fondements\\\emph{Ensembles}}
\author{}
\date{}
\begin{document}
\begin{frame}
    \titlepage
\end{frame}
\slideq{$\1{A\cup B}$}{1/6}
\slider{$\1{A}+\1{B}-\1{A\cap B}=\1{A}+\1{B}-\1{A}\1{B}$}{1/6}
\slideq{Lemme de (Kuratowski-)Zorn}{2/6}
\slider{Tout ensemble inductif admet un élément maximal\linebreak Avec l'axiome du choix}{2/6}
\slideq{$\1{A\uplus B}$}{3/6}
\slider{$\1{A}+\1{B}$}{3/6}
\slideq{$\1{A\cap B}$}{4/6}
\slider{$\1{A}\1{B}$}{4/6}
\slideq{Si $\l E,\leqslant\r$ est un ensemble ordonné\linebreak $E$ est un ensemble inductif}{5/6}
\slider{$\forall F\subset E$ avec $F$ totalement ordonné, $F$ admet un majorant dans $E$}{5/6}
\slideq{$\1{\complement_EA}$}{6/6}
\slider{$1-\1{A}$}{6/6}
\end{document}