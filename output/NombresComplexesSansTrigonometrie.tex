\documentclass[14pt,usepdftitle=false,aspectratio=169]{beamer}
\usepackage{preambule}
\setbeamercolor{structure}{fg=black}
\usepackage{trigo}\usepackage{complexes}
\hypersetup{pdftitle=Fondements -- Nombres complexes sans trigonométrie}
\title{Fondements\\\emph{Nombres complexes sans trigonométrie}}
\author{}
\date{}
\begin{document}
\begin{frame}
    \titlepage
\end{frame}
\slideq{Translation d'un vecteur $\vec{u}$ d'affixe $z_u$}{1/8}
\slider{$z\mapsto z+z_u$}{1/8}
\slideq{Homothétie de centre $A$ d'affixe $z_A$ et de rapport $\lambda$}{2/8}
\slider{$z\mapsto \lambda\l z-z_A\r+z_A$}{2/8}
\slideq{Équation de cercle}{3/8}
\slider{$z\bar{z}+\alpha z+\bar{\alpha}\bar{z}+\beta=0$, $\l\alpha,\beta\r\in\mathbb{C}\times\mathbb{R}$\linebreak Le cercle est vide si et seulement si $\beta>\alpha\bar\alpha$\linebreak$c=-\bar{\alpha}$, $r=\sqrt{\alpha\bar\alpha+\beta}$}{3/8}
\slideq{$\e^{\i x}+\e^{-\i x}$}{4/8}
\slider{$2\cos{x}$}{4/8}
\slideq{Rotation de centre $A$ d'affixe $z_A$ et d'angle $\theta$}{5/8}
\slider{$z\mapsto \e^{\i\theta}\l z-z_A\r+z_A$}{5/8}
\slideq{Symétrie orthogonale d'axe $D$, une droite passant par $A$ d'affixe $z_A$ et dirigée par $\vec{u}$ d'affixe $z_u$}{6/8}
\slider{$z\mapsto z_u^2\l\bar{z}-\bar{z_A}\r+z_A$}{6/8}
\slideq{$\l\overrightarrow{AC},\overrightarrow{AB}\r$}{7/8}
\slider{$\arg\l\frac{z_B-z_A}{z_C-z_A}\r$}{7/8}
\slideq{$\e^{\i x}-\e^{-\i x}$}{8/8}
\slider{$2\i\sin{x}$}{8/8}
\end{document}