\documentclass[14pt,usepdftitle=false,aspectratio=169]{beamer}
\usepackage{preambule}
\setbeamercolor{structure}{fg=black}
\usepackage{bigoperators}\usepackage{structures}
\hypersetup{pdftitle=Algèbre 1 -- Anneaux}
\title{Algèbre 1\\\emph{Anneaux}}
\author{}
\date{}
\begin{document}
\begin{frame}
    \titlepage
\end{frame}
\slideq{Monoïde}{1/24}
\slider{Muni d'une loi de composition interne, de l'associativité et d'un élément neutre}{1/24}
\slideq{Si $\l A,+,\times\r$ est un anneau\linebreak Un sous-ensemble $B$ de $A$ est un sous-anneau de $A$}{2/24}
\slider{$B$ est stable pour les lois $+$ et $\times$\linebreak$1_A\in B$\linebreak Les lois induites sur $B$ définissent sur $B$ une structure d'anneau}{2/24}
\slideq{Propriété sur $1$ et $0$ si l'anneau $A$ a plus d'un élément}{3/24}
\slider{$1\neq0$}{3/24}
\slideq{Élément absorbant dans un anneau $\l A,+,\times\r$}{4/24}
\slider{$0$}{4/24}
\slideq{Automorphisme d'anneaux}{5/24}
\slider{Endomorphisme et isomorphisme d'anneaux}{5/24}
\slideq{Groupe}{6/24}
\slider{Muni d'une loi de composition interne, de l'associativité, d'un élément neutre et de symétriques}{6/24}
\slideq{Endomorphisme d'anneaux}{7/24}
\slider{Homomorphisme d'anneaux de $E$ dans lui-même (muni des mêmes lois)}{7/24}
\slideq{Si $\l A,+,\times\r$ est un groupe et $B\subset A$\linebreak Caractérisation(s) des sous-anneaux}{8/24}
\slider{$1_A\in B$\quad$\forall\l x,y\r\in B,\;x-y\in B$\quad$\forall\l x,y\r\in B,\;xy\in B$}{8/24}
\slideq{Anneau intègre}{9/24}
\slider{Anneau commutatif non réduit à $\left\{0\right\}$ et sans diviseurs de $0$}{9/24}
\slideq{Groupe des inversibles d'un anneau}{10/24}
\slider{$A^\times$\linebreak$A^\times$ est un groupe multiplicatif}{10/24}
\slideq{Image directe et réciproque de sous-anneaux par un homomorphisme}{11/24}
\slider{Si $A$ et $B$ sont deux anneaux, et $f\!:\!A\to B$ un morphisme d'anneaux, $A'$ et $B'$ deux sous-anneaux respectivement de $A$ et $B$\linebreak $f\l A'\r$ est un sous-anneau de $B$\linebreak $f^{-1}\l B'\r$ est un sous-anneau de $A$}{11/24}
\slideq{Isomorphisme d'anneaux}{12/24}
\slider{Homomorphisme d'anneaux bijectif}{12/24}
\slideq{Diviseurs de zéro dans un anneau $A$}{13/24}
\slider{$a\in A$ est un diviseur de $0$ à gauche si et seulement s'il existe $b\in A$ tel que $ab=0$\linebreak$a\in A$ est un diviseur de $0$ à droite si et seulement s'il existe $b\in A$ tel que $ba=0$\linebreak$a\in A$ est un diciseur de$0$ si et seulement si $a$ est diviseur de $0$ à gauche et à droite}{13/24}
\slideq{Intersection de sous-anneaux\linebreak Si $A$ est un groupe, et $\l B_i\r_{i\in I}$ une famille de sous-anneaux de $A$}{14/24}
\slider{$\bigcap{i\in I}{}{B_i}$ est un sous-anneau de $A$}{14/24}
\slideq{Soient $\l A,\underset{\scriptscriptstyle A}{+},\underset{\scriptscriptstyle A}{\times}\r$ et $\l B,\underset{\scriptscriptstyle B}{+},\underset{\scriptscriptstyle B}{\times}\r$ deux anneaux\linebreak$f\!:\!A\to B$ est un homomorphisme d'anneaux}{15/24}
\slider{$\forall\l x,y\r\in A^2,\;f\l x\underset{\scriptscriptstyle A}{+}y\r=f\l x\r\underset{\scriptscriptstyle B}{+}f\l y\r$\linebreak$\forall\l x,y\r\in A^2,\;f\l x\underset{\scriptscriptstyle A}{\times}y\r=f\l x\r\underset{\scriptscriptstyle B}{\times}f\l y\r$\linebreak$f\l 1_A\r=1_B$}{15/24}
\slideq{Factorisation de $a^n-b^n$ dans un anneau $A$}{16/24}
\slider{$\l a,b\r\in A^2$ tel que $ab=ba$\linebreak$\l a-b\r\sum{k=0}{n-1}{a^{n-k-1}b^k}$}{16/24}
\slideq{Idéal principal}{17/24}
\slider{Idéal engendré par un unique élément $a$ de la forme $I=aA=\left\{ay,\;y\in A\right\}$\linebreak$I$ est souvent noté $\l a\r$}{17/24}
\slideq{Anneau principal}{18/24}
\slider{Un anneau intègre dont tous les idéaux sont principaux}{18/24}
\slideq{Anneau commutatif}{19/24}
\slider{Anneau dont la loi $\times$ est commutative}{19/24}
\slideq{Factorisation de $\l a+b\r^n$ dans un anneau $A$}{20/24}
\slider{$\l a,b\r\in A^2$ tel que $ab=ba$\linebreak$\sum{k=0}{n}{\binom{n}{k}a^kb^{n-k}}$}{20/24}
\slideq{Anneau}{21/24}
\slider{Muni de deux lois de composition internes (généralement notées $+$ et $\times$)\linebreak$\l A,+\r$ est un groupe abélien\linebreak$\l A,\times\r$ est un monoïde\linebreak$\times$ est distributive sur $+$}{21/24}
\slideq{Élément réguulier d'un anneau}{22/24}
\slider{L'élément n'est pas diviseur de $0$\linebreak La réciproque est vraie\linebreak S'adapte à gauche et à droite}{22/24}
\slideq{Si $A$ est un anneau commutatif et $I$ un idéal de $A$\linebreak Anneau quotient}{23/24}
\slider{$A/I$ peut être muni d'une multiplication avec pour tout $\l a,b\r\in A$, $\overline{ab}=\overline{a}\overline{b}$\linebreak$A/I$ est muni d'une structure d'anneau}{23/24}
\slideq{Si $\l A,+,\times\r$ est un anneau commutatif\linebreak Un sous-ensemble $I$ de $A$ est un sous-anneau idéal de $A$}{24/24}
\slider{$I$ est un sous-groupe de $\l A,+\r$\linebreak$\forall i\in I,\;\forall a\in A,\;ia\in I$}{24/24}
\end{document}