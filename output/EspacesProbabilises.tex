\documentclass[14pt,usepdftitle=false,aspectratio=169]{beamer}
\usepackage{preambule}
\setbeamercolor{structure}{fg=black}
\newcommand\p{\mathbb P}\let\oldOmega\Omega\let\Omega\varOmega\usepackage{bigoperators}
\hypersetup{pdftitle=Probabilités -- Espaces probabilisés}
\title{Probabilités\\\emph{Espaces probabilisés}}
\author{}
\date{}
\begin{document}
\begin{frame}
    \titlepage
\end{frame}
\slideq{Tribu des boréliens sur $\mathbb R^n$}{1/7}
\slider{$\mathcal B^n$\linebreak Tribu engendrée par les $I_1\times\cdots\times I_n$ où les $I_k$ sont des intervalles}{1/7}
\slideq{Intersection de tribus}{2/7}
\slider{Si $\l\mathcal T_i\r_{i\in I}$ est une famille de $\sigma$-algèbres sur $\Omega$, alors $\bigcap{i\in I}{}{\mathcal T_i}$ est une $\sigma$-algèbre sur $\Omega$}{2/7}
\slideq{Tribu engendrée par une famille}{3/7}
\slider{$\sigma\l\l A_i\r_{i\in I}\r$ avec $A_i$ des éléments de $\mathcal{P}\l\Omega\r$\linebreak Plus petite $\sigma$-algèbre de $\Omega$ contenant $\l A_i\r_{i\in I}$}{3/7}
\slideq{Système complet d'événements}{4/7}
\slider{Famille $\left\{A_i,\;i\in I\right\}$ formant une partition de $\Omega$}{4/7}
\slideq{$\sigma$-algèbre\linebreak Tribu}{5/7}
\slider{Une $\sigma$-algèbre $\mathcal T$ est un sous-ensemble de $\mathcal P\l \Omega\r$ vérifiant\linebreak $\Omega\in\mathcal{T}$\linebreak$A\in\mathcal T\Rightarrow\overline A\in\mathcal T$\linebreak Si $I$ est dénombrable et $\l A_i\r_{i\in I}$ une famille d'éléments de $\mathcal T$, $\bigcup{i\in I}{}{A_i}\in\mathcal T$}{5/7}
\slideq{Espace probabilisable}{6/7}
\slider{$\l \mathcal T,\Omega\r$\linebreak$\mathcal T$ est une $\sigma$-algèbre sur $\Omega$}{6/7}
\slideq{Tribu des boréliens}{7/7}
\slider{$\mathcal B^1$ ou $\mathcal B$\linebreak$\sigma\l\l\left]-\infty,a\right[\r_{a\in\mathbb R}\r$\linebreak\linebreak$\mathcal B^1$ est aussi engendrée par n'importe quel type d'intervalle de $\mathbb R$}{7/7}
\end{document}