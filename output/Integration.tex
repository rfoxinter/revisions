\documentclass[14pt,usepdftitle=false,aspectratio=169]{beamer}
\usepackage{preambule}
\setbeamercolor{structure}{fg=black}
\usepackage{analyse,equivalents}\newcommand{\cv}[2][]{\xrightarrow[\scriptscriptstyle{#2}]{\scriptscriptstyle{\text{#1}}}}
\hypersetup{pdftitle=Analyse -- Intégration}
\title{Analyse\\\emph{Intégration}}
\author{}
\date{}
\begin{document}
\begin{frame}
    \titlepage
\end{frame}
\slideq{Intégration des relations de comparaison dans le cas convergeant\linebreak$g$ est intégrable}{1/6}
\slider{$f=\O[b]{g}\Rightarrow\int[t][x][b]{f\l t\r}=\O[t\to b]{\int[t][x][b]{g\l t\r}}$\linebreak$f=\o[b]{g}\Rightarrow\int[t][x][b]{f\l t\r}=\o[t\to b]{\int[t][x][b]{g\l t\r}}$\linebreak$\eq[b]{f}{g}\Rightarrow\eq[t\to b]{\int[t][x][b]{f\l t\r}}{\int[t][x][b]{g\l t\r}}$\linebreak Dans ces trois cas, $f$ est intégrable}{1/6}
\slideq{Théorème de convergeance dominée}{2/6}
\slider{Si $f_n\cv[\text{CVS}]{n\to+\infty}f$, et il existe $\varphi$ intégrable tel que $\forall n\in\mathbb N$, $t\in I$, $\left|f_n\l t\r\right|\leqslant\varphi\l t\r$\linebreak Alors, $f_n$ est intégrable et $\int[t][I]{f_n\l t\r}\cv{n\to+\infty}\int[t][I]{f\l t\r}$}{2/6}
\slideq{Théorème de convergeance dominée à paramètre continu}{3/6}
\slider{Si $f_\lambda\l t\r\cv{\lambda\to l}f\l t\r$, et il existe $\varphi$ intégrable tel que $\forall \lambda\in J$, $t\in I$, $\left|f_\lambda\l t\r\right|\leqslant\varphi\l t\r$\linebreak Les $f_\lambda$ sont intégrables et $\int[t][I]{f_\lambda\l t\r}\cv{\lambda\to l}\int[t][I]{f\l t\r}$}{3/6}
\slideq{Minoration divergeante}{4/6}
\slider{Si $f\in\cm{\left[a,b\right[,\mathbb K}$ et $g\in\cm{\left[a,b\right[,\mathbb R_+}$ telles que $f\l t\r=\O[t\to b^-]{g\l t\r}$\linebreak Si $f$ n'est pas intégrable en $b$, alors $g$ non plus}{4/6}
\slideq{Sommation des relations de comparaison dans le cas divergeant\linebreak${g}$ est non intégrable}{5/6}
\slider{$f=\O[b]{g}\Rightarrow\int[t][a][x]{f\l t\r}=\O[t\to b]{\int[t][a][x]{g\l t\r}}$\linebreak$f=\o[b]{g}\Rightarrow\int[t][a][x]{f\l t\r}=\o[t\to b]{\int[t][a][x]{g\l t\r}}$\linebreak$\eq[b]{f}{g}\Rightarrow\eq[t\to b]{\int[t][a][x]{f\l t\r}}{\int[t][a][x]{g\l t\r}}$\linebreak Dans ce dernier cas, $f$ n'est pas intégrable}{5/6}
\slideq{Majoration convergeante}{6/6}
\slider{Si $f\in\cm{\left[a,b\right[,\mathbb K}$ et $g\in\cm{\left[a,b\right[,\mathbb R_+}$ telles que $f\l t\r=\O[t\to b^-]{g\l t\r}$\linebreak Si $g$ est intégrable en $b$, alors $f$ aussi}{6/6}
\end{document}