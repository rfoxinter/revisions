\documentclass[14pt,usepdftitle=false,aspectratio=169]{beamer}
\usepackage{preambule}
\setbeamercolor{structure}{fg=black}

\hypersetup{pdftitle=Algèbre 1 -- Corps}
\title{Algèbre 1\\\emph{Corps}}
\author{}
\date{}
\begin{document}
\begin{frame}
    \titlepage
\end{frame}
\slideq{Image directe et réciproque de sous-corps par un homomorphisme}{1/14}
\slider{Si $K$ et $L$ sont deux corps, et $f\!:\!K\to L$ un morphisme de corps, $K'$ et $L'$ deux sous-corps respectivement de $K$ et $L$\linebreak $f\l K'\r$ est un sous-corps de $L$\linebreak $f^{-1}\l L'\r$ est un sous-corps de $K$}{1/14}
\slideq{Propriété de $\mathbb{F}_p=\mathbb{Z}/p\mathbb{Z}$}{2/14}
\slider{$\mathbb{F}_p$ est un corps si et seulement si $p$ est premier}{2/14}
\slideq{Corps}{3/14}
\slider{Muni de deux lois de composition internes (généralement notées $+$ et $\times$)\linebreak$\l K,+,\times\r$ est un anneau commutatif\linebreak$\l K^*,\times\r$ est un groupe}{3/14}
\slideq{Si $K$ est un corps, d'élément neutre $1_K\neq0_K$, $H=\left\{n\times1_K,\;n\in\mathbb{Z}\right\}$ le sous-groupe monogène de $\l K,+\r$ engendré par $1_K$\linebreak Caractéristique d'un corps}{4/14}
\slider{Si $H$ est infini, $K$ est de caractéristique nulle\linebreak Si $H$ est fini de cardinal $p$, $K$ est de caractéristique $p$}{4/14}
\slideq{Si $K$ est un corps de caractéristique finie $p$\linebreak Propriété pour les éléments de $K$}{5/14}
\slider{$\forall x\in K,\;px=0_K$}{5/14}
\slideq{Groupe}{6/14}
\slider{Muni d'une loi de composition interne, de l'associativité, d'un élément neutre et de symétriques}{6/14}
\slideq{Si $\l K,+,\times\r$ est un corps\linebreak Un sous-ensemble $L$ de $K$ est un sous-corps de $K$}{7/14}
\slider{$L$ est stable pour les lois $+$ et $\times$\linebreak$1_K\in L$\linebreak Les lois induites sur $L$ définissent sur $L$ une structure de corps}{7/14}
\slideq{Soient $\l K,\underset{\scriptscriptstyle K}{+},\underset{\scriptscriptstyle K}{\times}\r$ et $\l L,\underset{\scriptscriptstyle L}{+},\underset{\scriptscriptstyle L}{\times}\r$ deux corps\linebreak$f\!:\!K\to L$ est un homomorphisme de corps}{8/14}
\slider{$f$ est un homomorphisme des anneaux de $K$ et $L$}{8/14}
\slideq{Anneau}{9/14}
\slider{Muni de deux lois de composition internes (généralement notées $+$ et $\times$)\linebreak$\l A,+\r$ est un groupe abélien\linebreak$\l A,\times\r$ est un monoïde\linebreak$\times$ est distributive sur $+$}{9/14}
\slideq{Propriété des homomorphismes de corps}{10/14}
\slider{Un homomorphisme de corps est injectif}{10/14}
\slideq{Propriété de la caractéristique d'un corps}{11/14}
\slider{Si $K$ est un corps de caractéristique $p$ non nulle, $p$ est premier}{11/14}
\slideq{Si $\l K,+,\times\r$ est un groupe et $L\subset K$\linebreak Caractérisation des sous-corps}{12/14}
\slider{$1_K\in L$\quad$\forall\l x,y\r\in L,\;x-y\in L$\quad$\forall\l x,y\r\in L,\;y\neq0\Rightarrow xy^{-1}\in L$}{12/14}
\slideq{Si $K$ est un corps de caractéristique nulle\linebreak Propriété pour les éléments de $K$}{13/14}
\slider{$\forall \l n,x\r\in\mathbb{Z}\times K$\linebreak$n\times x=0_K\Leftrightarrow\l x=0_K\vee n=0\r$}{13/14}
\slideq{Soient $\l A,\underset{\scriptscriptstyle A}{+},\underset{\scriptscriptstyle A}{\times}\r$ et $\l B,\underset{\scriptscriptstyle B}{+},\underset{\scriptscriptstyle B}{\times}\r$ deux anneaux\linebreak$f\!:\!A\to B$ est un homomorphisme d'anneaux}{14/14}
\slider{$\forall\l x,y\r\in A^2,\;f\l x\underset{\scriptscriptstyle A}{+}y\r=f\l x\r\underset{\scriptscriptstyle B}{+}f\l y\r$\linebreak$\forall\l x,y\r\in A^2,\;f\l x\underset{\scriptscriptstyle A}{\times}y\r=f\l x\r\underset{\scriptscriptstyle B}{\times}f\l y\r$\linebreak$f\l 1_A\r=1_B$}{14/14}
\end{document}