\documentclass[14pt,usepdftitle=false,aspectratio=169]{beamer}
\usepackage{preambule}
\setbeamercolor{structure}{fg=black}
\usepackage{dsfont}\usepackage{bigoperators}\usepackage{structures,arithmetique}
\hypersetup{pdftitle=Algèbre 1 -- Structures algébriques}
\title{Algèbre 1\\\emph{Structures algébriques}}
\author{}
\date{}
\begin{document}
\begin{frame}
    \titlepage
\end{frame}
\slideq{Associativité}{1/60}
\slider{$\star$ est associative si et seulement si$\forall\l x,y,z\r\in E^3,\;\l x\star y\r\star z=x\star\l y\star z\r$}{1/60}
\slideq{Sous-groupe engendrée par une partie $X$}{2/60}
\slider{$\la X\ra$\linebreak C'est le plus petit sous-groupe contenant $X$}{2/60}
\slideq{Théorème de Lagrange pour l'ordre des éléments d'un groupe}{3/60}
\slider{Si $G$ est un groupe fini et $x\in G$\linebreak$\ord{x}\div\left|G\right|$}{3/60}
\slideq{Description par le bas du sous-groupe engendré par une partie}{4/60}
\slider{$\la X\ra=\left\{x_1\cdots x_n,\;\l x_1,\cdots,x_n\r\in X^n\right\}$\linebreak${}\cup\left\{x^{-1},\;x\in X\right\}$\linebreak $e$ correspond au produit vide}{4/60}
\slideq{Automorphisme de X}{5/60}
\slider{Endomorphisme et isomorphisme de X}{5/60}
\slideq{Ordre d'un groupe\linebreak Si $G$ est un groupe}{6/60}
\slider{$\ord{G}=\left|G\right|$}{6/60}
\slideq{Distributivité généralisée\linebreak$\prod{i=1}{n}{\sum{j\in J_i}{}{x_{i,j}}}$}{7/60}
\slider{$\sum{\l j_1,\cdots,j_n\r\in J_1\times\cdots\times J_n}{}{\prod{i=1}{n}{x_{i,j_i}}}$}{7/60}
\slideq{Commutativité}{8/60}
\slider{$\star$ est commutative si et seulement si$\forall\l x,y\r\in E^2,\;x\star y=y\star x$}{8/60}
\slideq{Si $f\in\hom{G,K}$ et $H$ est un sous-groupe distingué}{9/60}
\slider{$f$ passe au quotient avec $\tilde{f}\!:\!G/H\to K$}{9/60}
\slideq{Si $G$ et $H$ sont deux groupes et $f\in\hom{G,H}$\linebreak$f\l x^{-1}\r$}{10/60}
\slider{$f\l x\r^{-1}$}{10/60}
\slideq{Soient $E$ muni d'une structure de X et $F\subset E$\linebreak$F$ est un sous-X de $E$}{11/60}
\slider{$F$ est stable par les lois de $E$\linebreak$F$ contient les neutres imposés par $E$\linebreak Les lois induites sur $F$ par les lois de $E$ vérifient les axiomes de la structure de X}{11/60}
\slideq{Si $\ker{f}=\left\{e_G\right\}$}{12/60}
\slider{$f$ est injectif (la réciproque est vraie)}{12/60}
\slideq{Si $\l G,\star\r$ est un groupe\linebreak Un sous-ensemble $H$ de $G$ est appelé sous-groupe de $G$}{13/60}
\slider{$H$ est stable pour la loi de $G$ et la loi induite définit sur $H$ une structure de groupe}{13/60}
\slideq{Image directe et réciproque de sous-groupes par un homomorphisme}{14/60}
\slider{Si $G$ et $H$ sont deux groupes, et $f\in\hom{G,H}$ un morphisme de groupes, $G'$ et $H'$ deux sous-groupes respectivement de $G$ et $H$\linebreak $f\l G'\r$ est un sous-groupe de $H$\linebreak $f^{-1}\l H'\r$ est un sous-groupe de $G$}{14/60}
\slideq{Soit $E$ et $F$ deux ensembles munis d'une structure de X, munis respectivement des lois de composition internes $\l\underset{\scriptscriptstyle1}{\star},\cdots,\underset{\scriptscriptstyle n}{\star}\r$ et $\l\underset{\scriptscriptstyle1}{\diamond},\cdots,\underset{\scriptscriptstyle n}{\diamond}\r$, et externes $\l\underset{\scriptscriptstyle1}{\square},\cdots,\underset{\scriptscriptstyle m}{\square}\r$ et $\l\underset{\scriptscriptstyle1}{\triangle},\cdots,\underset{\scriptscriptstyle m}{\triangle}\r$ sur $K_1,\cdots,K_m$\linebreak$f\!:\!E\to F$ est un homomorphisme}{15/60}
\slider{$f$ respecte les lois interne : soit $k\in\llb1,n\rrb$\linebreak$\forall \l x,y\r\in E^2,\;f\l x\underset{\scriptscriptstyle k}{\star}y\r=f\l x\r\underset{\scriptscriptstyle k}{\diamond}f\l y\r$\linebreak$f$ respecte les lois externes : soit $k\in\llb1,m\rrb$\linebreak$\forall \l\lambda,x\r\in K_k\times E,\;f\l \lambda\underset{\scriptscriptstyle k}{\square}y\r=\lambda\underset{\scriptscriptstyle k}{\triangle}f\l x\r$\linebreak$f$ est compatible avec le neutre (si le neutre $e_i$ pour la loi $\underset{\scriptscriptstyle i}{\star}$ est imposé dans les axiomes, donc le neutre $e_i'$ existe pour la loi $\underset{\scriptscriptstyle i}{\diamond}$) : $f\l e_i\r=e_i'$}{15/60}
\slideq{Description par le haut du sous-groupe engendré par une partie}{16/60}
\slider{Soient $\mathcal{G}$ l'ensemble des sous-groupes de $G$ et $\mathcal{H}=\left\{H\in\mathcal{H}\;|\;X\subset H\right\}$\linebreak$\la X\ra=\bigcap{x\in\mathcal{H}}{}{H}$}{16/60}
\slideq{Groupe cyclique}{17/60}
\slider{Groupe monogène fini}{17/60}
\slideq{Soient $E$ muni d'une loi $\star$, $F\subset E$\linebreak$F$ est stable par $\star$}{18/60}
\slider{$\forall\l x,y\r\in F^2,\;x\star y\in F$\linebreak La loi de $E$ se restreint en une loi $\star_F$ appelée loi induite sur $F$ par $\star$}{18/60}
\slideq{Soit $x\in E$\linebreak$x$ est un élement absorbant pour $\star$}{19/60}
\slider{$\forall y\in E,\;x\star y=x=y\star x$}{19/60}
\slideq{Symétrique de $x\star y$}{20/60}
\slider{$y^s\star x^s$}{20/60}
\slideq{Endomorphisme de X}{21/60}
\slider{Homomorphisme de X de $E$ dans lui-même (muni des mêmes lois)}{21/60}
\slideq{Distributivité}{22/60}
\slider{La loi $\star$ est distributive à gauche sur $\diamond$ si et seulement si$\forall\l x,y,z\r\in E^3,\;x\star\l y\diamond z\r=\l x\star y\r\diamond\l x\star z\r$\linebreak La loi $\star$ est distributive à droite sur $\diamond$ si et seulement si$\forall\l x,y,z\r\in E^3,\;\l y\diamond z\r\star x=\l y\star x\r\diamond\l z\star x\r$\linebreak La loi $\star$ est distributive sur $\diamond$ si et seulement si elle est distributive à gauche et à droite}{22/60}
\slideq{Sous-groupe propre de $G$}{23/60}
\slider{Sous-groupe de $G$ distinct de $G$ et $\left\{e_G\right\}$}{23/60}
\slideq{Si $G$ et $H$ sont deux groupes et $f\in\hom{g,h}$ un morphisme de groupes\linebreak$\ker{f}$}{24/60}
\slider{$f^{-1}\l e_H\r=\left\{y\in G\;|\;f\l y\r=e_H\right\}$}{24/60}
\slideq{Cardinal des classes de congruence}{25/60}
\slider{$\left|Ha,\;a\in G\right|=\left|Ha,\;a\in G\right|=\left|H\right|$}{25/60}
\slideq{Intersection de sous-groupes\linebreak Si $G$ est un groupe, et $\l H_i\r_{i\in I}$ une famille de sous-groupes de $G$}{26/60}
\slider{$\bigcap{i\in I}{}{H_i}$ est un sous-groupe de $G$}{26/60}
\slideq{Propriétés d'un groupe $\l G,\star\r$}{27/60}
\slider{$G$ admet un uique élément neutre pour $\star$\linebreak$\forall x\in G,\;\exists!x^s\in G$}{27/60}
\slideq{Propriété des groupes monogènes}{28/60}
\slider{Un groupe monogène est abélien}{28/60}
\slideq{Soient $e\in E$ un élément neutre pour la loi $\star$ et $x\in E$\linebreak$y$ est un symétrique de $x$ pour la loi $\star$}{29/60}
\slider{$x\star y=e=y\star x$}{29/60}
\slideq{Les classes à droite modulo $H$}{30/60}
\slider{$\left\{Ha,\;a\in G\right\}$}{30/60}
\slideq{Passage au quotient de la loi dans le cas d'un sous-groupe distingué\linebreak Si $G$ est un groupe et $H$ un sous-groupe distingué de $G$}{31/60}
\slider{${\equiv_g}={\equiv_d}$ et on note la relation $\equiv$\linebreak La loi induite corrrespond au produit des classes élément par élément : $\l ab\r H=\l aH\r\cdot\l bH\r=\left\{x\cdot y,\;x\in aH,\;y\in bH\right\}$\linebreak La loi induite sur l'ensemble quotient munit celui-ci d'une structure de groupe}{31/60}
\slideq{Soit $e\in E$\linebreak$e$ est un élément neutre pour la loi $\star$}{32/60}
\slider{$\forall x\in E,\;e\star x=x=x\star e$}{32/60}
\slideq{Groupe abélien}{33/60}
\slider{La loi $\star$ de $G$ est commutative}{33/60}
\slideq{Commutativité généralisée}{34/60}
\slider{Si $\star$ est une loi commutative et associative sur $E$, $\l x_1,\cdots,x_n\r\in E^n$ et $\sigma\in\mathfrak{S}_n$\linebreak$x_1\star\cdots\star x_n=x_{\sigma\l1\r}\star\cdots\star x_{\sigma\l n\r}$}{34/60}
\slideq{$x$ et $y$ sont dans la même classe à droite modulo $H$}{35/60}
\slider{$x\equiv_dy\,\left[H\right]\Leftrightarrow xy^{-1}\in H$}{35/60}
\slideq{Isomorphisme de X}{36/60}
\slider{Homomorphisme de X bijectif}{36/60}
\slideq{Associativité externe\linebreak$E$ est muni d'une loi decomposition externe $\diamond$ sur $\mathbb{K}$, muni d'une loi de composition interne $\star$}{37/60}
\slider{$\forall\l\lambda,\mu,x\r\in\mathbb{K}^2\times E,\;\l\lambda\star\mu\r\diamond x=\lambda\diamond\l\mu\diamond x\r$}{37/60}
\slideq{Théorème de Lagrange pour l'ordre des groupes}{38/60}
\slider{Si $G$ est un groupe fini et $H$ un sous-groupe de $G$\linebreak$\left|H\right|\div\left|G\right|$}{38/60}
\slideq{Élément régulier ou simplifiable}{39/60}
\slider{$x$ est régulier à gauche si et seulement si$\forall\l y,z\r\in E^2,\;x\star y=x\star z\Rightarrow y=z$\linebreak$x$ est régulier à droite si et seulement si$\forall\l y,z\r\in E^2,\;y\star x=z\star x\Rightarrow y=z$\linebreak$x$ est régulier si et seulement s'il est régulier à gauche et à droite\linebreak Si $x$ admet un symétrique, alors il est régulier}{39/60}
\slideq{Les classes à gauche modulo $H$}{40/60}
\slider{$\left\{aH,\;a\in G\right\}$}{40/60}
\slideq{Premier théorème d'isomorphisme}{41/60}
\slider{Si $f\in\hom{G,H}$\linebreak$\ker{f}$ est un sous-groupe distingué de $G$, et $f$ passe au quotient, définissant un morphisme de groupes $\tilde{f}\!:\!G/\ker{f}\to H$\linebreak $\tilde{f}$ est injectif et sa corestriction à son image est un isomorphisme}{41/60}
\slideq{Groupe}{42/60}
\slider{Muni d'une loi d'une composition interne, de l'associativité, d'un élément neutre et de symétriques\linebreak Un groupe est un monoïde}{42/60}
\slideq{Description des groupes monogènes\linebreak Si $G=\la x\ra$}{43/60}
\slider{Si $\ord{x}=+\infty$, $G$ est isomorphe à $\mathbb{Z}$\linebreak Si $\ord{x}=n\in\mathbb{N}^*$, $G$ est isomorphe à $\mathbb{Z}/n\mathbb{Z}$}{43/60}
\slideq{Si $H$ est un sous-groupe distingué de $G$}{44/60}
\slider{$\forall a\in G,\;aH=Ha\Leftrightarrow\forall a\in G,\;\forall h\in H,\;aha^{-1}\in H$}{44/60}
\slideq{Si $\l G,\star\r$ est un groupe et $H\subset G$\linebreak Caractérisation(s) des sous-groupes}{45/60}
\slider{$H\neq\varnothing$\quad$\forall\l x,y\r\in H,\;x\star y\in H$\quad$\forall x\in H,\;x^s\in H$\linebreak$H\neq\varnothing$\quad$\forall\l x,y\r\in H^2,\;x\star y^s\in H$\linebreak$e_G\in H$\quad$\forall\l x,y\r\in H^2,\;x\star y^s\in H$}{45/60}
\slideq{Ensemble formé par les classes à gauche et à droite}{46/60}
\slider{$\left\{Ha,\;a\in G\right\}$ est une partition de $G$\linebreak$\left\{aH,\;a\in G\right\}$ est une partition de $G$}{46/60}
\slideq{Si $f\in\hom{G,K}$ et $H$ est un sous-groupe distingué et $H\subset\ker{f}$}{47/60}
\slider{$f=\tilde{f}\circ\pi$\linebreak La réciproque est vraie}{47/60}
\slideq{Si $G$ est un gruope\linebreak Structure de $\l\aut G,\circ\r$}{48/60}
\slider{$\l\aut G,\circ\r$ est un groupe}{48/60}
\slideq{Ordre d'un élément d'un groupe}{49/60}
\slider{$\ord{x}=\min\left(\left\{n\in\mathbb{N}^*\;|\;x^n=e\right\}\right)$}{49/60}
\slideq{Si $\star$ est une loi associative sur $E$ et $\l x_1,\cdots,x_n\r\in E^n$}{50/60}
\slider{$x_1\star\cdots\star x_n$ ne dépend pas du parenthésage admissible}{50/60}
\slideq{$x$ et $y$ sont dans la même classe à gauche modulo $H$}{51/60}
\slider{$x\equiv_gy\,\left[H\right]\Leftrightarrow x^{-1}y\in H$}{51/60}
\slideq{Magma}{52/60}
\slider{Muni d'une loi de composition interne}{52/60}
\slideq{Sous-groupe monogène}{53/60}
\slider{$\la x\ra=\left\{x^n,\;n\in\mathbb{N}\right\}$}{53/60}
\slideq{Fibres de $f$\linebreak Soit $x\in f^{-1}\l\left\{y\right\}\r$}{54/60}
\slider{$f^{-1}\l\left\{y\right\}\r=x\times\ker{f}$\linebreak${}=\left\{x\times z,\;z\in\ker{f}\right\}=\ker{f}\times x$}{54/60}
\slideq{Réciproque d'isomorphisme}{55/60}
\slider{Si $f\!:\!F\to F$ est un isomorphisme, alors $f^{-1}$ est un isomorphisme}{55/60}
\slideq{Passage au quotient de la loi dans le cas abélien\linebreak Si $G$ est un groupe abélien et $H$ un sous-groupe de $G$}{56/60}
\slider{${\equiv_g}={\equiv_d}$ et on note la relation $\equiv$\linebreak La loi induite corrrespond au produit des classes élément par élément : $\l ab\r H=\l aH\r\cdot\l bH\r=\left\{x\cdot y,\;x\in aH,\;y\in bH\right\}$\linebreak La loi induite sur l'ensemble quotient munit celui-ci d'une structure de groupe abélien}{56/60}
\slideq{Résolution de $x^n=1$}{57/60}
\slider{$\left\{n\in\mathbb{N}^*\;|\;x^n=e\right\}$ est de la forme $a\mathbb{Z}$\linebreak$\ord{x}\in\mathbb{N}\Leftrightarrow a\neq0$ et $\ord{x}=a$}{57/60}
\slideq{Soient $\l G,\star\r$ et $\l H,\diamond\r$ deux groupes\linebreak$f\!:\!G\to H$ est un homomorphisme de groupe}{58/60}
\slider{$\forall\l x,y\r\in G^2,\;f\l x\star y\r=f\l x\r\diamond f\l y\r$\linebreak L'ensemble des homomorphisme de $G$ dans $H$ est noté $\hom{G,H}$\linebreak Si $\l G,\star\r=\l H,\diamond\r$, f est un endomorphisme\linebreak L'ensemble des automorphismes de $G$ est noté $\aut{G}$}{58/60}
\slideq{Si $G$ et $H$ sont deux groupes et $f\in\hom{G,H}$\linebreak$f\l e_G\r$}{59/60}
\slider{$f\l e_H\r$}{59/60}
\slideq{Monoïde}{60/60}
\slider{Muni d'une loi d'une composition interne, de l'associativité et d'un élément neutre\linebreak Un monoïde est un magma}{60/60}
\end{document}