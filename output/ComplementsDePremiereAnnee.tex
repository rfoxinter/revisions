\documentclass[14pt,usepdftitle=false,aspectratio=169]{beamer}
\usepackage{preambule}
\setbeamercolor{structure}{fg=black}
\usepackage{analyse}\let\epsilon\varepsilon\usepackage{equivalents}
\hypersetup{pdftitle=Analyse -- Compléments de première année}
\title{Analyse\\\emph{Compléments de première année}}
\author{}
\date{}
\begin{document}
\begin{frame}
    \titlepage
\end{frame}
\slideq{$\epi f$}{1/8}
\slider{$\left\{\l x,y\r\in\mathbb R^2,y\geqslant f\l x\r\right\}$}{1/8}
\slideq{<< Réciproque >> du théorème de Bolzano-Weirstrass}{2/8}
\slider{Si $\l u_n\r$ est bornée et admet une unique valeur d'adhérence, alors elle converge}{2/8}
\slideq{Lemme de l'escalier}{3/8}
\slider{Si $u_{n+1}-u_n\to l\in\mathbb R$, alors, $\eq{u_n}{nl}$\linebreak Si $u_{n+1}-u_n\to0$, alors, $u_n=\o{n}$}{3/8}
\slideq{Caractérisation topologique des valeurs d'adhérence}{4/8}
\slider{$x\in\va u$\linebreak$\forall\epsilon>0,\;\forall N\in\mathbb N,\;\exists n\geqslant N,\;\left|u_n-x\right|\leqslant\epsilon$}{4/8}
\slideq{Suite de Cauchy}{5/8}
\slider{$\forall\epsilon>0,\;\exists N\in\mathbb N,\;\forall n\geqslant N,\;\left|u_n-u_m\right|\leqslant\epsilon$\linebreak Une suite de Cauchy à valeurs dans $\mathbb R$ ou $\mathbb C$ converge}{5/8}
\slideq{$\mathcal{C}^n$-difféomorphisme}{6/8}
\slider{Si $A\subset\mathbb R$, $B\subset\mathbb R$, alors $f\!:\!A\to B$ est un $\mathcal C^n$-difféomorphisme si c'est une application $\mathcal C^n$, bijective et dont la réciproque est $\mathcal C^n$}{6/8}
\slideq{Condition nécessaire simple pour réaliser un $\mathcal C^n$-difféomorphisme}{7/8}
\slider{Si $f$ est $\mathcal C^n$ et de dérivée ne s'annulant pas sur un intervalle $I$, alors $f$ est un $\mathcal C^n$-difféomorphisme de $I$ sur $f\l I\r$}{7/8}
\slideq{Caractérisation ensembliste des valeurs d'adhérence}{8/8}
\slider{$\va u=\bigcap{n\in\mathbb N}{}{\overline{\left\{u_k,k\geqslant n\right\}}}$}{8/8}
\end{document}