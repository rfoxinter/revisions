\documentclass[14pt,usepdftitle=false,aspectratio=169]{beamer}
\usepackage{preambule}
\setbeamercolor{structure}{fg=black}
\let\oldsup\sup\renewcommand\sup[2][]{\oldsup\limits_{#1}\l#2\r}\let\oldinf\inf\renewcommand\inf[2][]{\oldinf\limits_{#1}\l#2\r}
\hypersetup{pdftitle=Analyse -- Fonctions continues ou dérivables sur un intervalle}
\title{Analyse\\\emph{Fonctions continues ou dérivables sur un intervalle}}
\author{}
\date{}
\begin{document}
\begin{frame}
    \titlepage
\end{frame}
\slideq{Théorème de Rolle}{1/14}
\slider{Si $f\in\mathcal C^0\l\left[a,b\right]\r$ et $f\in\mathcal D^1\l\left]a,b\right[\r$, et $f\l a\r=f\l b\r$, alors il existe $c\in\left]a,b\right[$ tel que $f\l c\r=0$}{1/14}
\slideq{Théorème des acroissements finis}{2/14}
\slider{À faire}{2/14}
\slideq{Théorème de compacité}{3/14}
\slider{Soit $f\in\mathcal{C}^0\l \left[a,b\right]\r$ à valeurs dans $\mathbb R$, alors $f$ est bornée et atteint ses bornes}{3/14}
\slideq{Théorème de Rolle sur $\mathbb R$}{4/14}
\slider{À faire}{4/14}
\slideq{Théorème des valeurs intermédiaires\linebreak Si $f\in\mathcal C^0\l I\r$, avec $I$ un intervalle d'extrémités $\l a,b\r\in\overline{\mathbb R}$}{5/14}
\slider{Si $f\l a\r f\l b\r<0$, il existe $c\in\left]a,b\right[$ tel que $f\l c\r=0$\linebreak Pour tout $x\in\left]\inf[x\in I]{f\l x\r},\sup[x\in I]{f\l x\r}\right[$, il existe $c\in\left]a,b\right[$ tel que $f\l c\r=x$\linebreak L'image d'un intervalle par $f$ est un intervalle}{5/14}
\slideq{Théorème de Rolle itéré}{6/14}
\slider{À faire}{6/14}
\slideq{Théorème de Rolle pour un intervalle infini d'un côté}{7/14}
\slider{À faire}{7/14}
\slideq{Théorème de Heine dans $\mathbb C$}{8/14}
\slider{À faire}{8/14}
\slideq{Homéomorphisme}{9/14}
\slider{Si $A\subset\mathbb R$, $B\subset\mathbb R$, alors $f\!:\!A\to B$ est un homéomorphisme si c'est une application continue, bijective et dont la réciproque est continues}{9/14}
\slideq{Théorème de Heine}{10/14}
\slider{Si $f\in\mathcal{C}^0\l \left[a,b\right]\r$, alors $f$ est uniformément continue sur $\left[a,b\right]$}{10/14}
\slideq{Inégalité des acroissements finis}{11/14}
\slider{À faire}{11/14}
\slideq{Inégalité des acroissements finis dans $\mathbb C$}{12/14}
\slider{À faire}{12/14}
\slideq{Continuité uniforme}{13/14}
\slider{Si $X\subset\mathbb R$\linebreak$\forall \varepsilon>0,\;\exists\eta>0,\;\forall\l x,y\r\in X^2$\linebreak$\left|x-y\right|<\eta\Rightarrow\left|f\l x\r-f\l y\r\right|<\varepsilon$}{13/14}
\slideq{Compact}{14/14}
\slider{$K\subset\mathbb R$ est un compact si de toute suite $\l k_n\r$ de $K$, on peut extraire une suite convergente vers un élément de $K$}{14/14}
\end{document}