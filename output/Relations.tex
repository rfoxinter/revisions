\documentclass[14pt,usepdftitle=false,aspectratio=169]{beamer}
\usepackage{preambule}
\setbeamercolor{structure}{fg=black}
\newcommand\R{\mathcal{R}}
\hypersetup{pdftitle=Fondements -- Relations}
\title{Fondements\\\emph{Relations}}
\author{}
\date{}
\begin{document}
\begin{frame}
    \titlepage
\end{frame}
\slideq{ Réflexive, antisymétrique et transitive\linebreak Notée $\leqslant$ ou $\geqslant$}{1/11}
\slider{Relation d'ordre large}{1/11}
\slideq{ Irréflexive et transitive\linebreak Notée $<$ ou $>$}{2/11}
\slider{Relation d'ordre strict}{2/11}
\slideq{Irréflixivité ou antiréfléxivité}{3/11}
\slider{$\neg\left(x\R x\right)$}{3/11}
\slideq{Relation d'équivalence}{4/11}
\slider{ Réflexive, symétrique et transitive\linebreak Notée $\equiv$ ou $\sim$}{4/11}
\slideq{$\left(x\R y\right)\wedge\left(y\R x\right)\Rightarrow\left(x=y\right)$}{5/11}
\slider{Antisymértie}{5/11}
\slideq{Asymétrie}{6/11}
\slider{$\left(x\R y\right)\Rightarrow\neg\left(y\R x\right)$}{6/11}
\slideq{Symétrie}{7/11}
\slider{$x\R y\Rightarrow y\R x$}{7/11}
\slideq{La relation d'équivalence $\R$ est une congruence sur $\left(E,\times_1,\cdots,\times_n\right)$}{8/11}
\slider{$\forall\left(x,x^\prime,y,y^\prime\right)\in E^4,\forall i\in\llb1,n\rrb\linebreak\left(x\R x^\prime\right)\wedge\left(y\R y^\prime\right)\Rightarrow\left(x\times_iy\right)\R\left(x^\prime\times_iy^\prime\right)$}{8/11}
\slideq{Théorème de la factorisation d'une application constante sur les classes d'équivalences}{9/11}
\slider{$\left(\forall\left(x,y\right)\in E^2,x\R y\Rightarrow f\left(x\right)=f\left(y\right)\right)\linebreak\Leftrightarrow\left(\exists g\!:\!E/\R\to F\;|\;f=g\circ\pi_\R\right)$}{9/11}
\slideq{Réflexivité}{10/11}
\slider{$x\R x$}{10/11}
\slideq{$\left(x\R y\right)\wedge\left(y\R z\right)\Rightarrow\left(x\R z\right)$}{11/11}
\slider{Transitivité}{11/11}
\end{document}