\documentclass[14pt,usepdftitle=false,aspectratio=169]{beamer}
\usepackage{preambule}
\setbeamercolor{structure}{fg=black}
\usepackage{matrices}\usepackage{bigoperators}
\hypersetup{pdftitle=Algèbre 1 -- Matrices}
\title{Algèbre 1\\\emph{Matrices}}
\author{}
\date{}
\begin{document}
\begin{frame}
    \titlepage
\end{frame}
\slideq{Matrice élémentaite $E_{i,j}$ de $\mat{n}{p}{\mathbb{K}}$}{1/10}
\slider{$\forall\l k,l\r\in\llb1,n\rrb\times\llb1,p\rrb,\;e_{i,j}=\delta_{\l i,j\r,\l k,l\r}=\delta_{i,k}\delta_{j,l}$}{1/10}
\slideq{Propriété des matrices $E_{i,j}$ dans $\mat{n}{p}{\mathbb{K}}$}{2/10}
\slider{La famille $\l E_{i,j}\r_{\l i,j\r\in\llb1,n\rrb\times\llb1,p\rrb}$ est une base canonique de $\mat{n}{p}{\mathbb{K}}$\linebreak Toute matrice $A$ de $\mat{n}{p}{\mathbb{K}}$ peut s'exprimer de la forme $\sum{\l i,j\r\in\llb1,n\rrb\times\llb1,p\rrb}{}{\lambda_{i,j}E_{i,j}}$}{2/10}
\slideq{Description du produit matriciel par colonne\linebreak$\mathrm{C}_i\l M\r$ représente la $i$-ième colonne de $M$}{3/10}
\slider{$A\in\mat{n}{p}{\mathbb{K}}$, $B\in\mat{p}{q}{\mathbb{K}}$, $C\in\mat{n}{q}{\mathbb{K}}$\linebreak$C=A\times B$\linebreak$\forall k\in\llb1,q\rrb,\;\mathrm{C}_k\l{C}\r=\sum{j=1}{p}{b_{j,k}\mathrm{C}_j\l A\r}$}{3/10}
\slideq{Description du produit matriciel par ligne\linebreak$\mathrm{L}_i\l M\r$ représente la $i$-ième ligne de $M$}{4/10}
\slider{$A\in\mat{n}{p}{\mathbb{K}}$, $B\in\mat{p}{q}{\mathbb{K}}$, $C\in\mat{n}{q}{\mathbb{K}}$\linebreak$C=A\times B$\linebreak$\forall i\in\llb1,n\rrb,\;\mathrm{L}_i\l{C}\r=\sum{j=1}{p}{a_{i,j}\mathrm{L}_j\l B\r}$}{4/10}
\slideq{Matrice identité}{5/10}
\slider{$I_n=\begin{pNiceMatrix}[columns-width=24pt, margin]\mxh1&0&\cdots&0\\\mxh0&\ddots&\ddots&\vdots\\\mxh\vdots&\ddots&\ddots&0\\\mxh0&\cdots&0&1\\\end{pNiceMatrix}$}{5/10}
\slideq{Factorisation de $A^n-B^n$}{6/10}
\slider{$\l A,B\r\in \mat{n}{}{\mathbb{K}}^2$ tel que $AB=BA$\linebreak$\l A-B\r\sum{k=0}{n-1}{A^{n-k-1}B^k}$}{6/10}
\slideq{Factorisation de $\l A+B\r^n$}{7/10}
\slider{$\l A,B\r\in \mat{n}{}{\mathbb{K}}^2$ tel que $AB=BA$\linebreak$\sum{k=0}{n}{\binom{n}{k}A^kB^{n-k}}$}{7/10}
\slideq{Transposée de la matrice $A=\begin{pNiceMatrix}[columns-width=24pt, margin]\mxh a_{1,1}&\cdots&a_{1,p}\\\mxh\vdots&\ddots&\vdots\\\mxh a_{n,1}&\cdots&a_{n,p}\end{pNiceMatrix}$}{8/10}
\slider{$A^\top=\begin{pNiceMatrix}[columns-width=24pt, margin]\mxh a_{1,1}&\cdots&a_{n,1}\\\mxh\vdots&\ddots&\vdots\\\mxh a_{1,p}&\cdots&a_{n,p}\end{pNiceMatrix}$}{8/10}
\slideq{Structure de $\mat{n}{}{\mathbb{K}}$}{9/10}
\slider{$\mat{n}{}{\mathbb{K}}$ est un anneau non commutatif}{9/10}
\slideq{Définition du produit matriciel}{10/10}
\slider{$A\in\mat{n}{p}{\mathbb{K}}$, $B\in\mat{p}{q}{\mathbb{K}}$, $C\in\mat{n}{q}{\mathbb{K}}$\linebreak$C=A\times B$\linebreak$\forall\l i,k\r\in\llb1,n\rrb\times\llb1,q\rrb,\;c_{i,k}=\sum{j=1}{p}{a_{i,j}b_{j,k}}$}{10/10}
\end{document}