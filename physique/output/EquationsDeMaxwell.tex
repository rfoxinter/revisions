\documentclass[14pt,usepdftitle=false,aspectratio=169]{beamer}
\usepackage{preambule}
\setbeamercolor{structure}{fg=black}
\let\div\relax\DeclareMathOperator{\oldrot}{rot}\newcommand{\rot}{\mathop{\vec\oldrot}}\DeclareMathOperator{\div}{div}\usepackage{analyse}\usepackage{esint}\renewcommand{\d}{{{}\dd}}
\hypersetup{pdftitle=Équations de Maxwell}
\title{Équations de Maxwell}
\author{}
\date{}
\begin{document}
\begin{frame}
    \titlepage
\end{frame}
\slideq{Formule d'Ostrogradski}{1/9}
\slider{$\displaystyle{\varoiint_S}\vec F\cdot\d\vec S=\displaystyle{\iiint_V}\div\vec F\d\tau$}{1/9}
\slideq{Équation de Poisson}{2/9}
\slider{$\Delta V=-\frac\rho{\varepsilon_0}$}{2/9}
\slideq{Équation de Maxwell-Gauss}{3/9}
\slider{$\div\vec E=\frac\rho{\varepsilon_0}$}{3/9}
\slideq{Équation de Maxwell-Ampère}{4/9}
\slider{$\rot\vec B=\mu_0\vec\jmath+\mu_0\varepsilon_0\pder[][t]{\vec E}{}$}{4/9}
\slideq{Équation de Maxwell-Thomson}{5/9}
\slider{$\div\vec B=0$}{5/9}
\slideq{Formule de Stokes}{6/9}
\slider{$\displaystyle{\oint_\varGamma}\vec F\cdot\d\vec l=\displaystyle{\iint_S}\rot\vec F\cdot\d\vec S$}{6/9}
\slideq{Équation globale de la conservation de la charge}{7/9}
\slider{$\displaystyle{\iiint_V}\rho\d\tau+\displaystyle{\varoiint_S}\vec\jmath\cdot\d\vec S=0$}{7/9}
\slideq{Équation de Maxwell-Faraday}{8/9}
\slider{$\rot\vec E=-\pder[][t]{\vec B}{}$}{8/9}
\slideq{Équation locale de la conservation de la charge}{9/9}
\slider{$\pder[][t]{\rho}{}+\div\vec\jmath=0$}{9/9}
\end{document}