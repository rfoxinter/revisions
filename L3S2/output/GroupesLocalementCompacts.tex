\documentclass[14pt,usepdftitle=false,aspectratio=169]{beamer}
\usepackage{preambule}
\setbeamercolor{structure}{fg=black}
\usepackage{bigoperators}\togglebigoppar\DeclareMathOperator{\oldco}{Co}\newcommand{\Co}[2][]{\oldco_{#1}\l#2\r}
\hypersetup{pdftitle=Groupes localement compacts -- Groupes localement compacts}
\title{Groupes localement compacts\\\emph{Groupes localement compacts}}
\author{}
\date{}
\begin{document}
\begin{frame}
    \titlepage
\end{frame}
\slideq{Propriétés des sous-groupes d'un groupe localement compact et $\sigma$-compact}{1/12}
\slider{Si $H$ est un sous-groupe ouvert de $G$ alors $H$ st d'indice au plus dénombrable\linebreak Si $H$ est un sous-groupe fermé de $G$ alors $H$ est ouvert}{1/12}
\slideq{Propriétés d'un sous-groupe fermé d'un groupe localement compact}{2/12}
\slider{Si $G$ est localement compact et $H\leqslant G$ est ouvert alors $H$ est d'indice fini}{2/12}
\slideq{Isomorphisme de groupes topologiques}{3/12}
\slider{$G_1$ et $G_2$ sont isomorphes s'il existe $\varphi\!:\!G_1\to G_2$ qui est morphisme de groupes et un homéomorphisme}{3/12}
\slideq{Groupe profini}{4/12}
\slider{$G$ est profini s'il est isomorphe (en tant que groupe topologique) à un sous-groupe fermé d'un produit de groupes finis}{4/12}
\slideq{Théorème de Kakutani-Kodaira}{5/12}
\slider{Soit $G$ un groupe localement compact et $\sigma$-compact, alors pour tout voisinage $U$ de $1$, il existe un sous-groupe normal dans $G$ tel que $K\subset U$ et $G/K$ est métrisable (et $G/K$ est donc séparable)}{5/12}
\slideq{CNS de $X$ séparable si $X$ est métrisable et localement compact}{6/12}
\slider{$X$ admet une base dénombrable d'ouverts\linebreak$X$ est $\sigma$-compact}{6/12}
\slideq{Propriétés d'un sous-groupe fermé d'un groupe localement compact}{7/12}
\slider{Si $G$ est localement compact et $H\leqslant G$ est fermé alors $H$ et $G/H$ sont localement compacts\linebreak En particulier, si $H$ est dintingué dans $G$ alors $G/H$ est un groupe localement compact}{7/12}
\slideq{Théorème de van Dantzig}{8/12}
\slider{Si $G$ est un groupe localement compact totalement discontinu alors pour tout voisinage $U$ de $1$ dans $G$, il existe un sous-groupe $V$ compact et ouvert de $G$ et tel que $V\subset U$}{8/12}
\slideq{Propriété de $\Co[G]H=\bigcap{g\in G}{}{gHg^{-1}}$}{9/12}
\slider{$\Co[G]H$ est d'indice fini dans $G$ et $\Co[G]H\leqslant H$}{9/12}
\slideq{CNG pour avoir un groupe profini}{10/12}
\slider{$G$ est compact et totalement discontinu}{10/12}
\slideq{Espace $\sigma$-compact}{11/12}
\slider{Espace $X$ tel que $X=\bigcup{n\in\mathbb N}{}{K_n}$ avec les $K_n$ compacts}{11/12}
\slideq{Groupe localement compact}{12/12}
\slider{Groupe dont la topologie associée est localement compacte}{12/12}
\end{document}