\documentclass[14pt,usepdftitle=false,aspectratio=169]{beamer}
\usepackage{preambule}
\setbeamercolor{structure}{fg=black}
\newcommand\kk{\mathbb K}\renewcommand\ll{\mathbb L}\newcommand\mm{\mathbb M}\newcommand\ff{\mathbb F}\usepackage{structures}\DeclareMathOperator{\oldgal}{Gal}\newcommand{\gal}[2]{\oldgal\l#1/#2\r}
\hypersetup{pdftitle=Algèbre 2 -- Résolubilité par radicaux}
\title{Algèbre 2\\\emph{Résolubilité par radicaux}}
\author{}
\date{}
\begin{document}
\begin{frame}
    \titlepage
\end{frame}
\slideq{$G$ est résoluble}{1/10}
\slider{Les constituants simples $G_i/G_{i+1}$ de sa suite de composition maximale sont abéliens}{1/10}
\slideq{Transfert du caractère radical}{2/10}
\slider{Si $\ll/\kk$, $\ll'/\kk$ et $\mm/\ll$ sont radicales et $\ff/\kk$ est finie alors $\mm/\kk$ est radicale, $\ll\cdot\ll'/\kk$ est radicale, $\ll\cdot\ff/\ff$ est radicale\linebreak Si $\mm/\kk$ est radicale et $\ll$ est une extension intermédiaire alors $\mm/\kk$ est radicale}{2/10}
\slideq{Transfert du caractère résoluble}{3/10}
\slider{Si $G$ est résoluble alors $H\leqslant G$ est résoluble\linebreak$G$ est résoluble si et seulement si $H$ et $G/H$ sont résolubles pour $H\vartriangleleft G$}{3/10}
\slideq{Transfert du caractère résoluble par radicaux}{4/10}
\slider{Si $\ll/\kk$, $\ll'/\kk$ sont résolubles par radicaux si et seulement si $\ll\cdot\ll'/\kk$ est résoluble par radicaux\linebreak Si $\mm/\kk$ est radicale et $\ll$ est une extension intermédiaire alors $\mm/\kk$ est résoluble par radicaux}{4/10}
\slideq{CNS pour avoir $\ll/\kk$ résoluble par radicaux}{5/10}
\slider{Si $\car\kk=0$, $\ll/\kk$ résoluble par radicaux si et seulement si $\gal{\ll^\text{gal}}\kk$ est résoluble}{5/10}
\slideq{Exemples de groupes résolubles, non résolubles}{6/10}
\slider{$G$ abélien fini est résoluble\linebreak Si $\left|G\right|<60$ alors $G$ est résoluble\linebreak$\mathfrak S_n$ et $\mathfrak A_n$ ne sont par résolubles pour $n\geqslant5$}{6/10}
\slideq{$\ll/\kk$ est radicale}{7/10}
\slider{Il existe une tour d'extensions $\kk=\ll_0\subset\ll_1\subset\cdots\subset\ll_n=\ll$ avec $\ll_{i+1}/\ll_i$ radicielle}{7/10}
\slideq{Théorème de Jordan-Hölder}{8/10}
\slider{Deux suites de composition maximales sont équivalentes et la suite non ordonnée et avec multiplicité des constituants simples $G_i/G_{i+1}$ est intrinsèque au groupe $G$}{8/10}
\slideq{Suite de composition pour $G$ fini}{9/10}
\slider{Suite de groupes $\left\{0\right\}=G_r\subset G_{r-1}\subset\cdots\subset G_0=0$ telle que $G_{i+1}\vartriangleleft G_i$}{9/10}
\slideq{Relations sur les suites de composition}{10/10}
\slider{$\varSigma_1\subset\varSigma_2$ si $\varSigma_1$ s'obtient en supprimant des termes de $\varSigma_2$\linebreak$\varSigma_1\sim\varSigma_2$ s'il existe $\sigma\in\mathfrak S_n$ telle que $G_i/G_{i+1}\cong H_{\sigma\l i\r}/H_{\sigma\l i\r+1}$}{10/10}
\end{document}