\documentclass[14pt,usepdftitle=false,aspectratio=169]{beamer}
\usepackage{preambule}
\setbeamercolor{structure}{fg=black}
\usepackage{probas}\usepackage{usuelles}\usepackage{complexes}\usepackage{bigoperators}
\hypersetup{pdftitle=Concentration de la mesure -- Inégalités de concentration}
\title{Concentration de la mesure\\\emph{Inégalités de concentration}}
\author{}
\date{}
\begin{document}
\begin{frame}
    \titlepage
\end{frame}
\slideq{Borne de Chernov}{1/10}
\slider{$\p{X\geqslant t}\leqslant\e^{\psi^*\l t\r}$ où $\psi^*\l t\r=-\oldsup\limits_{\lambda\geqslant0}\l\lambda t-\psi\l\lambda\r\r$}{1/10}
\slideq{Inégalité de Hoeffding}{2/10}
\slider{Si $X_1,\cdots,X_n$ sont des varaibles aléatoires indépendantes avec $X_i$ à valeurs dans $\left[a_i,b_i\right]$ et si $S_n=X_1+\cdots+X_n$ alors\linebreak$\p{\left|S_n-\esp{S_n}\right|\geqslant t}\leqslant2\exp{\frac{-2t^2}{\oldsum\limits_{i=1}^{n}\l b_i-a_i\r^2}}$}{2/10}
\slideq{Inégalité de Bienaymé-Tchebychev}{3/10}
\slider{$\forall t>0$, $\p{\left|X-\esp X\right|\geqslant t}\leqslant\frac{\var X}{t^2}$}{3/10}
\slideq{Inégalité de Chernov}{4/10}
\slider{Si $X_1,\cdots,X_n$ sont des varaibles de Bernoulli indépendantes avec $X_i$ de paramètre $p_i$ et si $S_n=X_1+\cdots+X_n$ et $\mu=p_1+\cdots+p_n$ alors $\p{S_n\geqslant t}\leqslant\e^{-\mu}\l\frac{\e\mu}{t}\r^t$}{4/10}
\slideq{$\oldinf\limits_{a\in\mathbb R}\l\esp{\l X-a\r^2}\r$}{5/10}
\slider{$\esp{\left|X-\esp X\right|}=\var X$}{5/10}
\slideq{Inégalité de Markov}{6/10}
\slider{$\p{X\geqslant t}\leqslant\frac{\esp X}\lambda$}{6/10}
\slideq{Généralisation de l'inégalité de Bienaymé-Tchebychev}{7/10}
\slider{$\forall t>0$, $\forall a\in\mathbb R$, $\p{\left|X-a\right|\geqslant t}\leqslant\frac{\esp{\left|X-a\right|^p}}{t^p}$}{7/10}
\slideq{$f\!:\!\mathbb R^n\to\mathbb R$ vérifie une inégalité de concentration de concentration $\alpha$}{8/10}
\slider{$\exists a\in\mathbb R$, $\forall t\geqslant0$, $\mu\l\left\{x\in\mathbb R^n,\left|f\l x\r-a\right|\geqslant t\right\}\r\leqslant\alpha\l t\r$}{8/10}
\slideq{Transformée log-Laplace de $X$}{9/10}
\slider{$\psi\l\lambda\r=\esp{\e^{\lambda X}}$\linebreak$\psi$ est convexe}{9/10}
\slideq{$\oldinf\limits_{a\in\mathbb R}\l\esp{\left|X-a\right|}\r$}{10/10}
\slider{$\esp{\left|X-m_X\right|}$ avec $m_X$ une médiane de $X$}{10/10}
\end{document}