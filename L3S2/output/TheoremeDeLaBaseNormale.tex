\documentclass[14pt,usepdftitle=false,aspectratio=169]{beamer}
\usepackage{preambule}
\setbeamercolor{structure}{fg=black}
\usepackage{bigoperators}\def\kk{\mathbb K}\def\ll{\mathbb L}\DeclareMathOperator{\oldgal}{Gal}\def\gal#1#2{\oldgal\l#1/#2\r}
\hypersetup{pdftitle=Algèbre 2 -- Théorème de la base normale}
\title{Algèbre 2\\\emph{Théorème de la base normale}}
\author{}
\date{}
\begin{document}
\begin{frame}
    \titlepage
\end{frame}
\slideq{Théorème de la base normale\linebreak Existence}{1/4}
\slider{Si $\ll/\kk$ est une extension galoisienne finie alors il existe une base normale de $\ll/\kk$}{1/4}
\slideq{Base normale de $\ll/\kk$}{2/4}
\slider{Base de la forme $\left\{g\l h\r,g\in\gal\ll\kk\right\}$ pour un certain $h\in\kk$}{2/4}
\slideq{Théorème de la base normale\linebreak Morphisme}{3/4}
\slider{Si $\ll/\kk$ est une extension galoisienne finie alors on dispose d'un isomorphisme qui commute avec l'action de $\gal\ll\kk$ donné par $\kk\left[\gal\ll\kk\right]\cong\ll$, $\sum{g\in\gal\ll\kk}{}{a_gg}\mapsto\sum{g\in\gal\ll\kk}{}{a_gb_g}$ où $\left\{b_g,g\in\gal\ll\kk\right\}$ est une $\kk$-base de $\ll$}{3/4}
\slideq{$\mathbb K\left[G\right]$}{4/4}
\slider{$\left\{\sum{g\in G}{}{a_gg}, a_g\in\mathbb K\right\}$}{4/4}
\end{document}