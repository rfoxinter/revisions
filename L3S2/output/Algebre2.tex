\documentclass[14pt,usepdftitle=false,aspectratio=169]{beamer}
\usepackage{preambule}
\setbeamercolor{structure}{fg=black}
\newcommand\kk{\mathbb K}\renewcommand\ll{\mathbb L}\newcommand\mm{\mathbb M}\usepackage{polynomes}\DeclareMathOperator{\dec}{D}\DeclareMathOperator{\olddegsep}{deg sep}\newcommand{\degsep}[2][]{\olddegsep_{#1}\l#2\r}
\hypersetup{pdftitle=Racines des polynômes minimaux -- Algèbre 2}
\title{Racines des polynômes minimaux\\\emph{Algèbre 2}}
\author{}
\date{}
\begin{document}
\begin{frame}
    \titlepage
\end{frame}
\slideq{Séparabilité dans les tours d'extensions}{1/4}
\slider{Si $\ll/\mathbb F/\kk$ sont des extensions algèbriques et $\alpha\in\ll$ est séparable sur $\kk$ alors $\alpha$ est séparable sur $\mathbb F$}{1/4}
\slideq{$\alpha$ est inséparable sur $\kk$}{2/4}
\slider{Il existe $Z_\alpha\in\pol KX$ irréductible tel que $P_{\alpha,\kk}=S_\alpha\l X^{p^n}\r$\linebreak$\rac[\ll]{P_{\alpha,\kk}}\hookrightarrow\rac[\ll]{S_\alpha}$ et c'est une bijection si et seulement si $\dec_\kk\l P\r\subset\ll$\linebreak$\alpha^{p^n}$ est séparable avec $P_{\alpha^{p^n},\kk}=S_\alpha$\linebreak$\val[\alpha]{P_{\alpha,\kk}}=p^n$}{2/4}
\slideq{$\degsep[\kk]\alpha$}{3/4}
\slider{$\deg{S_\alpha}$}{3/4}
\slideq{$\alpha$ est séparable sur $\kk$}{4/4}
\slider{$P_{\alpha,\kk}$ n'a que des racines simples}{4/4}
\end{document}