\documentclass[14pt,usepdftitle=false,aspectratio=169]{beamer}
\usepackage{preambule}
\setbeamercolor{structure}{fg=black}
\usepackage{bigoperators,arithmetique,analyse,usuelles}\newcommand{\zz}[1][p]{\mathbb Z_{#1}}
\hypersetup{pdftitle=Groupes localement compacts -- Nombres \textit{p}-adiques}
\title{Groupes localement compacts\\\emph{Nombres \textit{p}-adiques}}
\author{}
\date{}
\begin{document}
\begin{frame}
    \titlepage
\end{frame}
\slideq{$\zz$}{1/9}
\slider{$\left\{\l x_n\r\in\prod{n\geqslant1}{}{\mathbb Z/p^n\mathbb Z},\forall n\in\mathbb N^*,\cgr{x_{n+1}}{x_n}{p^n}\right\}$\linebreak C'est un anneau intègre pour les opérations coordonnée par coordonnée\linebreak Le morphisme d'anneau $\appl{i}{\mathbb Z}{\zz}{x}{\l x\bmod p^n\r}$ est injectif}{1/9}
\slideq{Valuation $p$-adique}{2/9}
\slider{$v_p\l x\r$ est le plus petit $n\in\mathbb N$ pour lequel $x=u\times p^n$ avec $u\in\zz^*$\linebreak$v_p\l0\r=0$\linebreak$v_p\l xy\r=v_p\l x\r +v_p\l y\r$\linebreak$v_p\l x+y\r\geqslant\min{v_p\l x\r,v_p\l y\r}$}{2/9}
\slideq{Distance sur $\zz$}{3/9}
\slider{$\appl{d}{\zz\times\zz}{\mathbb{R}_+}{\l x,y\r}{\left|x-y\right|_p=p^{-v_p\l x-y\r}}$\linebreak C'est une distance ultramétrique invariante par translation}{3/9}
\slideq{$\mathcal B_F\l x,p^n\r$}{4/9}
\slider{$x+p^n\zz$}{4/9}
\slideq{Propriétés topologiques de $\mathcal B_F\l x,r\r$ et $\mathcal B_O\l x,r\r$}{5/9}
\slider{Ce sont des ouverts-fermés}{5/9}
\slideq{Unités $p$-adiques}{6/9}
\slider{$\zz^*$\linebreak$\zz^*=\left\{\l x_n\r\in\zz,x_1\neq0\right\}$}{6/9}
\slideq{$\mathcal B_O\l x,r\r$}{7/9}
\slider{$\mathcal B_F\l x,p^{-m}\r$ où $m=\min{\left\{n\in\mathbb N,p^{-n}\leqslant r\right\}}$}{7/9}
\slideq{Propriété de la topologie induite sur $\zz$ par sa distance}{8/9}
\slider{C'est la topologie induite par $\prod{n\geqslant1}{}{\mathbb Z/p^n\mathbb Z}$\linebreak En particulier, $\zz$ est compact}{8/9}
\slideq{Décoposition classique d'un élément de $\zz$}{9/9}
\slider{$u\times p^n$ où $u\in\zz^*$}{9/9}
\end{document}