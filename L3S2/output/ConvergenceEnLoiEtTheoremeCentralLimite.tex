\documentclass[14pt,usepdftitle=false,aspectratio=169]{beamer}
\usepackage{preambule}
\setbeamercolor{structure}{fg=black}
\usepackage{analyse}\toggleanalysepar\usepackage{probas}
\hypersetup{pdftitle=Intégration et probabilités -- Convergence en loi et théorème central limite}
\title{Intégration et probabilités\\\emph{Convergence en loi et théorème central limite}}
\author{}
\date{}
\begin{document}
\begin{frame}
    \titlepage
\end{frame}
\slideq{Convergence étroite}{1/2}
\slider{$\l\mu_n\r_{n\in\llb1,+\infty\rrb}$ une suite de mesures de probabilités sur un espace métrique $\l E,d\r$ converge étroitement vers $\mu_\infty$ si pour toute $f$ continue et bornée alors $\int[x][E][][\mu_n]{f\l x\r}\xrightarrow[n\to+\infty]{}\int[x][E][][\mu_\infty]{f\l x\r}$}{1/2}
\slideq{Convergence en loi}{2/2}
\slider{$\l X_n\r$ une suite de variables aléatoires dans $\l E,d\r$ converge en loi vers $X$ si $\mathbb P_{X_n}$ converge étroitement vers $\mathbb P_X$\linebreak De manière équivalente, si pour toute $f$ continue et bornée, $\esp{f\l X_n\r}\xrightarrow[n\to+\infty]{}\esp{f\l X\r}$}{2/2}
\end{document}