\documentclass[14pt,usepdftitle=false,aspectratio=169]{beamer}
\usepackage{preambule}
\setbeamercolor{structure}{fg=black}
\usepackage{analyse}\toggleanalysepar\usepackage{probas}
\hypersetup{pdftitle=Intégration et probabilités -- Convergence en loi et théorème central limite}
\title{Intégration et probabilités\\\emph{Convergence en loi et théorème central limite}}
\author{}
\date{}
\begin{document}
\begin{frame}
    \titlepage
\end{frame}
\slideq{Convergence en loi}{1/5}
\slider{$\l X_n\r$ une suite de variables aléatoires dans $\l E,d\r$ converge en loi vers $X$ si $\mathbb P_{X_n}$ converge étroitement vers $\mathbb P_X$\linebreak De manière équivalente, si pour toute $f$ continue et bornée, $\esp{f\l X_n\r}\xrightarrow[n\to+\infty]{}\esp{f\l X\r}$}{1/5}
\slideq{Lemme de Scheffé}{2/5}
\slider{Si les $\l f_n\r_{n\in\llb1,+\infty\rrb}$ sont des densités de mesures de probabilités et si pour $\lambda$ presque tout $x\in\mathbb R^d$, $f_n\l x\r\xrightarrow[n\to+\infty]{}f_\infty\l x\r$ alors, pour $\l X_n\r$ tel que $\mathbb P_{X_n}\l\dd x\r=f_n\l x\r\dd x$, alors $X_n\xrightarrow[n\to+\infty]{\text{loi}}X_\infty$}{2/5}
\slideq{Convergence étroite}{3/5}
\slider{$\l\mu_n\r_{n\in\llb1,+\infty\rrb}$ une suite de mesures de probabilités sur un espace métrique $\l E,d\r$ converge étroitement vers $\mu_\infty$ si pour toute $f$ continue et bornée alors $\int[x][E][][\mu_n]{f\l x\r}\xrightarrow[n\to+\infty]{}\int[x][E][][\mu_\infty]{f\l x\r}$}{3/5}
\slideq{Convergence étroite pour des variables aléatoires à valeurs dans $\mathbb N$}{4/5}
\slider{Si $\l X_n\r_{n\in\llb1,+\infty\rrb}$ sont des variables aléatoires à valeurs dans $\mathbb N$ alors $X_n\xrightarrow[n\to+\infty]{\text{loi}}X_\infty$ si et seulement si, pour tout $x\in\mathbb N$, $\p{X_n=k}\xrightarrow[n\to+\infty]{}\p{X_\infty=k}$}{4/5}
\slideq{Lien entre convergence en probabilités et convergence en loi}{5/5}
\slider{Si $\l X_n\r$ converge en probabilités vers $X$ sur $\l E,d\r$ alors $\l X_n\r$ converge en loi vers $X$\linebreak Si $\l X_n\r$ converge en loi vers une constante alors $\l X_n\r$ converge en probabilités vers cette constante}{5/5}
\end{document}