\documentclass[14pt,usepdftitle=false,aspectratio=169]{beamer}
\usepackage{preambule}
\setbeamercolor{structure}{fg=black}
\usepackage{analyse,al,matrices}\toggleanalysepar\usepackage{probas}\usepackage{usuelles}
\hypersetup{pdftitle=Intégration et probabilités -- Convergence en loi et théorème central limite}
\title{Intégration et probabilités\\\emph{Convergence en loi et théorème central limite}}
\author{}
\date{}
\begin{document}
\begin{frame}
    \titlepage
\end{frame}
\slideq{Caractérisation de la convergence en loi par la fonction de répartition}{1/19}
\slider{Si $\l X_n\r_{n\in\mathbb N}$ et $X$ sont des variables aléatoires réelles, alors $X_n\xrightarrow[n\to+\infty]{\text{loi}}X$ si et seulement si pour tout $x$ tel que $F_X$ est continue en $x$, $F_{X_n}\l x\r\xrightarrow[n\to+\infty]{}F_X\l x\r$}{1/19}
\slideq{Conséquence du théorème de Helly pour la convergence en probabilités}{2/19}
\slider{Si $\l X_n\r$ est une suite de variables aléatoires à valeurs dans $\mathbb R$ (ou $\mathbb R^d$), si $\sup{\p{\left|X_n\right|>K}}\xrightarrow[K\to+\infty]{}0$ alors il existe $\l X_{n_k}\r$ qui converge en loi vers $X$}{2/19}
\slideq{$AX$ pour $\mathcal M_{k,d}\l\mathbb R\r$ et $X\sim\normal m\varSigma$}{3/19}
\slider{$AX\sim\normal{Am}{A\varSigma A^\top}$\linebreak En particulier, si $A\in O_d\l\mathbb R\r$ et $X\sim\normal0{I_d}$ alors $X$ et $AX$ ont la même loi}{3/19}
\slideq{Convergence étroite}{4/19}
\slider{$\l\mu_n\r_{n\in\llb1,+\infty\rrb}$ une suite de mesures de probabilités sur un espace métrique $\l E,d\r$ converge étroitement vers $\mu_\infty$ si pour toute $f$ continue et bornée alors $\int[x][E][][\mu_n]{f\l x\r}\xrightarrow[n\to+\infty]{}\int[x][E][][\mu_\infty]{f\l x\r}$}{4/19}
\slideq{Théorème de sélection de Helly}{5/19}
\slider{Si $\l F_n\r$ est une suite de fonctions de répartition alors il existe $\l F_{n_k}\r$ qui converge simplement vers $F$ croissante, continue à droite et à valeurs dans $\left[0,1\right]$ tel que pour tout $x$ tel que $F_X$ est continue en $x$, $F_{X_n}\l x\r\xrightarrow[n\to+\infty]{}F_X\l x\r$}{5/19}
\slideq{Théorème de convergence de Lévy\linebreak Version forte}{6/19}
\slider{Si $\l X_n\r$ est une suite de variables aléatoires à valeurs dans $\mathbb R^d$ telle que $\varphi_{X_n}\l\xi\r\xrightarrow[n\to+\infty]{}\psi\l x\r$ continue en $0$ alors il existe une variable aléatoire $X$ telle que $\psi=\varphi_X$ et $X_n\xrightarrow[n\to+\infty]{\text{loi}}X$\linebreak Réciproquement, si $X_n\xrightarrow[n\to+\infty]{\text{loi}}X$ alors $\varphi_{X_n}\l\xi\r\xrightarrow[n\to+\infty]{}\varphi\l\xi\r$ pour tout $\xi\in\mathbb R^d$}{6/19}
\slideq{Lemme de Scheffé}{7/19}
\slider{Si les $\l f_n\r_{n\in\llb1,+\infty\rrb}$ sont des densités de mesures de probabilités et si pour $\lambda$ presque tout $x\in\mathbb R^d$, $f_n\l x\r\xrightarrow[n\to+\infty]{}f_\infty\l x\r$ alors, pour $\l X_n\r$ tel que $\mathbb P_{X_n}\l\dd x\r=f_n\l x\r\dd x$, alors $X_n\xrightarrow[n\to+\infty]{\text{loi}}X_\infty$}{7/19}
\slideq{Théorème de Portemanteau}{8/19}
\slider{{\normalsize Si $\l X_n\r_{n\in\mathbb N}$ et $X$ sont des variables à valeurs dans $\l E,d\r$ métrique, il y a équivalence entre\linebreak$X_n\xrightarrow[n\to+\infty]{\text{loi}}X$\linebreak$\forall f\!:\!E\to\mathbb R$, $1$-lipschitzienne bornée, $\esp{f\l X_n\r}\xrightarrow[n\to+\infty]{}\esp{f\l X\r}$\linebreak$\forall O\subset E$ ouvert, $\limi[n\to+\infty]{\p{X_n\in O}}\geqslant\p{X\in O}$\linebreak$\forall F\subset E$ fermé, $\lims[n\to+\infty]{\p{X_n\in F}}\leqslant\p{X\in F}$\linebreak$\forall A\in\mathcal B\l E\r$, $\p{X\in\partial A}=0\Rightarrow\lim[n\to+\infty]{\p{X_n\in A}}=\p{X\in A}$\linebreak$\forall f\!:\!E\to\mathbb R$ bornée, continue $\mathbb P_X$-pp, $\esp{f\l X_n\r}\xrightarrow[n\to+\infty]{}\esp{f\l X\r}$}}{8/19}
\slideq{Théorème central limite sur $\mathbb R$}{9/19}
\slider{Soit $\l X_i\r$ une suite de variables aléatoires iid dans $L^2$ telles que $\var{X_n}>0$ alors $\frac{X_1+\cdots+X_n-n\esp{X_1}}{\sqrt{n\var{X_1}}}\xrightarrow[n\to+\infty]{\text{loi}}N$ où $N\sim\normal01$\linebreak Ou $\frac{X_1+\cdots+X_n-n\esp{X_1}}{\sqrt n}\xrightarrow[n\to+\infty]{\text{loi}}N$ où $N\sim\normal0{\var{X_1}}$}{9/19}
\slideq{Condition d'indépendance de vecteurs gaussiens}{10/19}
\slider{Si $Z=\l X_1,\cdots,X_d,Y_1,\cdots,Y_k\r$ est un vecteur gaussien tel que pour tout $i\in\llb1,d\rrb$ et tout $j\in\llb1,k\rrb$, $\cov{X_i}{Y_j}=0$ alors $X=\l X_1,\cdots,X_d\r$ et $Y=\l Y_1,\cdots,Y_d\r$ sont indépendants}{10/19}
\slideq{Théorème de convergence de Lévy\linebreak Version faible}{11/19}
\slider{Si $\l X_n\r$ est une suite de variables aléatoires à valeurs dans $\mathbb R^d$ telle que $\varphi_{X_n}\l\xi\r\xrightarrow[n\to+\infty]{}\varphi_X\l x\r$ alors $X_n\xrightarrow[n\to+\infty]{\text{loi}}X$\linebreak Réciproquement, si $X_n\xrightarrow[n\to+\infty]{\text{loi}}X$ alors $\varphi_{X_n}\l\xi\r\xrightarrow[n\to+\infty]{}\varphi\l\xi\r$ pour tout $\xi\in\mathbb R^d$}{11/19}
\slideq{Densité des lois gaussiennes}{12/19}
\slider{Si $X\sim\normal m\varSigma$ et $\rg\varSigma=d$  alors $X$ est à densité $g_{m,\varSigma}=\frac1{\sqrt{\det{2\pi\varSigma}}}\exp{-\frac{\varSigma^{-1}\l x-m\r\cdot\l x-m\r}2}$\linebreak Si $\rg\varSigma<d$, $X$ n'est pas à densité}{12/19}
\slideq{Lien entre convergence en probabilités et convergence en loi}{13/19}
\slider{Si $\l X_n\r$ converge en probabilités vers $X$ sur $\l E,d\r$ alors $\l X_n\r$ converge en loi vers $X$\linebreak Si $\l X_n\r$ converge en loi vers une constante alors $\l X_n\r$ converge en probabilités vers cette constante}{13/19}
\slideq{Restriction des fonctions test pour la convergence en probabilités sur $\mathbb R^d$}{14/19}
\slider{Si $H$ est un ensemble de fonctions mesurables $\mathbb R^d\to\mathbb R$ dont l'adhérence pour $\anrm\cdot$ contient $\mathcal C_c\l\mathbb R^d,\mathbb R\r$ alors si $\l X_n\r_{n\in\mathbb N}$ et $X$ sont des variables aléatoires dans $\mathbb R^d$, si $\esp{f\l X_n\r}\xrightarrow[n\to+\infty]{}\esp{f\l X\r}$ pour tout $f\in H$ alors $X_n\xrightarrow[n\to+\infty]{\text{loi}}X$}{14/19}
\slideq{Convergence en loi}{15/19}
\slider{$\l X_n\r$ une suite de variables aléatoires dans $\l E,d\r$ converge en loi vers $X$ si $\mathbb P_{X_n}$ converge étroitement vers $\mathbb P_X$\linebreak De manière équivalente, si pour toute $f$ continue et bornée, $\esp{f\l X_n\r}\xrightarrow[n\to+\infty]{}\esp{f\l X\r}$}{15/19}
\slideq{Vecteur gaussien}{16/19}
\slider{$X$ est un vecteur gaussien si pour tout $\xi\in\mathbb R^d$, $\xi\cdot X$ est une variable gaussienne, ie $\xi\cdot X\sim\normal{m_\xi}{\sigma_\xi^2}$\linebreak$\normal m0=\delta_m$}{16/19}
\slideq{Stabilité de la convergence en loi}{17/19}
\slider{Si $X_n\xrightarrow[n\to+\infty]{\text{loi}}X_\infty$ et $f\in\mathcal C_b\l E,F\r$ avec $F$ un espace métrique alors $f\l X_n\r\xrightarrow[n\to+\infty]{\text{loi}}f\l X_\infty\r$}{17/19}
\slideq{Convergence étroite pour des variables aléatoires à valeurs dans $\mathbb N$}{18/19}
\slider{Si $\l X_n\r_{n\in\llb1,+\infty\rrb}$ sont des variables aléatoires à valeurs dans $\mathbb N$ alors $X_n\xrightarrow[n\to+\infty]{\text{loi}}X_\infty$ si et seulement si, pour tout $x\in\mathbb N$, $\p{X_n=k}\xrightarrow[n\to+\infty]{}\p{X_\infty=k}$}{18/19}
\slideq{Théorème central limite sur $\mathbb R^d$}{19/19}
\slider{Soit $\l X_i\r$ une suite de vecteurs aléatoires iid dans $\mathbb R^d$ à coordonnées dans $L^2$ alors $\frac{X_1+\cdots+X_n-n\esp{X_1}}{\sqrt{n}}\xrightarrow[n\to+\infty]{\text{loi}}N$ où $N\sim\normal0{\varSigma_X}$ où $\varSigma_X=\var X$}{19/19}
\end{document}