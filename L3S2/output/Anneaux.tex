\documentclass[14pt,usepdftitle=false,aspectratio=169]{beamer}
\usepackage{preambule}
\setbeamercolor{structure}{fg=black}
\usepackage{structures}\usepackage{al}\usepackage{bigoperators}\togglebigoppar\DeclareMathOperator{\ev}{ev}
\hypersetup{pdftitle=Algèbre 2 -- Anneaux}
\title{Algèbre 2\\\emph{Anneaux}}
\author{}
\date{}
\begin{document}
\begin{frame}
    \titlepage
\end{frame}
\slideq{CNS pour $P\vartriangleleft A$ premier}{1/27}
\slider{$A/P$ est intègre}{1/27}
\slideq{$a\in A$ est idempotent}{2/27}
\slider{$a^2=a$}{2/27}
\slideq{$I\vartriangleleft A$}{3/27}
\slider{$\l I,+\r\leqslant\l A,+\r$ et $\forall a\in A$, $\forall i\in I$, $ai\in I$}{3/27}
\slideq{Caractéristique de $A$}{4/27}
\slider{$n$ tel que $\mathbb Z/n\mathbb Z=\ker\varphi$\linebreak$\appl{\varphi}{\mathbb{Z}}{A}{n}{n\cdot 1_A}$}{4/27}
\slideq{$a\in A$ est irréductible}{5/27}
\slider{$a\notin A^\times$, $a\neq0$, $a=bc\Rightarrow a\in A^\times\lor c\in A^\times$}{5/27}
\slideq{$I\cdot J$ pour $I+J=A$}{6/27}
\slider{$I\cap J$}{6/27}
\slideq{Lemme de factorisation des morphismes d'anneaux}{7/27}
\slider{Si $f\!:\!A\to B$ est un morphisme d'anneaux alors il existe un unique morphisme d'anneaux $\overline f\!:\!A/\ker f\to B$ tel que $f=\overline f\circ\pi$}{7/27}
\slideq{$a\in A$ est nilpotent}{8/27}
\slider{Il existe $n\in\mathbb N$ tel que $a^n=0$}{8/27}
\slideq{Conservation de la primalité par morphisme}{9/27}
\slider{$f^{-1}\l P\r$ est premier\linebreak Si $f$ est surjectif alors $f\l A\r$ est premier}{9/27}
\slideq{$a\in A$ est diviseur de $0$}{10/27}
\slider{Il existe $b\in A^\times$ tel que $ab=0$}{10/27}
\slideq{$I\vartriangleleft A$ et $J\vartriangleleft A$ sont premiers entre eux}{11/27}
\slider{$I+J=A$}{11/27}
\slideq{Théorème de localisation}{12/27}
\slider{Si $S$ est une partie multiplicative de $A$, il existe un anneau $S^{-1}A$ et un morphisme $\varphi_S\!:\!S^{-1}A\to A$ tel que si $B$ est un anneau et $f\!:\!A\to B$ est un morphisme tel que $f\l s\r\in A^\times$ pour tout $s\in S$ alors il existe un unique morphisme $\widetilde f\!:\!S^{-1}A\to B$ tel que $f=\widetilde f\circ\varphi_S$ et de plus, pour tout $s\in S$, $\varphi_S\l s\r\in\l S^{-1}A\r^\times$}{12/27}
\slideq{Lien entre premier et irréductible}{13/27}
\slider{Si $A$ est intègre et $x\in A$ est premier alors $x$ est irréductible}{13/27}
\slideq{$A$ est intègre}{14/27}
\slider{$A$ ne possède pas de diviseurs de $0$}{14/27}
\slideq{Propriété de $G\leqslant A^\times$ fini}{15/27}
\slider{Un tel $G$ est cyclique}{15/27}
\slideq{PU du produit d'anneaux}{16/27}
\slider{Pour tout anneau $B$ et tout morphisme $f\!:\!B\to A_i$ il existe un unique $f\!:\!B\to\prod{i\in I}{}{A_i}$ tel que $\pi_i\circ f=f_i$}{16/27}
\slideq{Théorème chinois}{17/27}
\slider{Si $I_1,\cdots,I_n\vartriangleleft A$ sont deux à deux premiers entre eux, alors on a un isomorphisme d'anneaux $f\!:\!A/\l I_1\cdot\cdots\cdot I_n\r\to\prod{i=1}{n}{A/I_i}$}{17/27}
\slideq{Corps des fractions}{18/27}
\slider{Si $A$ est intègre, le corps des fractions de $A$ est $S^{-1}A$ pour $S=A^*$\linebreak$S^{-1}A$ est alors un corps}{18/27}
\slideq{CNS pour $M\vartriangleleft A$ maximal}{19/27}
\slider{$A/M$ est un corps}{19/27}
\slideq{PU du $A$-module $A\left[X\right]$}{20/27}
\slider{Pour tout $f\!:\!A\to B$ et $b\in B$, il existe un unique $\appl{\ev_{f,b}}{A\left[X\right]}{B}{\sum{i=0}{n}{a_iX^i}}{\sum{i=1}{n}{a_ib^i}}$}{20/27}
\slideq{$a\in A$ est racine de l'unité}{21/27}
\slider{Il existe $n\in\mathbb N$ tel que $a^n=1$}{21/27}
\slideq{$I+J$\linebreak$I\vartriangleleft A$ et $J\vartriangleleft A$}{22/27}
\slider{$I+J=\l I\cup J\r\vartriangleleft A$}{22/27}
\slideq{Image d'un idéal par un morphisme d'anneaux}{23/27}
\slider{$f\l I\r$ est un idéal si $f$ est surjectif}{23/27}
\slideq{$S$ est une partie multiplicative}{24/27}
\slider{$1\in S$\linebreak$\forall\l a,b\r\in S^2$, $ab\in S$}{24/27}
\slideq{$p\in A$ est premier}{25/27}
\slider{$p\notin A^\times$, $p\neq0$, $p\mid ab\Rightarrow p\mid a\lor p\mid b$}{25/27}
\slideq{$I\cdot J$\linebreak$I\vartriangleleft A$ et $J\vartriangleleft A$}{26/27}
\slider{$I\cdot J=\l i\times j, i\in I,j\in J\r\vartriangleleft A$}{26/27}
\slideq{$a\mid b$}{27/27}
\slider{$\exists c\in A$, $b=ac$}{27/27}
\end{document}