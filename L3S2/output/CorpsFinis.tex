\documentclass[14pt,usepdftitle=false,aspectratio=169]{beamer}
\usepackage{preambule}
\setbeamercolor{structure}{fg=black}
\newcommand\kk{\mathbb K}\newcommand\ff{\mathbb F}\usepackage{structures}\DeclareMathOperator{\oldfrob}{frob}\newcommand{\frob}[1]{\oldfrob_#1}\DeclareMathOperator{\id}{id}
\hypersetup{pdftitle=Algèbre 2 -- Corps finis}
\title{Algèbre 2\\\emph{Corps finis}}
\author{}
\date{}
\begin{document}
\begin{frame}
    \titlepage
\end{frame}
\slideq{Propriétés de $\frob p\!:\!\kk\to\kk$\linebreak$\left|\kk\right|=p^n$}{1/4}
\slider{$\frob p$ est un morphisme de corps $\ff_p$ linéaire\linebreak$\l\frob p\r^n=\id_\kk$}{1/4}
\slideq{CNS pour $\ff_q\subset\ff_{q'}$}{2/4}
\slider{$n'=dn$ avec $q=p^n$ et $q'=p^{n'}$}{2/4}
\slideq{$\left|\kk\right|$ pour $\kk$ un corps fini}{3/4}
\slider{$\car\kk^n$ pour $n\in\mathbb N^*$\linebreak$\kk$ est un $\ff_p$-ev de dimension $n$ finie}{3/4}
\slideq{Propriété des corps de cardinal $p^n$}{4/4}
\slider{Il existe un unique sous-corps de $\overline{\ff_p}$ de cardinal $p^n$ à isomorphisme près, c'est le corps de décomposition de $X^{p^n}-X$\linebreak On le note $\ff_q$, $q=p^n$}{4/4}
\end{document}