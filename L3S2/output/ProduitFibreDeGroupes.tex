\documentclass[14pt,usepdftitle=false,aspectratio=169]{beamer}
\usepackage{preambule}
\setbeamercolor{structure}{fg=black}
\usepackage{tikz-cd}\tikzcdset{every arrow/.append style={line cap=round}}\newcommand\kk{\mathbb K}\renewcommand\ll{\mathbb L}\newcommand\ee{\mathbb E}\newcommand\edeg[2]{\left[#1{:}#2\right]}\DeclareMathOperator{\oldgal}{Gal}\newcommand{\gal}[2]{\oldgal\l#1/#2\r}
\hypersetup{pdftitle=Algèbre 2 -- Produit fibré de groupes}
\title{Algèbre 2\\\emph{Produit fibré de groupes}}
\author{}
\date{}
\begin{document}
\begin{frame}
    \titlepage
\end{frame}
\slideq{$\left|G_1\times_HG_2\right|$}{1/4}
\slider{Si $\pi_1$ et $\pi_2$ sont surjectives, $\frac{\left|G_1\right|\cdot\left|G_2\right|}{\left|H\right|}$}{1/4}
\slideq{$G_1\times_HG_2$\linebreak$\pi_i\!:\!G_i\to H$}{2/4}
\slider{$\left\{\l g_1,g_2\r,\pi_1\l g_1\r=\pi_2\l g_2\r\right\}$}{2/4}
\slideq{Propriétés de $\ll\cdot\ee/\ee$\linebreak$\ll/\kk$ galoisienne}{3/4}
\slider{$\ll\cdot\ee/\ee$ est galoisienne et $\gal{\ll\cdot\ee}\ee\cong\gal\ll{\ll\cap\ee}$\linebreak Si $\ll/\kk$ est finie alors $\edeg{\ll\cdot\ee}\ll=\edeg\ll{\ll\cap\ee}$\linebreak Si $\ee/\kk$ est aussi galoisienne alors $\gal{\ll\cdot\ee}\kk\cong\gal\ll\kk\times_{\gal{\ll\cap\ee}\kk}\gal\ee\kk$}{3/4}
\slideq{Propriété universelle du produit fibré}{4/4}
\slider{\begin{tikzcd}[ampersand replacement=\&]G\arrow[drr, bend left, "q_1"]\arrow[ddr, bend right, "q_2"]\arrow[dr, dashed, "q"] \& \& \\\& G_1 \times_H G_2 \arrow[r, "p_1"] \arrow[d, "p_2"] \& G_1 \arrow[d, "\pi_1"] \\\& G_2 \arrow[r, "\pi_2"]\& H\end{tikzcd}}{4/4}
\end{document}