\documentclass[14pt,usepdftitle=false,aspectratio=169]{beamer}
\usepackage{preambule}
\setbeamercolor{structure}{fg=black}
\usepackage{analyse,al,structures}\usepackage{footnotes}\let\phi\varphi
\hypersetup{pdftitle=Calcul différentiel -- Sous-variétés}
\title{Calcul différentiel\\\emph{Sous-variétés}}
\author{}
\date{}
\begin{document}
\begin{frame}
    \titlepage
\end{frame}
\slideq{Définition par redressement}{1/12}
\slider{Une partie non vide $M$ de $\mathbb R^n$ est une sous-variété de classe $\mathcal C^k$ de dimension $p$ si pour tout $x\in M$, il existe un voisinage ouvert $U$ de $x$ dans $\mathbb R^n$, un voisinage ouvert $V$ de $0$ dans $\mathbb R^n$ et un $\mathcal C^k$-difféomorphisme $f\!:\!U\to V$ tels que $f\l M\cap U\r=V\cap\l\mathbb R^d\times\left\{0\right\}\r$}{1/12}
\slideq{Carte pour une variété topologique}{2/12}
\slider{$\l U,\phi\r$ avec $U$ un ouvert de $X$ et $\phi\!:\!U\to\phi\l U\r\subset\mathbb R^n$ un homéomorphisme}{2/12}
\slideq{Espace tangent pour une sous-variété définie par un graphe}{3/12}
\slider{$T_xM=\left\{\l h,\dd\phi_x\l h\r\r,h\in\mathbb R^d\right\}$\linebreak Pour $M=\left\{\l x,\phi\l x\r\r,x\in U\right\}$, $U$ un ouvert de $\mathbb R^d$ et $\phi\!:\!U\to\mathbb R^{n-d}$}{3/12}
\slideq{Espace tangent pour une sous-variété définie par paramétrisation}{4/12}
\slider{$T_xM=\im{\dd h_0}$}{4/12}
\slideq{Définition par les graphes}{5/12}
\slider{Une partie non vide $M$ de $\mathbb R^n$ est une sous-variété de classe $\mathcal C^k$ de dimension $p$ si pour tout $x\in M$, il existe un voisinage ouvert $U$ de $x$ dans $\mathbb R^n$ tel que $M\cap U$ soit le graphe d'une application $f$ de classe $\mathcal C^k$ d'un ouvert de $\mathbb R^d\cong\mathbb R^d\times\left\{0\right\}$ dans $\mathbb R^{n-d}\cong\left\{0\right\}\times\mathbb R^{n-d}$}{5/12}
\slideq{Définition par submersion}{6/12}
\slider{Une partie non vide $M$ de $\mathbb R^n$ est une sous-variété de classe $\mathcal C^k$ de dimension $p$ si pour tout $x\in M$, il existe un voisinage ouvert $U$ de $x$ dans $\mathbb R^n$ et une submersion\footnote{$\dd g_x$ est surjective pour tout $x$} $g\!:\!U\to\mathbb R^{n-p}$ de classe $\mathcal C^k$ tels que $M\cap U=g^{-1}\l0_{\mathbb R^{n-p}}\r$\linebreak Il suffit d'avoir la surjectivité sur $M$ car elle se conserve localement}{6/12}
\slideq{$T_xM$}{7/12}
\slider{$\left\{v\in\mathbb R^n,\exists\gamma\!:\!\left]-\varepsilon,\varepsilon\right[{\to}M,\gamma\l0\r{=}x\wedge\gamma'\l0\r{=}v\right\}$\linebreak C'est un espace vectoriel de $\mathbb R^n$ de dimension $\dim M$}{7/12}
\slideq{Fibré tangent}{8/12}
\slider{$\left\{\l x,v\r,x\in M,v\in T_xM\right\}$}{8/12}
\slideq{$X$ est une variété topologique}{9/12}
\slider{$X$ est un espace séparé tel que pour tout $x\in X$, il existe un voisinage ouvert de $x$ homéomorphe à un ouvert de $\mathbb R^n$}{9/12}
\slideq{Atlas pour une variété topologique}{10/12}
\slider{Famille $\l\l U_i,\phi_i\r\r_{i\in I}$ tel que $X=\bigcup{i\in I}{}{U_i}$}{10/12}
\slideq{Définition par paramétrisation}{11/12}
\slider{Une partie non vide $M$ de $\mathbb R^n$ est une sous-variété de classe $\mathcal C^k$ de dimension $p$ si pour tout $h\in M$, il existe un voisinage ouvert $U$ de $x$ dans $\mathbb R^n$, un voisinage ouvert $\varOmega$ de $0$ dans $\mathbb R^p$ et une appication $h\!:\!\varOmega\to\mathbb R^n$ qui soit une immersion\footnote{$\dd h_x$ est injective} et un homéomorphisme de classe $\mathcal C^k$ sur $M\cap U$}{11/12}
\slideq{Espace tangent pour une sous-variété définie par submersion}{12/12}
\slider{$T_xM=\bigcap{i=1}{n-d}{\ker{\dd\l g_i\r_x}}$}{12/12}
\end{document}