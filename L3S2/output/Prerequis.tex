\documentclass[14pt,usepdftitle=false,aspectratio=169]{beamer}
\usepackage{preambule}
\setbeamercolor{structure}{fg=black}
\usepackage{bigoperators}
\hypersetup{pdftitle=Groupes localement compacts -- Prérequis}
\title{Groupes localement compacts\\\emph{Prérequis}}
\author{}
\date{}
\begin{document}
\begin{frame}
    \titlepage
\end{frame}
\slideq{Espace topologique séparé}{1/5}
\slider{$X$ est un espace toplogique séparé si pour tout $x\neq y$, il existe $V_x$ et $V_y$ des voisinages ouverts de $x$ et $y$ tels que $V_x\cap V_y=\varnothing$\linebreak De manière équivalente, $\left\{\l x,x\r,x\in X\right\}$ est fermé dans $X^2$}{1/5}
\slideq{Espace connexe}{2/5}
\slider{$X$ est connexe si pour tout couple $\l U,V\r$ d'ouverts tels que $X=U\sqcup V$, $U=\varnothing$ ou $V=\varnothing$\linebreak De manière équivalente, les seuls ouverts-fermés de $X$ sont $X$ et $\varnothing$}{2/5}
\slideq{Espace totalement discontinu}{3/5}
\slider{$X$ est totalement discontinu si les composantes connexes de $X$ sont les singletons}{3/5}
\slideq{Espace localement compact}{4/5}
\slider{$X$ est localement compact s'il est séparé et pour tout $x\in X$, il existe un compact $K$ tel que $x\in K$ et il existe $U\subset K$ ouvert tel que $x\in U$}{4/5}
\slideq{Espace compact séparé}{5/5}
\slider{$X$ est compact séparé si pour toute famille d'ouvers $\l U_i\r_{i\in I}$ telle que $X\subset\bigcup{i\in I}{}{U_i}$, il existe $\widetilde I\subset I$ fini tel que $X\subset\bigcup{i\in\widetilde I}{}{U_i}$}{5/5}
\end{document}