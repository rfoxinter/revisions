\documentclass[14pt,usepdftitle=false,aspectratio=169]{beamer}
\usepackage{preambule}
\setbeamercolor{structure}{fg=black}
\newif\ifhtmlcard\makeatletter\@ifclassloaded{beamer}{\htmlcardfalse}{\htmlcardtrue}\makeatother
\hypersetup{pdftitle=Calcul différentiel -- Théorèmes du TD}
\title{Calcul différentiel\\\emph{Théorèmes du TD}}
\author{}
\date{}
\begin{document}
\begin{frame}
    \titlepage
\end{frame}
\slideq{Théorème des fonctions implicites}{1/3}
\slider{\ifhtmlcard\else\large\fi{}Si $X$, $Y$ et $Z$ sont trois espaces de Banach sur $\mathbb R$ et $F\!:\!U\to Z$ est de classe $\mathcal C^1$ avec $U$ un ouvert de $X\times Y$ tels que $\l x_0,y_0\r\in U$ et $F\l x_0,y_0\r=0$ et ${\operatorname{D}}_yf_{\l x_0,y_0\r}\!:\!Y\to Z$ est un isomorphisme de Banach alors il existe un voisinage ouvert $V$ de $x_0$ et $W$ de $y_0$ ainsi qu'une application $\phi\!:\!V\to W$ de classe $\mathcal C^1$ telle que $\phi\l x_0\r=y_0$ et pour tout $\l x,y\r\in V\times W$, $F\l x,y\r=0$ si et seulement si $y=\phi\l x\r$}{1/3}
\slideq{Lemme de sortie de tout compact}{2/3}
\slider{\ifhtmlcard\else\large\fi{}Soit $x\!:\!\left]T_-,T_+\right[$ la solution maximale au problème de Cauchy $x'=f\l t,x\r$ pour $\l t,x\r\in I\times U$, $I\subset\mathbb R$ et $U\subset E$ deux ouverts où $E$ est un Banach et $f$ est localement lipschitzienne par rapport à $x$\linebreak Si $T_+\leqslant\sup\l I\r$ alors $t\mapsto x\l t\r$ sort de tout compact de $U$ au voisinage de $T_+$\linebreak Si de plus $U=E$ et $E$ est de dimension finie alors $\lim\limits_{t\to T_+}\l\left\|x\l t\r\right\|\r=+\infty$ (idem pour $T_-$)}{2/3}
\slideq{Théorème du rang constant}{3/3}
\slider{Si $U\subset\mathbb R^n$ est ouvert et $f\in\mathcal C^1\l U,\mathbb R^p\r$ est telle que ${\operatorname{d}}f_x$ est de rang constant $r$ pour tout $x\in U$ alors pour tout $a\in U$, il existe un voisinage $V$ de $a$ et un voisinage $W$ de $f\l a\r$ ainsi que des difféomorphismes $v\!:\!V\to v\l V\r\subset U$ et $w\!:\!W\to w\l W\r\subset\mathbb R^p$ tels que $f=w^{-1}Av$ où $A\l x_1,\cdots,x_n\r=\l x_1,\cdots,x_p,0,\cdots,0\r$}{3/3}
\end{document}