\documentclass[14pt,usepdftitle=false,aspectratio=169]{beamer}
\usepackage{preambule}
\setbeamercolor{structure}{fg=black}
\usepackage{analyse}\toggleanalysepar\DeclareMathOperator{\gl}{GL}\DeclareMathOperator{\orth}{O}
\hypersetup{pdftitle=Groupes localement compacts -- Mesure de Haar}
\title{Groupes localement compacts\\\emph{Mesure de Haar}}
\author{}
\date{}
\begin{document}
\begin{frame}
    \titlepage
\end{frame}
\slideq{Mesure de Radon sur un espace $X$ localement compact}{1/12}
\slider{Mesure borélienne $\mu\!:\!\left[0,+\infty\right]$ qui est finie sur tout compact, et intérieurement et extérieurement régulière}{1/12}
\slideq{Mesure de Haar d'un ouvert}{2/12}
\slider{La mesure de Haar d'un ouvert est finie}{2/12}
\slideq{Caractérisation de la compacité par la mesure de Haar}{3/12}
\slider{$G$ est compact si et seulement si $\mu\l G\r<+\infty$}{3/12}
\slideq{Théorème de Riesz}{4/12}
\slider{Pour toute forme linéaire positive $I\!:\!\mathcal C_c\l X\r\to\mathbb R$, il existe une unique mesure de Radon $\mu$ telle que $I=I_\mu$ où $I_\mu\l f\r=\int[x][X][][\mu]{f\l x\r}$}{4/12}
\slideq{Lien entre $\mu\l U\r$ et $\mu\l V\r$ pour $\mu$ une mesure de Haar sur $G$ et $U$ et $V$ deux compacts-ouverts de $G$}{5/12}
\slider{$\frac{\mu\l U\r}{\mu\l V\r}=\frac{\left[U{:}U\cap V\right]}{\left[V{:}U\cap V\right]}$}{5/12}
\slideq{Mesure de Radon bi-invariante}{6/12}
\slider{Mesure de Radon invariante à gauche et à droite}{6/12}
\slideq{Mesure de Haar sur $G$}{7/12}
\slider{Mesure de Radon non nulle et invariante à gauche sur $G$}{7/12}
\slideq{Fonction modulaire}{8/12}
\slider{Morphisme de groupes $\varDelta\!:\!G\to\mathbb R_+^*$ tel que $\mu\l\cdot g^{-1}\r=\varDelta\l g\r\mu\l\cdot\r$ pour $\mu$ une mesure de Haar sur $G$\linebreak C'est un morphisme continu}{8/12}
\slideq{Mesure de Radon invariante à gauche}{9/12}
\slider{Mesure de Radon vérifiant $\mu\l gE\r=\mu\l E\r$}{9/12}
\slideq{Caractérisation de l'invariance d'une mesure de Radon par les intégrales}{10/12}
\slider{$\mu$ est invariante à gauche si et seulement si, pour tout $f\in\mathcal C_c\l G\r$ et tout $g\in G$, $\int[x][G][][\mu]{f\l x\r}=\int[x][G][][\mu]{f\l gx\r}$}{10/12}
\slideq{Sous-groupes compacts de $\gl_n\l\mathbb R\r$}{11/12}
\slider{Tout sous-groupe compact de $\gl_n\l\mathbb R\r$ est conjugué à un sous-groupe de $\orth_n\l\mathbb R\r$}{11/12}
\slideq{Théorème de Haar}{12/12}
\slider{Si $G$ est un groupe localement compact alors il existe une mesure de Haar sur $G$ et deux mesures de Haar sont proportionnelles}{12/12}
\end{document}