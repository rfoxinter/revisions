\documentclass[14pt,usepdftitle=false,aspectratio=169]{beamer}
\usepackage{preambule}
\setbeamercolor{structure}{fg=black}
\newcommand\kk{\mathbb K}\renewcommand\ll{\mathbb L}\newcommand\mm{\mathbb M}\newcommand\edeg[2]{\left[#1{:}#2\right]}\DeclareMathOperator{\ev}{ev}\usepackage{polynomes,structures}\DeclareMathOperator{\oldgal}{Gal}\newcommand{\gal}[2]{\oldgal\l#1/#2\r}
\hypersetup{pdftitle=Algèbre 2 -- Extensions de corps}
\title{Algèbre 2\\\emph{Extensions de corps}}
\author{}
\date{}
\begin{document}
\begin{frame}
    \titlepage
\end{frame}
\slideq{Tranfert d'algébricité}{1/18}
\slider{Si $\alpha$ est algébrique sur $\kk$ et $\beta$ est algébrique sur $\pol K\alpha$ alors $\beta$ est algébrique sur $\kk$}{1/18}
\slideq{Clôture algébrique}{2/18}
\slider{Une clôture algébrique de $\kk$ est une extension algébrique de $\kk$ qui est algébriquement close}{2/18}
\slideq{Compositum $\mathbb E_1\cdot\mathbb E_2$}{3/18}
\slider{Plus petit corps contenant $\mathbb E_1$ et $\mathbb E_2$}{3/18}
\slideq{Corps de rupture}{4/18}
\slider{Si $P\in\pol KX$, un corps de rupture de $P$ est une extension algébrique de $\kk$ sur laquelle $P$ admet une racine\linebreak Si $Q$ est un facteur irréductible de $P$ alors $\pol KX/\l Q\r$ est un corps de rupture de $P$ sur $\kk$}{4/18}
\slideq{$\kk^{\text{alg},\ll}$}{5/18}
\slider{$\left\{l\in\ll,l\text{ est algébrique sur }\kk\right\}$\linebreak C'est une extension intermédiaire entre $\kk$ et $\ll$}{5/18}
\slideq{Propriété de $\edeg{\mathbb E_1\mathbb E_2}\kk$}{6/18}
\slider{Si les deux extensions sont algébriques, $\edeg{\mathbb E_1\mathbb E_2}\kk\leqslant\edeg{\mathbb E_1}\kk\edeg{\mathbb E_2}\kk$\linebreak Si $\edeg{\mathbb E_1}\kk\wedge\edeg{\mathbb E_2}\kk=1$ alors on a égalité}{6/18}
\slideq{$\ll/\kk$ est une extension de corps}{7/18}
\slider{Morphisme d'anneau $\kk\to\ll$\linebreak$\ll$ est une $\kk$-algèbre}{7/18}
\slideq{Corps de décomposition}{8/18}
\slider{Si $P\in\pol KX$, un corps de rupture de $P$ est une extension algébrique de $\kk$ sur laquelle $P$ est scindé}{8/18}
\slideq{Images des racines d'un polynôme par un morphisme $\sigma\!:\!\kk\to\kk^\text{a}$}{9/18}
\slider{Si $P\in\pol KX$ et $\alpha\in\rac P$ alors $\sigma\l\alpha\r\in\rac P$ et $\sigma$ définit une bijection de $\rac P$\linebreak En particulier, si $P$ est irréductible, alors l'action $\gal{\kk^\text{a}}\kk\curvearrowright\rac P$ est transitive}{9/18}
\slideq{Théorème de Steiniz}{10/18}
\slider{Tout corps admet une clôture algébrique\linebreak Cette clôture est unique à isomorphisme près}{10/18}
\slideq{Propriété de $\edeg{\pol K\alpha}\kk$ si $\edeg\ll\kk=n<+\infty$ et $\alpha\in\ll$}{11/18}
\slider{$\edeg{\pol K\alpha}\kk\mid n$}{11/18}
\slideq{Extension finie}{12/18}
\slider{$\edeg\ll\kk<+\infty$}{12/18}
\slideq{$\mathbb K$-algèbre}{13/18}
\slider{Anneau $A$ et morphisme d'anneau $\sigma\!:\!\mathbb K\to A$}{13/18}
\slideq{Corps algébriquement clos}{14/18}
\slider{$\kk$ est algébriquement clos si tout $P\in\pol KX$ admet une racine dans $\kk$}{14/18}
\slideq{Théorème de la base télescopique}{15/18}
\slider{Si $\mm/\ll/\kk$ sont deux extensions finies, $\l\beta_j\r_{j\in J}$ une $\ll$-base de $\mm$ et $\l\alpha_i\r_{i\in I}$ une $\kk$-base de $\ll$ alors $\l\alpha_i\beta_j\r_{\l i,j\r\in I\times J}$ est une $\kk$-base de $\mm$}{15/18}
\slideq{Prolongement d'un morphisme $\sigma\!:\!\mm'\to\ll$}{16/18}
\slider{Si $\ll/\kk$ est algébriquement clos et $\mm/\mm'/\kk$ des extensions algébriques alors $\sigma$ se prolonge en $\widetilde\sigma\!:\!\mm\to\ll$\linebreak Si $\mm$ est une clôture algébrique de $\kk$, c'est un isomorphisme}{16/18}
\slideq{$\pol K\alpha$}{17/18}
\slider{$\pol K\alpha\cong\pol KX/\ker{\ev_\alpha}$\linebreak Si $\ker{\ev_\alpha}=\l0\r$, $\alpha$ est transcendant et $\fr K\alpha\cong\fr KX$, sinon, $\alpha$ est algébrique et $\fr K\alpha=\pol K\alpha$ et $\edeg{\pol K\alpha}\kk<\infty$}{17/18}
\slideq{$\edeg\ll\kk$}{18/18}
\slider{$\dim_\kk\l\ll\r$}{18/18}
\end{document}