\documentclass[14pt,usepdftitle=false,aspectratio=169]{beamer}
\usepackage{preambule}
\setbeamercolor{structure}{fg=black}
\newcommand\kk{\mathbb K}\newcommand\ff{\mathbb F}\usepackage{polynomes}\DeclareMathOperator{\dec}{D}\DeclareMathOperator{\oldgal}{Gal}\newcommand{\gal}[2]{\oldgal\l#1/#2\r}
\hypersetup{pdftitle=Algèbre 2 -- Extensions d’Artin-Schreier}
\title{Algèbre 2\\\emph{Extensions d'Artin-Schreier}}
\author{}
\date{}
\begin{document}
\begin{frame}
    \titlepage
\end{frame}
\slideq{Théorème d'Artin-Schreier pour une extension d'ordre $p$\linebreak$\car\kk=p$}{1/2}
\slider{Toute extension galoisienne d'ordre $p$ est de cette forme $\dec_\kk\l X^p-X-a\r$, $a\in\kk$}{1/2}
\slideq{Théorème d'Artin-Schreier pour $X^p-X-a$\linebreak$\car\kk=p$}{2/2}
\slider{$X^p-X-a$ est soit irréductible soit scindé sur $\kk$ et si $\alpha$ est une racines de ce polynôme alors les autres sont données par $\alpha+k$, $k\in\ff_p$\linebreak $\fr K\alpha/\kk$ est abélienne et si $\alpha\notin\kk$, $\gal{\fr K\alpha}\kk\cong\mathbb Z/p\mathbb Z$}{2/2}
\end{document}