\documentclass[14pt,usepdftitle=false,aspectratio=169]{beamer}
\usepackage{preambule}
\setbeamercolor{structure}{fg=black}
\usepackage{probas}\usepackage[nopar]{bigoperators,analyse}\usepackage{usuelles}
\hypersetup{pdftitle=Probabilités avancées -- Chaînes de Markov}
\title{Probabilités avancées\\\emph{Chaînes de Markov}}
\author{}
\date{}
\begin{document}
\begin{frame}
    \titlepage
\end{frame}
\slideq{Cas particuliers du théorème ergodique pour les chaînes irréductibles nulles}{1/31}
\slider{Soit $\l X_n\r$ une chaîne de Markov irréductible récurrente nulle alors $\frac1n\sum{i=0}{n-1}{f\l X_i\r}\xrightarrow[n\to+\infty]{}0$ $\bbP_{\!\!x}$-ps}{1/31}
\slideq{Définition et support de la mesure $\mu_x$}{2/31}
\slider{$\mu_x\l y\r=\esp[x]{\sum{k=0}{T-1}{\mathbb1_{X_k=y}}}$\linebreak En particulier, $\mu_x\l x\r=1$\linebreak$\operatorname{supp}\l\mu_x\r=\set{y\in E,x\rightsquigarrow y}$}{2/31}
\slideq{Période de $x\in E$}{3/31}
\slider{$d_x=\operatorname{pgcd}\l I_x\r$ où $I_x=\set{n\in\bbN,P^n\l x,x\r>0}$\linebreak Si $x$ est récurrent, $I_x-I_x=d_x\bbZ$ et il existe $N\in\bbN$ tel que $\left\llbracket N,+\infty\right\llbracket\subset I_x$}{3/31}
\slideq{Théorème ergodique}{4/31}
\slider{Soit $\l X_n\r$ une chaîne de Markov irréductible récurrente et $\mu$ une mesure invariante, soient $f,g\!:\!E\to\bbR_+$ deux fonctions mesurables telles que $\int[\mu]g>0$ et $\int[\mu]f$ ou $\int[\mu]g$ soit finie, alors pour tout $x\in E$, {\toggleanalysedisplay\togglebigopdisplay\togglebigoplimits$\frac{\sum{i=0}{n-1}{f\l X_i\r}}{\sum{i=0}{n-1}{g\l X_i\r}}\xrightarrow[n\to+\infty]{}\frac{\int[\mu]f}{\int[\mu]g}$} $\bbP_{\!\!x}$-ps}{4/31}
\slideq{Structure des mesures invariantes sur les chaînes irréductible récurrentes}{5/31}
\slider{Les mesures invariantes sont proportionnelles\linebreak De plus, soit toutes les mesures invariantes ont une masse totale infinie et pour tout $x\in E$, $\esp[x]{T_x}=+\infty$\linebreak Soit toutes les mesures invariantes sont de masse totale finie et il existe alors une unique loi invariante $\pi$ telle que $\pi\l x\r=\frac1{\esp[x]{T_x}}>0$}{5/31}
\slideq{Mesure $\mathbb P_\mu$ pour un noyau de transition $P$ et une mesure de probabilités $\mu$}{6/31}
\slider{$\mathbb P_\mu=\sum{x\in E}{}{\mu\l\set x\r\mathbb P_x}$}{6/31}
\slideq{Propriétés des mesures invariantes sur les chaînes irréductibles transientes}{7/31}
\slider{Toutes les mesures invariantes ont une masse totale infinie}{7/31}
\slideq{$\mu$ est une mesure invariante pour $P$}{8/31}
\slider{$\mu\neq0$, pour tout $x\in E$, $\mu\l x\r<+\infty$ et pour tout $y\in E$, $\mu\l y\r=\sum{x\in E}{}{\mu\l x\r P\l x,y\r}$\linebreak On a $\mu P=\mu$, donc $\mu$ est un vecteur propre de $P^\top$ associé à la veleur propre $1$}{8/31}
\slideq{Propriété de Markov forte}{9/31}
\slider{Si $\l X_n\r$ est une chaîne de Markov, $\mathcal F_n$ est la filtration naturelle et $T$ est un temps d'arrêt alors\linebreak$\forall A\in\mathcal F_T$, $\forall F\!:\!E^{\mathbb N}\to\mathbb R_+$ mesurable, $\esp[x]{F\l\l X_{n+T}\r_{n\geqslant0}\r\mathbb1_A\sq T<+\infty,X_T=y}=$\linebreak$\p[\!\!x]{A\sq T<+\infty,X_T=y}\esp[y]{F}$\linebreak Pour toute $F\!:\!E^{\mathbb N}\to\mathbb R_+$ mesurable, $\esp[x]{F\l\l X_{n+T}\r_{n\in\mathbb N}\r\sq\mathcal F_T}=\esp[X_T]{F}\mathbb1_{T<+\infty}$}{9/31}
\slideq{Noyau de transition}{10/31}
\slider{$P\!:\!E\times E\to\lc0,1\rc$ telle que, pour tout $x\in E$, $\sum{y\in E}{}{P\l x,y\r}=1$\linebreak Le noyau de transition associé à une chaîne de Markov est $P\l x,y\r=\p{X_1=y\sq X_0=x}$}{10/31}
\slideq{Propriétés de la fonction de Green}{11/31}
\slider{$G\l x,y\r=\sum{n=0}{+\infty}{P^n\l x,y\r}$\linebreak Si $x$ est récurrent, $G\l x,x\r=+\infty$\linebreak Si $x$ est transient, $G\l x,x\r<+\infty$\linebreak Si $x\neq y$ alors $G\l x,y\r=\rho\l x,y\r G\l y,y\r$, en particulier, $G\l x,y\r\leqslant G\l y,y\r$\linebreak$G\l x,x\r=1+\rho\l x,x\r G\l x,x\r$}{11/31}
\slideq{Chaîne de Markov irréductible récurrente positive}{12/31}
\slider{Toutes les mesures invariantes ont une masse totale finie}{12/31}
\slideq{$x\in E$ est récurrent/transient}{13/31}
\slider{$x$ est récurrent si $\p[\!\!x]{T_x<+\infty}=1$, et alors $N_x=+\infty$\linebreak$x$ est transient si $\p[\!\!x]{T_x<+\infty}<1$, et alors $\esp[x]{N_x}=\frac{1}{\p[\!\!x]{T_x=+\infty}}<+\infty$}{13/31}
\slideq{Relation $x\rightsquigarrow y$ sur les chaînes de Markov}{14/31}
\slider{$x\rightsquigarrow y$ si et seulement si $G\l x,y\r>0$\linebreak C'est une relation réflexive et transitive\linebreak Si $x$ est récurrent et $x\rightsquigarrow y$ alors $y\rightsquigarrow x$ et $\rho\l x,y\r=\rho\l y,x\r=1$}{14/31}
\slideq{Mesure des éléments transients d'une chaîne de Markov non irréductible par une mesure invariante de masse totale finie}{15/31}
\slider{$\mu\l T\r=0$}{15/31}
\slideq{Fonction de Green}{16/31}
\slider{$G\!:\!E\times E\to\lc0,+\infty\rc$ telle que $G\l x,y\r=\esp[x]{N_y}$\linebreak$\rho\!:\!E\times E\to\lc0,1\rc$ telle que $\rho\l x,y\r=\p[\!\!x]{T_y<+\infty}$}{16/31}
\slideq{Chaîne de Markov homogène}{17/31}
\slider{Processus aléatoire $\l X_n\r$ à valeurs dans $E$ au plus dénombrable tel que $\p{X_{n+1}=x_{n+1}\sq X_0,\cdots,X_n}=\p{X_{n+1}=x_{n+1}\sq X_n}$ et $\p{X_{n+1}=y\sq X_n=x}=\p{X_1=y\sq X_0=x}$\linebreak$\l X_n\r$ est une chaîne de Markov (homogène) s'il existe un noyau de probabilités $P$ tel que $\p{X_{n+1}=y\sq X_0,\cdots,X_n}=P\l X_n,y\r$}{17/31}
\slideq{Mesure $\mathbb P_x$ pour un noyau de transition $P$}{18/31}
\slider{Il existe une unique mesure $\mathbb P_x$ sur $E^{\mathbb N}$ telle que si $X_i\!:\!E^{\mathbb N}\to E$ est la projection sur le coefficient $i$ alors $\p[x]{X_0=x}=1$ et telle que $\l X_n\r$ soit une chaîne de Markov de transition $P$}{18/31}
\slideq{Période d'une chaîne irréductible récurrente}{19/31}
\slider{Si $\l X_n\r$ est irréductible récurrente, $d_x=d_y$ pour tout $x,y\in E$\linebreak La période de $\l X_n\r$ est la période $d$ commune\linebreak$\l X_n\r$ est apériodique si $d=1$}{19/31}
\slideq{Définition et lien entre $T_x$ temps d'arrêt associé à $x$ et $N_x$ le nombre de passages en $x$}{20/31}
\slider{$N_x=\sum{n=0}{+\infty}{\mathbb1_{X_n=x}}$\linebreak$T_x=\inf{\set{n\in\mathbb N,X_n=x}}$\linebreak$\set{T_x<+\infty}=\set{N_x\geqslant\mathbb1_{X_0=x}+1}$}{20/31}
\slideq{Convergence de $P^n$ pour une chaîne irréductible, récurrente positive et apériodique}{21/31}
\slider{$\sum{y\in E}{}{\left|P^n\l x,y\r-\pi\l y\r\right|}\xrightarrow[n\to+\infty]{}0$ pour tout $x\in E$ et en particulier, $X_n\xrightarrow[n\to+\infty]{\calL}\pi$}{21/31}
\slideq{Chaîne de Markov irréductible récurrente nulle}{22/31}
\slider{Toutes les mesures invariantes ont une masse totale infinie}{22/31}
\slideq{Propriétés de l'espace des mesures invariantes}{23/31}
\slider{C'est un espace convexe\linebreak Si $\mu$ et $\nu$ sont deux mesures invariantes pour $P$ et $\alpha_1,\alpha_2\geqslant0$ sont tels que $\alpha_1+\alpha_2>0$ alors $\alpha_1\mu+\alpha_2\nu$ est une mesure invariante pour $P$}{23/31}
\slideq{La chaîne de Markov $\l X_n\r$ est irréductible}{24/31}
\slider{Pour tout $\l x,y\r\in E$, $x\rightsquigarrow y$ et $y\rightsquigarrow x$\linebreak En particulier, soit tout état est récurrent, soit tout état est transient\linebreak Si $E$ est fini, tout état est récurrent}{24/31}
\slideq{Relation d'équivalence $x\sim y$ sur les chaînes de Markov}{25/31}
\slider{$x\sim y$ si et seulement si $x\rightsquigarrow y$ et $y\rightsquigarrow x$}{25/31}
\slideq{$F\subset E$ est close}{26/31}
\slider{Pour tout $x\in F$ et tout $y\notin F$, $\p[\!\!x]{T_y=\infty}=1$\linebreak En particulier, les classes de récurrence sont closes}{26/31}
\slideq{Propriété de Markov faible}{27/31}
\slider{Si $\l X_n\r$ est une chaîne de Markov alors\linebreak$\forall m,n\geqslant1$, $\forall A\subset E^{n+1}$, $\forall B\subset E^{n}$, $\p[\!\!x]{\l X_0{,}{\cdot}{\cdot}{\cdot}{,}X_n\r{\in}A{,}\l X_n{,}{\cdot}{\cdot}{\cdot}{,}X_{n+m}\r{\in}B\!\sq\!X_n{=}y}{=}$\linebreak$\p[\!\!x]{\l X_0{,}{\cdot}{\cdot}{\cdot}{,}X_n\r{\in}A\!\sq\!X_n{=}y}\p[\!\!y]{\l X_n{,}{\cdot}{\cdot}{\cdot}{,}X_{n+m}\r{\in}B}$\linebreak Pour toute $F\!:\!E^{\mathbb N}\to\mathbb R_+$ mesurable, $\esp[x]{F\l\l X_{n+m}\r_{n\in\mathbb N}\r\sq\mathcal F_m}=\esp[X_m]{F}$}{27/31}
\slideq{Théorème sur la structure des états des chaînes de Markov $\l X_n\r$ sur $E$}{28/31}
\slider{{\togglebigopdisplay On a une partition en classes d'équivalence $E=\bigsqcup{i\in I}{}{R_i}\sqcup\bigsqcup{j\in J}{}{T_j}$, $R=\bigsqcup{i\in I}{}{R_i}$ et $T=\bigsqcup{j\in J}{}{T_j}$ les ensembles des états récurrents et transients\linebreak Si $x\in R_i$ alors $\mathbb P_x$-ps, si $y\in R_i$ alors $N_y=+\infty$ et si $z\notin R_i$, $N_z=0$\linebreak Si $x\in T_j$ et $T=\inf{\set{n\geqslant0,X_n\in R}}$, alors $\mathbb P_{\!\!x}\l\l T=\infty\wedge\l\forall y\in E, N_y<+\infty\r\r\vee{}\right.$\linebreak$\left.\l T<\infty\wedge\l\exists i\in I,\forall n\leqslant T,X_n\in R_i\r\r\r=1$}}{28/31}
\slideq{$\mu$ est une mesure réversible pour $P$}{29/31}
\slider{$\mu\neq0$, pour tout $x\in E$, $\mu\l x\r<+\infty$ et pour tout $y\in E$, $\mu\l x\r P\l x,y\r=\mu\l y\r P\l y,x\r$}{29/31}
\slideq{Cas particuliers du théorème ergodique pour les chaînes irréductibles positives}{30/31}
\slider{Soit $\l X_n\r$ une chaîne de Markov irréductible récurrente positive alors $\frac1n\sum{i=0}{n-1}{f\l X_i\r}\xrightarrow[n\to+\infty]{}\int[\pi]f$ $\bbP_{\!\!x}$-ps pour $\pi$ la loi invariante\linebreak En particulier, $\pi\l y\r=\lim[n\to+\infty]{\frac1n\sum{i=0}{n-1}{\mathbb1_{X_i=y}}}$}{30/31}
\slideq{Propriétés de $\mu_x$}{31/31}
\slider{$\mu_x\geqslant\mu_xP$\linebreak Si $x$ est récurrent alors $\mu_x$ est une mesure invariante et $\operatorname{supp}\l\mu_x\r=\set{y\in E,x\sim y}$}{31/31}
\end{document}