\documentclass[14pt,usepdftitle=false,aspectratio=169]{beamer}
\usepackage{preambule}
\setbeamercolor{structure}{fg=black}
\DeclareMathOperator{\oldtg}{T}\def\tg{{\oldtg}}\let\upGamma\Gamma\def\cv#1{{\operatorname{\upGamma}}\l#1,\tg#1\r}\DeclareMathOperator{\oldder}{Der}\def\Der#1{\oldder\l#1\r}\usepackage[nopar]{analyse,bigoperators}\togglebigoppar\usepackage{arrows}
\hypersetup{pdftitle=Géométrie avancée -- Champs de vecteurs}
\title{Géométrie avancée\\\emph{Champs de vecteurs}}
\author{}
\date{}
\begin{document}
\begin{frame}
    \titlepage
\end{frame}
\slideq{$\lc\delta_1,\delta_2\rc$, $\delta_1,\delta_2\in\Der M$}{1/34}
\slider{$\delta_1\delta_2-\delta_2\delta_1\in\Der M$}{1/34}
\slideq{$\phi\!:\!M\to N$ difféomorphisme\linebreak$\phi^*$}{2/34}
\slider{$\phi^*X=\l\phi_*\r^{-1}$}{2/34}
\slideq{Restriction d'une dérivation}{3/34}
\slider{Si $U\subset V\subset M$ sont ouverts alors on dispose d'une application canonique de restriction $\rho\!:\!\Der V\to\Der U$ définie par $\rho\l\delta\r\l f\r=\delta_{|U}\l f\r$}{3/34}
\slideq{$\varOmega\subset I\times U$ ouvert tel que $\set0\times U\subset\varOmega$\linebreak$h\!:\!\varOmega\to U$ est un groupe à $1$-paramètre}{4/34}
\slider{$h\l0,\cdot\r=\id_U$\linebreak Dès que ça a un sens, $h\l t,h\l t',x\r\r=h\l t+t',x\r$}{4/34}
\slideq{$\phi\!:\!M\to N$ difféomorphisme\linebreak$\phi_*\lc X,Y\rc$}{5/34}
\slider{$\lc\phi_*X,\phi_*Y\rc$}{5/34}
\slideq{$X\in\cv M$ appartient à $D$}{6/34}
\slider{Pour tout $x\in M$, $X\l x\r\in D$}{6/34}
\slideq{$X\in\Der M$, $\phi\!:\!M\to N$ difféomorphisme\linebreak$\phi_*\delta$}{7/34}
\slider{$\appl{\phi_*\delta}{\mathcal C^\infty N}{\mathcal C^\infty N}{f}{\delta\l f\circ\phi\r\circ\phi^{-1}}$}{7/34}
\slideq{$D$ est involutive}{8/34}
\slider{$D$ est stable par crochet\linebreak$X,Y\in D\Rightarrow\lc X,Y\rc\in D$}{8/34}
\slideq{Théorème de redressement simultané de champs de vecteurs}{9/34}
\slider{Si $M$ est une variété différentielle et $X_1,\cdots,X_p\in\cv M$ sont tels que $\l X_1\l x_0\r,\cdots,X_p\l x_0\r\r$ est libre et pour tout $\l i,j\r\in\llb1,p\rrb^2$, $\lc X_i,X_j\rc=0$, alors in existe une carte $\l U,\psi\r$ autour de $x_0$ telle que ${X_i}_{|U}=\partial_i$ (ie, $\psi_*X_i=\pder[][x_i]{}{}$)}{9/34}
\slideq{Théorème de redressement de champs de vecteurs}{10/34}
\slider{Si $M$ est une variété différentielle et $X\in\cv M$ est tel que $X\l x_0\r\neq0$ alors in existe une carte $\l U,\psi\r$ autour de $x_0$ telle que $X_{|U}=\partial_1$ (ie, $\psi_*X=\pder[][x_1]{}{}$)}{10/34}
\slideq{Flot de champ de vecteur $X$}{11/34}
\slider{$\appl{\phi^X}{\varOmega}{U}{\l t,x\r}{c_x\l t\r}$\linebreak On note $\phi^X_t=\phi^X\l t,\cdot\r$}{11/34}
\slideq{Identité de Jacobi}{12/34}
\slider{$\lc X,\lc Y,Z\rc\rc+\lc Y,\lc Z,X\rc\rc+\lc Z,\lc X,Y\rc\rc=0$}{12/34}
\slideq{$X\in\cv M$ est complet}{13/34}
\slider{Le flot $\phi^X$ est défini sur $\bbR\times M$}{13/34}
\slideq{$\psi\!:\!U\to V$ difféomorphisme\linebreak$\phi^{\psi_*X}$}{14/34}
\slider{$\psi\circ\phi^X\circ\psi^{-1}$}{14/34}
\slideq{$\lc X,Y\rc$, $X,Y\in\cv M$}{15/34}
\slider{L'unique élément de $\Der M$ tel que $L_{\lc X,Y\rc}=\lc L_X,L_Y\rc$}{15/34}
\slideq{Lien entre $\cv U$ et $\Der U$, $U\subset\mathbb R^n$ ouvert}{16/34}
\slider{L'application $X=\l X_1,\cdots,X_n\r\mapsto L_X=\sum{i=1}n{X_i\pder[][x_i]{}{}}$ est un isomorphisme d'espaces vectoriels sur $\mathbb R$}{16/34}
\slideq{$\cv U$}{17/34}
\slider{Ensemble des champs de vecteurs lisses sur $U$\linebreak Ensemble des sections lisses $s\!:\!U\to\tg U$}{17/34}
\slideq{Propriétés des sous-niveaux fermés $M_a=f^{-1}\l\left]-\infty,a\rc\r$ pour $M$ une variété différentielle et $f\!:\!M\to\bbR$ lisse}{18/34}
\slider{Si $f^{-1}\l\lc a,b\rc\r$ n'a pas de points critiques et est compact alors il existe $\phi\in\operatorname{Diff\mkern-3mu\acute{\mkern2mu e}o}\l M\r$ tel que $\phi\l M_a\r=M_b$}{18/34}
\slideq{Propriétés de $\phi^X$}{19/34}
\slider{$\phi^X_0=\id_U$\linebreak$\phi^X_{t_2}\circ\phi^X_{t_1}=\phi^X_{t_1+t_2}=\phi^X_{t_1}\circ\phi^X_{t_2}$\linebreak$\phi^X\!:\!\varOmega\cap\l\set t\times U\r\to\varOmega\cap\l\set{-t}\times U\r$ est un difféomorphisme}{19/34}
\slideq{Solutions à $c_x'=X_{c_x}$, $X\in\cv U$}{20/34}
\slider{Si $x\in U$ alors il existe un intervalle $I$ ouvert et contenant $0$ et une courbe intégrable $c_x$ tels que $c_x\l 0\r=x$ et $c'=X_c$, un tel $c$ est unique et on peut définir un intervalle maximal de définition $I\l x\r=\left]a\l x\r,b\l x\r\right[$ de $c$\linebreak$\varOmega=\bigcup{x\in U}{}{I\l x\r\times\set x}$ est un ouvert qui contient $\set0\times U$}{20/34}
\slideq{$X\in\cv M$, $\phi\!:\!M\to N$ difféomorphisme\linebreak$\phi_*X$}{21/34}
\slider{$\appl{\phi_*X}{N}{\tg N}{y}{\dd_{\phi^{-1}\l y\r}\phi\l X_{\phi^{-1}\l y\r}\r}\in\cv N$}{21/34}
\slideq{Lien entre $\cv M$ et $\Der M$, $M$ variété différentielle}{22/34}
\slider{L'application $X\mapsto L_X={x\mapsto \dd_xf\l X\l x\r\r}$ est un isomorphisme d'espaces vectoriels sur $\mathbb R$}{22/34}
\slideq{$X\in\cv M$, une courbe lisse $c$ est itégrable}{23/34}
\slider{$c'=X_c=X\circ c$}{23/34}
\slideq{Théorème de Frobenius}{24/34}
\slider{Une distribution involutive est complètement intégrable}{24/34}
\slideq{Propriétés des champs de vecteurs dans le cas où $M$ est compact}{25/34}
\slider{Ils sont complets}{25/34}
\slideq{Lien immédiat entre distribution involutive et distribution complètement intégrable}{26/34}
\slider{Une distribution complètement intégrable est involutive}{26/34}
\slideq{Distribution $D$ de rang $p\in\llb1,n\rrb$}{27/34}
\slider{Sous-fibré vectoriel de rang $p$ de $\tg M$}{27/34}
\slideq{Dérivation sur $U\subset M$ un ouvert dans une vériété différentielle}{28/34}
\slider{Application linéaire $\delta\!:\!\mathcal C^\infty\l U,\mathbb R\r\to\mathcal C^{\infty}\l U,\mathbb R\r$ qui vérifie la règle de Leibniz: $\delta\l fg\r=\delta\l f\r g+f\delta\l g\r$\linebreak On note $\Der U$ les dérivations sur $U$}{28/34}
\slideq{Sous-fibré vectoriel de $E$}{29/34}
\slider{Fibré vectoriel et sous-variété de $E$ qui peut être simultanément trivialisé avec $E$}{29/34}
\slideq{$D$ est complètement intégrable}{30/34}
\slider{Pour tout $x\in M$, il existe un ouvert $U$ contenant $x$ et $x\in Z\subset U$ une sous-variété de dimension $p$ telle que, pour tout $z\in Z$, $\tg_zZ=D_z$ où $D_z$ est la fibre au dessus de $z$ dans $D$}{30/34}
\slideq{Construction de dérivations sur $M$}{31/34}
\slider{Si $\l U_i\r$ est un recouvrement d'ouverts de $M$ et $\delta_i\in\Der{U_i}$ sont des dérivations telles que ${\delta_i}_{|U_i\cap U_j}={\delta_j}_{|U_i\cap U_j}$ alors il existe une unique dérivation $\delta\in\Der M$ telle que $\delta_{|U_i}=\delta_i$}{31/34}
\slideq{Propriétés de l'action $\operatorname{Diff\mkern-3mu\acute{\mkern2mu e}o}\l M\r\acts M$}{32/34}
\slider{L'action est $k$-transitive pour tout $k\in\bbN$}{32/34}
\slideq{Lien entre $\cv M$ et les groupes locaux à un paramètre}{33/34}
\slider{À un champ de vecteurs $X$, on peut lui associer sur les ouverts de l'atlas $\l U_i,\psi_i\r$ le flot $\phi_i^X=\psi_i^{-1}\circ\phi^{\l\psi_i\r_*X}\circ\psi_i$, qui se recollent sur $M$ pour définir un flot $\phi$\linebreak Réciproquement, $x\mapsto\pder[][t]{h}{\l0,x\r}$ est un champ de vecteurs sur $M$}{33/34}
\slideq{$\left.\hskip-\nulldelimiterspace\der[][t]{}{\l\phi_t^Y\r_*X}\right|_{t=0}$}{34/34}
\slider{$\lc X,Y\rc$}{34/34}
\end{document}