\documentclass[14pt,usepdftitle=false,aspectratio=169]{beamer}
\usepackage{preambule}
\setbeamercolor{structure}{fg=black}
\usepackage{probas,usuelles,bigoperators}\togglebigoppar\def\none{\ensuremath{\emptyset}}\def\ssn{sous/sur/\none}
\hypersetup{pdftitle=Probabilités avancées -- Martingales à temps discret}
\title{Probabilités avancées\\\emph{Martingales à temps discret}}
\author{}
\date{}
\begin{document}
\begin{frame}
    \titlepage
\end{frame}
\slideq{Martingale}{1/9}
\slider{$\l X_n\r$ est une martingale par rapport à $\l\mathcal F_n\r$ si $\l X_n\r$ est adaptée à $\l \mathcal F_n\r$ et $\esp{X_{n+1}\sq\mathcal F_n}=X_n$}{1/9}
\slideq{Stabilités des sous/sur/\none-martingales}{2/9}
\slider{Si $\l X_n\r$ et $\l Y_n\r$ sont deux \ssn-martingales alors $\l X_n+Y_n\r$ aussi\linebreak Si $\l X_n\r$ et $\l Y_n\r$ sont des sous-martingales (resp. sur-martingale) alors $\l\max{X_n,Y_n}\r$ (resp. $\l\min{X_n,Y_n}\r$) aussi\linebreak Si $\l X_n\r$ est une martingale et $\phi$ est convexe telle que $\esp{\left|\phi\l X_n\r\right|}<+\infty$ alors $\l\phi\l X_n\r\r$ est une sous-martingale}{2/9}
\slideq{Processus adapté à une filtration $\l\mathcal F_n\r$}{3/9}
\slider{$\l X_n\r$ une suite de variables aléatoires avec $X_n$ qui est $\mathcal F_n$-mesurable}{3/9}
\slideq{Filtration}{4/9}
\slider{$\l\mathcal F_n\r$ une suite croissante de sous-tribus de $\mathcal F$}{4/9}
\slideq{Sous-martingale}{5/9}
\slider{$\l X_n\r$ est une sous-martingale par rapport à $\l\mathcal F_n\r$ si $\l X_n\r$ est adaptée à $\l \mathcal F_n\r$ et $\esp{X_{n+1}\sq\mathcal F_n}\geqslant X_n$}{5/9}
\slideq{Processus prévisible}{6/9}
\slider{$\l H_n\r_{n\in\mathbb N^*}$ est un processus prévisible par rapport à $\l X_n\r_{n\in\mathbb N}$ adapté à $\mathcal F_n$ si $H_n$ est $\mathcal F_{n-1}$-mesurable}{6/9}
\slideq{Sur-martingale}{7/9}
\slider{$\l X_n\r$ est une sur-martingale par rapport à $\l\mathcal F_n\r$ si $\l X_n\r$ est adaptée à $\l \mathcal F_n\r$ et $\esp{X_{n+1}\sq\mathcal F_n}\leqslant X_n$}{7/9}
\slideq{Intégrales stochastiques de \ssn-martingales}{8/9}
\slider{Si $\l X_n\r$ est une martingale et $\l H_n\r$ est un processus prévisible de $L^\infty$ alors $\l\l H\cdot X\r_n\r$ est une martingale\linebreak Si $\l X_n\r$ est une sous/sur-martingale et $\l H_n\r$ est un processus prévisible positif de $L^\infty$ alors $\l\l H\cdot X\r_n\r$ est une sous/sur-martingale\linebreak Si $\l X_n\r$ est dans $L^2$ alors on peut avoir $\l H_n\r$ dans $L^2$}{8/9}
\slideq{Intégrale stochastique (discrète)}{9/9}
\slider{Soit $\l X_n\r$ un processus adapté à $\mathcal F_n$ et $\l H_n\r$ un processus prévisible, l'intégrale stochastique de $\l H_n\r$ par rapport à $\l X_n\r$ est $\l H\cdot X\r_n=\sum{k=1}{n}{H_k\l X_k-X_{k-1}\r}$}{9/9}
\end{document}