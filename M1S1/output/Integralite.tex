\documentclass[14pt,usepdftitle=false,aspectratio=169]{beamer}
\usepackage{preambule}
\setbeamercolor{structure}{fg=black}
\DeclareMathOperator{\oldFrac}{Frac}\def\Frac#1{\oldFrac\l#1\r}\usepackage{polynomes}
\hypersetup{pdftitle=Algèbre avancée -- Intégralité}
\title{Algèbre avancée\\\emph{Intégralité}}
\author{}
\date{}
\begin{document}
\begin{frame}
    \titlepage
\end{frame}
\slideq{Propriété des éléments entiers d'un anneau factoriels}{1/16}
\slider{Les anneaux factoriels sont intégralement clos}{1/16}
\slideq{$B$ est une $A$-algèbre}{2/16}
\slider{Morphisme d'anneaux de $A$ dans $B$}{2/16}
\slideq{$A$ est intégralement clos\linebreak Normalisé de $A$}{3/16}
\slider{$A$ est un anneau intègre et $A$ est intégralement fermé dans $\Frac A$\linebreak Le normalisé de $A$ est la fermeture intégrale de $A$ dans $\Frac A$, c'est un anneau intégralement clos}{3/16}
\slideq{CNS pour avoir $x\in B$ entier sur $A$}{4/16}
\slider{$x$ est fini sur $A$\linebreak Il existe un sous-anneau $B'$ de $B$ contenant $A$ et $x$ tel que $B'$ est fini sur $A$}{4/16}
\slideq{CNS pour $A$ intégralement clos sur les idéaux premiers et maximaux}{5/16}
\slider{$A$ est intégralement clos\linebreak Si et seulement si, pour tout idéal ${\mathfrak p}$ premier, $A_{\mathfrak p}$ est intégralement clos\linebreak Si et seulement si, pour tout idéal maximal ${\mathfrak m}$, $A_{\mathfrak m}$ est intégralement clos}{5/16}
\slideq{$B$ est une extension de $A$}{6/16}
\slider{$A$ est un sous-anneau de $B$}{6/16}
\slideq{$B$ est fini sur $A$}{7/16}
\slider{$B$ est un $A$-module de type fini}{7/16}
\slideq{$x\in B$ est entier sur $A$}{8/16}
\slider{Il existe $P\in A\lc X\rc$ unitaire tel que $P\l x\r$}{8/16}
\slideq{$x\in\mathbb L$ avec $\mathbb L/\Frac A=\mathbb K$ finie est entier sur $A$ intégralement clos}{9/16}
\slider{$\mu_{\mathbb K,x}\in A\lc X\rc$}{9/16}
\slideq{$x\in B$ est fini sur $A$}{10/16}
\slider{$A\lc x\rc\subset B$ est fini sur $A$}{10/16}
\slideq{Fermeture intégrale de $A$ dans $B$\linebreak$A$ est intégralement fermé}{11/16}
\slider{$C=\set{x\in B,x\text{ est entier sur} A}$ est la fermeture intégrale de $A$ dans $B$\linebreak$A$ est intégralement fermé dans $B$ si $A=C$\linebreak En particulier, $C$ est intégralement fermé dans $B$}{11/16}
\slideq{$B$ est entier sur $A$}{12/16}
\slider{Tout $x\in B$ est entier sur $A$}{12/16}
\slideq{Théorème de normalisation de Noether}{13/16}
\slider{Si $A$ est une $\mathbb k$-algèbre de type fini alors il existe $n\in\mathbb N$ et $\l x_1,\cdots,x_n\r\in A^n$ algèbriquement indépendants sur $\mathbb k$ tels que $A$ est finie sur $\pol k{x_1,\cdots,x_n}$\linebreak$n$ est unique, et si $A$ est engendré par $m$ éléments alors $n\leqslant m$}{13/16}
\slideq{Localisation des extensions entières et fermetures intégrales}{14/16}
\slider{Si $B$ est entier sur $A$ alors $S^{-1}B$ est entier sur $S^{-1}A$\linebreak Si $C$ est la fermeture intégrale de $A$ dans $B$ alors $S^{-1}C$ est la fermeture intégrale de $S^{-1}A$ dans $S^{-1}B$}{14/16}
\slideq{Stabilité des extensions entières et finies et des éléments entiers}{15/16}
\slider{Si $B$ est entière/finie sur $A$ et $C$ est entière/finie sur $B$ alors $C$ est entière/finie sur $A$\linebreak Si $x$ et $y$ sont entiers sur $A$ alors $xy$ et $x-y$ sont entiers sur $A$\linebreak En particulier, $\set{x\in B,x\text{ est entier sur} A}$ est un sous-anneau de $B$ contenant $A$}{15/16}
\slideq{$\l a_1,\cdots,a_n\r\in A^n$ sont linéairement indépendants sur $\mathbb k$ avec $A$ une $\mathbb k$-algèbre}{16/16}
\slider{Pour tout $P\in\pol k{X_1,\cdots,X_n}$, si $P\l a_1,\cdots,a_n\r=0$ alors $P=0$}{16/16}
\end{document}