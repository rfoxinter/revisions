\documentclass[14pt,usepdftitle=false,aspectratio=169]{beamer}
\usepackage{preambule}
\setbeamercolor{structure}{fg=black}
\def\mm{\mathfrak m}\def\pp{\mathfrak p}\def\kk{\mathbb k}\usepackage{complexes}\usepackage{tikz-cd}\tikzcdset{every arrow/.append style={line cap=round, outer sep = -2.25pt}, every diagram/.append style={cramped, column sep=scriptsize, row sep=scriptsize}}
\hypersetup{pdftitle=Algèbre avancée -- Localisation}
\title{Algèbre avancée\\\emph{Localisation}}
\author{}
\date{}
\begin{document}
\begin{frame}
    \titlepage
\end{frame}
\slideq{Propriétés des modules projectifs de $A$ un anneau local}{1/11}
\slider{Tout module projectif de type fini est libre\linebreak Ce résultat reste vrai même si $P$ n'est pas de type fini (Kaplansky)}{1/11}
\slideq{$A$ est local}{2/11}
\slider{$A$ est local s'il possède un unique idéal maximal}{2/11}
\slideq{Propriétés de $\l x_1,\cdots,x_n\r$ si $\l\bar{x_1},\cdots,\bar{x_n}\r$ est une base de $M/\mm M$ en tant que $A/\mm$-espace vectoriel pour $\l A,\mm\r$ un anneau local et $M$ un module de type fini}{3/11}
\slider{C'est une famille génératrice de $M$ en tant que $A$-module}{3/11}
\slideq{Corps résiduel}{4/11}
\slider{$\kk=A/\mm$ est le corps résiduel de $\l A,\mm\r$ local}{4/11}
\slideq{Localisé de $A$ en $\pp$}{5/11}
\slider{Si $\pp$ est un idéal premier de $A$, le localisé de $A$ en $\pp$ est $A_\pp=S^{-1}A$ pour $S=A\setminus\pp$}{5/11}
\slideq{CNS pour que $A$ soit local d'idéal maximal $I$}{6/11}
\slider{$A^\times=A\setminus I$}{6/11}
\slideq{Propriétés du localisé de $A$ en $\pp$}{7/11}
\slider{C'est un anneau local d'idéal maximal $\mm=\set{\frac as\in A_\pp,a\in\pp,s\notin\pp}$ et $A_\pp/\mm\cong\operatorname{Frac}\l A\r$}{7/11}
\slideq{$S^{-1}A$}{8/11}
\slider{Anneau $A\times S/{\sim}$ où $\l a,s\r\sim\l a',s'\r$ si et seulement s'il existe $t\in S$ tel que $t\l sa'-s'a\r=0$}{8/11}
\slideq{Lemme de Nakayama}{9/11}
\slider{Si $\l A,\mm\r$ est un anneau local et $M$ est un $A$-module de type fini tel que $\mm M=M$ alors $M=\set0$}{9/11}
\slideq{$S$ est une partie multiplicative de $A$}{10/11}
\slider{$1\in S$ et si $\l x,y\r\in S^2$ alors $xy\in S$}{10/11}
\slideq{PU de l'anneau des fractions\linebreak$A$ un anneau, $S$ une partie multiplicative de $A$}{11/11}
\slider{Soit $\phi\!:\!A\to S^{-1}A$ le morphisme canonique, alors pour tout anneau $B$ et tout morphisme $\psi\!:\!A\to B$ tel que $\psi\l S\r\subset B^\times$, il existe un unique morphisme d'anneaux $\widetilde\psi$ tel que $\psi=\widetilde\psi\circ\phi$\linebreak\begin{tikzcd}[ampersand replacement=\&]A\ar[rr,"\psi"']\ar[dr,"\phi"]\&\&B\\\&S^{-1}A\ar[ur,"\widetilde\psi",dashed]\&\end{tikzcd}}{11/11}
\end{document}