\documentclass[14pt,usepdftitle=false,aspectratio=169]{beamer}
\usepackage{preambule}
\setbeamercolor{structure}{fg=black}
\usepackage{structures,al,matrices,bigoperators}\togglebigoppar
\hypersetup{pdftitle=Algèbre avancée -- Modules sur un anneau}
\title{Algèbre avancée\\\emph{Modules sur un anneau}}
\author{}
\date{}
\begin{document}
\begin{frame}
    \titlepage
\end{frame}
\slideq{Structure isomorphe à $\hom[A]{A^m,A^n}$}{1/11}
\slider{$\mat nmA$ via l'image de la <<~base canonique~>>}{1/11}
\slideq{Premier théorème d'isomorphisme}{2/11}
\slider{$\overline f\!:\!M/\ker f\to\im f$ est un isomorphisme de $A$-modules}{2/11}
\slideq{$A$-module}{3/11}
\slider{Groupe abélien $\l M,+\r$ muni d'une application $A\times M\to M$ telle que\linebreak$a\l m+m'\r=am+am'$\linebreak$\l a+a'\r m=am+a'm$\linebreak$\l aa'\r m=a\l a'm\r$\linebreak$1_Am=m$}{3/11}
\slideq{Sous-module}{4/11}
\slider{Sous-groupe stable par l'action de l'anneau}{4/11}
\slideq{$f\!:\!M\to N$\linebreak$\coker f$}{5/11}
\slider{$N/\im f$}{5/11}
\slideq{Structure de $\hom[A]{M,N}$}{6/11}
\slider{$A$-module en posant $\l f+g\r\l m\r=f\l m\r+g\l m\r$}{6/11}
\slideq{Isomorphisme de $A$-modules}{7/11}
\slider{$f\in\hom[A]{M,N}$ pour laquelle il existe $g\in\hom[A]{N,M}$ telle que $f\circ g=\id_M$ et $g\circ f=\id_N$}{7/11}
\slideq{PU du quotient}{8/11}
\slider{Si $M$ et $P$ sont deux modules, et $N$ est un sous-module de $M$, soit $f\!:\!M\to P$ une application $A$-linéaire telle que $N\subset\ker f$ alors il existe une unique application $A$-linéaire $\overline f$ telle que $f=\overline f\circ\pi$}{8/11}
\slideq{Module $M/N$}{9/11}
\slider{Le groupe quotient d'un $A$-module par un sous-module peut être muni d'une unique structure de $A$-module qui rend $\pi\!:\!M\to M/N$ $A$-linéaire}{9/11}
\slideq{Application linéaire entre $A$-modules}{10/11}
\slider{$f\!:\!M\to N$ telle que\linebreak$f\l am\r=af\l m\r$\linebreak$f\l n+m\r=f\l n\r+f\l m\r$}{10/11}
\slideq{PU de la somme directe de $A$-modules}{11/11}
\slider{Si $\l M_i\r_{i\in I}$ est une famille de $A$-modules et $N$ est un $A$-module et $f_i\!:\!M_i\to N$ est une famille de $A$-modules alors il existe une unique application linéaire $f\!:\!\bigoplus{i\in I}{}{M_i}\to N$ telle que $f_{|M_i}=f_i$}{11/11}
\end{document}