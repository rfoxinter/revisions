\documentclass[14pt,usepdftitle=false,aspectratio=169]{beamer}
\usepackage{preambule}
\setbeamercolor{structure}{fg=black}
\usepackage{analyse,usuelles,topologie,complexes,structures}\usepackage[nopar]{bigoperators}
\hypersetup{pdftitle=Analyse avancée -- Hahn-Banach theorems}
\title{Analyse avancée\\\emph{Hahn-Banach theorems}}
\author{}
\date{}
\begin{document}
\begin{frame}
    \titlepage
\end{frame}
\slideq{Structure of complex-valued linear functionals on a TVS $X$}{1/5}
\slider{A linear functional $\phi\!:\!X\to\mathbb C$ is of the form $\psi\l\cdot\r-\ii\psi\l\ii\cdot\r$ for $\psi\!:\!X\to\mathbb R$ a linear functional\linebreak Conversely, any function of this form defines a linear functional $X\to\mathbb C$}{1/5}
\slideq{Geometric Hahn-Banach theorems}{2/5}
\slider{If $X$ is a TVS and $A,B\subset X$ are convex disjoint, if $A$ is open, there exists a continuous functional $\phi$ on $X$ and $\gamma\in\mathbb R$ such that $\Re{\phi\l u\r}<\gamma\leqslant\Re{\phi\l v\r}$ for all $u\in A$, $v\in B$\linebreak If $X$ is an LCTVS, $A$ is closed and and $B$ compact then there exists $\gamma_1<\gamma_2\in\mathbb R$ such that $\Re{\phi\l u\r}<\gamma_1<\gamma_2<\Re{\phi\l v\r}$ for all $u\in A$, $v\in B$}{2/5}
\slideq{Corollary of Hahn-Banach theorem in $\mathbb K=\mathbb R$ or $\mathbb K=\mathbb C$ on the existence of specific functionals on a TVS $X$}{3/5}
\slider{If $Y$ is a closed subspace and $v\notin Y$ then there exists a functional $\phi\in X^*$ such that $\phi\equiv0$ on $Y$, $\phi\l v\r=d\l v,Y\r>0$ and $\nrm[X^*]\phi=1$\linebreak If $u\in X$, there exists $\phi\in X^*$ such that $\nrm[X^*]\phi=1$ and $\phi\l u\r=\nrm[X]u$\linebreak$J\!:\!u\mapsto\operatorname{ev}_u$ is a isometry on its image}{3/5}
\slideq{Analytic Hahn-Banach theorems for functionals in $\mathbb K=\mathbb R$ or $\mathbb K=\mathbb C$}{4/5}
\slider{If $p\!:\!X\to\mathbb R$ is a semi-norm, if $Y$ is a subspace of $X$ and $\phi\!:\!Y\to\mathbb K$ is a functional such that $\left|\phi\right|\leqslant p$ on $Y$ then $\phi$ can be extended to a linear functional on $X$ with its modulus still dominated by $p$}{4/5}
\slideq{Analytic Hahn-Banach theorem for functionals in $\mathbb R$}{5/5}
\slider{If $p\!:\!X\to\mathbb R$ is a positively homogeneous and sub-additive, if $Y$ is a subspace of $X$ and $\phi\!:\!Y\to\mathbb R$ is a functional such that $\phi\leqslant p$ on $Y$ then $\phi$ can be extended to a linear functional on $X$ still dominated by $p$}{5/5}
\end{document}