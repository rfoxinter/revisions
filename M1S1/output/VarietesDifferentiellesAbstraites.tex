\documentclass[14pt,usepdftitle=false,aspectratio=169]{beamer}
\usepackage{preambule}
\setbeamercolor{structure}{fg=black}
\usepackage[nopar]{bigoperators,analyse}\usepackage{tikz-cd}\tikzcdset{every arrow/.append style={line cap=round}, every diagram/.append style={cramped, column sep=scriptsize, row sep=scriptsize}}\usepackage{arrows,matrices}
\hypersetup{pdftitle=Géométrie avancée -- Variétés différentielles abstraites}
\title{Géométrie avancée\\\emph{Variétés différentielles abstraites}}
\author{}
\date{}
\begin{document}
\begin{frame}
    \titlepage
\end{frame}
\slideq{Variété topologique de dimension $n$}{1/36}
\slider{Espace topologique séparé tel que tout point admette un voisinage ouvert homéomorphe à un ouvert de $\mathbb R^n$}{1/36}
\slideq{Propriétés des fibres lorsque la base est connexe}{2/36}
\slider{Les fibres d'un fibré différentiel sont difféomorphes}{2/36}
\slideq{$f\!:\!M\to N$ continue est lisse avec $M$ et $N$ deux variétés différentielles}{3/36}
\slider{Pour tout $a\in M$, il existe une carte $\l U\ni a,\phi\r$ de $M$ et une carte $\l V\ni f\l a\r,\psi\r$ telles que $\psi\circ f\circ\phi^{-1}\!:\!\phi\l U\cap f^{-1}\l V\r\r\to\psi\l V\r$ est lisse}{3/36}
\slideq{Atlas d'une variété topologique $X$}{4/36}
\slider{Famille de cartes $\l U_i,\phi_i\r_{i\in I}$ telles que $X=\bigcup{i\in I}{}{U_i}$}{4/36}
\slideq{Atlas différentiel d'une variété topologique $X$\linebreak Atlas différentiel maximal}{5/36}
\slider{Un atlas tel que deux cartes sont toujours compatibles\linebreak Il est dit maximal si toute carte compatible avec toutes les autres de l'atlas est dans l'atlas}{5/36}
\slideq{Fibration lisse de base $B$ et d'espace total $E$}{6/36}
\slider{Application $f\!:\!E\to B$ telle que, pour tout $x\in B$, il existe un voisinage ouvert $U$ de $x$, une vériété différentielle $F$ et un homéomorphisme $\phi$ telq que le diagramme suivant commute\linebreak\begin{tikzcd}[ampersand replacement=\&]U\times F\ar[r,"\phi"']\ar[d,"\operatorname{pr}_1"]\& f^{-1}\l U\r\ar[d,"f"']\\U\ar[r,"\operatorname{id}"]\&U\end{tikzcd}}{6/36}
\slideq{Une variété $M$ est muni d'un atlas d'orientation $\l U_i,\phi_i\r$}{7/36}
\slider{Pour tout $i,j$, $\det{\dd_x\l\phi_i\circ\phi_j^{-1}\r}>0$}{7/36}
\slideq{Propriétés de $E/G$ pour $E$ une variété topologique localement compact, $G$ discret et $G\acts E$ cotinue}{8/36}
\slider{$E/G$ est localement compact et $\pi\!:\!E\to E/G$ est ouverte}{8/36}
\slideq{$M\subset X$ est une sous-variété de dimension $p$ d'une variété différentielle $X$}{9/36}
\slider{Pour tout $x\in M$, il existe une carte $\l U\ni x,\phi\r$ telle que $\phi\l U\cap M\r$ est une sous-variété de $\mathbb R^p$}{9/36}
\slideq{Coordonnées centrées en $x$}{10/36}
\slider{$\phi\l x\r=0$}{10/36}
\slideq{$f\!:\!M\to N$ lisse est un difféomorphisme}{11/36}
\slider{$f$ est bijective et $f^{-1}$ est lisse}{11/36}
\slideq{Deux atlas d'orientation $\l U_i,\phi_i\r$ et $\l V_j,\psi_j\r$ sont compatibles}{12/36}
\slider{Pour tout $i,j$, $\det{\dd_x\l\phi_i\circ\psi_j^{-1}\r}>0$}{12/36}
\slideq{$M$ est orientable}{13/36}
\slider{$M$ admet un atlas d'orientation\linebreak Dans le cas où $M$ est connexe et orientable alors elle possède exactement $2$ orientation}{13/36}
\slideq{Variété à bord\linebreak$\mathbb H^n=\set{X\in\mathbb R^n,x_n\ge0}$ muni de la topologie induite}{14/36}
\slider{Variété topologique dénombrable à l'infini muni d'un atlas $\l U_i,\phi_i\!:\!U_i\to\mathbb H^n\r$ à bord\linebreak Les fonctions de transition $\phi_i\circ\phi_j^{-1}\!:\!\phi_j\l U_i\cap U_j\r\to\phi_t\l U_i\cap U_j\r$ sont des difféomorphismes (lisses sur $\mathring{\mathbb H^n}$ et dont la dérivée partielle à gauche {\toggleanalysepar\toggleanalysedisplay$\l x_1,\cdots,x_{n-1}\r\mapsto\pder[][x_n]{f}{x_1,\cdots,x_{n-1},0}$ est lisse})\linebreak Le bord de $M$ est {\togglebigopdisplay$\partial M=\bigcup{}{}{\phi_i^{-1}\l\partial\mathbb H^n\r}$}}{14/36}
\slideq{$X$ est une variété différentielle, $G$ un groupe discret\linebreak$G\acts X$ de manière lisse}{15/36}
\slider{$\forall g,x\mapsto g\cdot x$ est un difféomorphisme}{15/36}
\slideq{Isomorphisme entre deux fibrés différentiels $f_1\!:\!E_1\to B$ et $f_2\!:\!E_2\to B_1$}{16/36}
\slider{La donnée d'un homéomorphisme $\phi$ tel que le diagramme suivant commute\linebreak\begin{tikzcd}[ampersand replacement=\&]E_1\ar[r,"\phi"']\ar[d,"f_1"]\&E_2\ar[d,"f_2"']\\B\ar[r,"\operatorname{id}"]\&B\end{tikzcd}}{16/36}
\slideq{$G$ agit sur $X$ proprement (et de manière continue)}{17/36}
\slider{$X$ est localement compact, $G$ est discret, $G\acts X$ est continue et pour tout $K$ et $L$ compacts de $X$, $\set{g\in G,g\cdot K\cap L\neq\varnothing}$ est fini\linebreak De manière équivalente, $\set{g\in G,g\cdot K\cap K\neq\varnothing}$ est fini}{17/36}
\slideq{Variété différentielle de dimension $n$}{18/36}
\slider{Variété topologique de dimension $n$ et muni d'un atlas différentiel maximal}{18/36}
\slideq{$X$ est paracompact}{19/36}
\slider{Toute couverture de $X$ admet un rafinement localement fini}{19/36}
\slideq{$X$ est un espace topologique dénombrable à l'infini}{20/36}
\slider{Il existe une couverture dénombrable de $X$ par des compacts}{20/36}
\slideq{$G\acts X$ est libre}{21/36}
\slider{Pour tout $g\in G\setminus\set 1$, $\set{x\in X,g\cdot x=x}=\varnothing$}{21/36}
\slideq{Groupe de Lie}{22/36}
\slider{Groupe qui est une veriété différetielle et dans lequel le produit et l'inverse sont lisses}{22/36}
\slideq{Carte d'une variété topologique $X$ de dimension $n$}{23/36}
\slider{$\l U,\phi\r$ où $U\subset X$ est ouvert et $\phi\!:\!U\overset{\sim}{\longrightarrow}V\subset\mathbb R^n$}{23/36}
\slideq{$\l U_1,\phi_1\r$ et $\l U_2,\phi_2\r$ sont compatibles}{24/36}
\slider{$U_1\cap U_2=\varnothing$ ou $\phi_2\circ\phi_1^{-1}\!:\!\phi_1\l U_1\cap U_2\r\to\phi_2\l U_1\cap U_2\r$ est un difféomorphisme}{24/36}
\slideq{Rafinement de $\l U_i\r_{i\in I}$}{25/36}
\slider{$\l V_j\r_{j\in J}$ tel que, pour tout $j\in J$, il existe $i\in I$ tel que $V_j\subset U_i$}{25/36}
\slideq{Partition de l'unité d'une variété différentielle $M$}{26/36}
\slider{Famille de fonctions lisses $\phi_i\!:\!M\to\mathbb R_+$ telles que $\set{\operatorname{supp}\l\phi_i\r}$ est localement fini et $\sum{}{}{\phi_i}\equiv1$\linebreak Une telle famille existe toujours}{26/36}
\slideq{Fibré différentiel de base $B$ et d'espace total $E$}{27/36}
\slider{Fibration losse $f$ de base $B$ et d'espace total $E$\linebreak$f^{-1}\l b\r$ est appelé la fibre au dessus de $b$}{27/36}
\slideq{$\l A_\alpha\r_\alpha$ est localement fini}{28/36}
\slider{$\forall x\in X,\exists U\ni x,\left|\set{\alpha,A_\alpha\cap U\neq\varnothing}\right|<+\infty$}{28/36}
\slideq{Un difféomorphisme local $f\!:\! M\to N$ préserve l'orientation de deux vériétés orientées}{29/36}
\slider{$\det{\dd_x\l\phi_i\circ f\circ\psi_j^{-1}\r}>0$}{29/36}
\slideq{Orientation d'une variété différentielle}{30/36}
\slider{Donnée d'une classe d'équivalence d'atlas d'orientation}{30/36}
\slideq{Groupe discret}{31/36}
\slider{Groupe muni de la topologie discrète}{31/36}
\slideq{Fibré différentiel trivialisable de base $B$ et de fibre $F$}{32/36}
\slider{Fibré différentiel isomorphe à $\operatorname{pr}_1\!:\!F\times B\to B$}{32/36}
\slideq{Propriétés de $X/G$ pour $X$ une variété différentielle, $G$ discret et $G\acts X$ de manière lisse, propre et libre}{33/36}
\slider{$X/G$ a une unique structure de variété différentielle telle que $\pi\!:\!X\to X/G$ est un revêtement (lisse)\linebreak$\pi$ est alors un revêtement de $X/G$\linebreak En particulier $f\!:\!X/G\to Y$ est lisse  si et seulement si $f\circ\pi\!:\!X\to Y$ est lisse}{33/36}
\slideq{Revêtement d'une variété topologique}{34/36}
\slider{Donnée d'une fibration topologique dont les fibres sont munies de la topologie discrète\linebreak Donnée de $p\!:\!E\to B$ tel que, pour tout $x\in B$, il existe un voisinage $U$ de $x$ tel que $p^{-1}\l U\r\simeq\coprod{\alpha\in A}{}{V_\alpha}$\linebreak La mâme définition peut être donnée pour la \textsl{categorie des variétés différentielles}}{34/36}
\slideq{La partition de l'unité $\l\phi_i\r_{i\in I}$ is subordonnée à la couverture $\l U_\alpha\r_{\alpha\in A}$\linebreak La partition de l'unité $\l\phi_i\r_{i\in I}$ est subordonnée avec les mêmes indices à la couverture $\l U_\alpha\r_{\alpha\in A}$}{35/36}
\slider{Pour tout $i\in I$, il existe $\alpha\in A$ tel que $\operatorname{supp}\l\phi\r\subset U_\alpha$\linebreak Elle est subordonnée avec même indice si $I=A$ et $\operatorname{supp}\l\phi_i\r\subset U_i$}{35/36}
\slideq{Variété orientée}{36/36}
\slider{Variété munie d'une orientation}{36/36}
\end{document}