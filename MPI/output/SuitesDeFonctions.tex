\documentclass[14pt,usepdftitle=false,aspectratio=169]{beamer}
\usepackage{preambule}
\setbeamercolor{structure}{fg=black}
\newcommand{\cv}[2][]{\xrightarrow[\scriptscriptstyle{#2}]{\scriptscriptstyle{\text{#1}}}}\usepackage{analyse,usuelles}\let\epsilon\varepsilon\usepackage{al,complexes,trigo}
\hypersetup{pdftitle=Analyse -- Suites de fonctions}
\title{Analyse\\\emph{Suites de fonctions}}
\author{}
\date{}
\begin{document}
\begin{frame}
    \titlepage
\end{frame}
\slideq{Primitivation d'une limite uniforme}{1/14}
\slider{Si $\forall n\in\mathbb N$, $u_n$ est continue sur un intervalle $I$ et $u_n\cv[CVU]{n\to+\infty}u$ sur $I$, alors $\int[t][a][x]{u_n\l t\r}\cv[CVU]{n\to+\infty}\int[t][a][x]{u\l t\r}$}{1/14}
\slideq{$\anrm f$}{2/14}
\slider{$\sup{\left\{\left|f\l x\r\right|,\;x\in A\right\}}$\linebreak$\anrm\cdot$ est une norme}{2/14}
\slideq{Polynôme trigonométrique complexe}{3/14}
\slider{$\vect{\left\{\e^{\i nt},n\in\mathbb Z\right\}}$}{3/14}
\slideq{$u_n\cv[CVU]{n\to+\infty}u$\linebreak Caractérisation par $\epsilon$}{4/14}
\slider{$\forall\epsilon>0,\;\exists N\in\mathbb N,\;\forall n\geqslant N$\linebreak$\forall x\in\mathbb R,\;\left|u_n\l x\r-u\l x\r\right|\leqslant\epsilon$}{4/14}
\slideq{Propriétés conservées par convergence simple}{5/14}
\slider{Paricité\linebreak$T$-périodicité\linebreak Monotonie\linebreak Caractère $k$-lipsichtzien}{5/14}
\slideq{Théorème de la double limite}{6/14}
\slider{Si $\forall n\in\mathbb N$, $u_n\l x\r\cv{x\to a}l_n\in\mathbb C$ et $u_n\cv[CVU]{n\to+\infty}u$ alors $\l l_n\r$ converge vers $l\in\mathbb C$ et $u\l x\r\cv{x\to a}l$}{6/14}
\slideq{$u_n\cv[CVU]{n\to+\infty}u$\linebreak Caractérisation par les limites}{7/14}
\slider{$\anrm{u_n-u}\cv{n\to+\infty}0$}{7/14}
\slideq{$u_n\cv[CVS]{n\to+\infty}u$\linebreak Caractérisation par $\epsilon$}{8/14}
\slider{$\forall x\in\mathbb R,\;\forall\epsilon>0,\;\exists N\in\mathbb N$\linebreak$\forall n\geqslant N,\;\left|u_n\l x\r-u\l x\r\right|\leqslant\epsilon$}{8/14}
\slideq{Convergence uniforme à partir des dérivées}{9/14}
\slider{Si $\forall n\in\mathbb N$, $u_n$ est $\mathcal C^k$ sur un intervalle $I$\linebreak$\forall j\in\llb0,k-1\rrb$, $u_n^{\l j\r}\cv[CVS]{n\to+\infty}v_j$ sur $I$\linebreak$u_n^{\l k\r}\cv[CVU]{n\to+\infty}v_k$\linebreak alors $v_0$ est de classe $\mathcal C^k$\linebreak$\forall j\in\llb0,k\rrb$, $v_0^{\l j\r}=v_j$\linebreak$\forall j\in\llb0,k-1\rrb$, $u_n^{\l j\r}\cv[CVU]{n\to+\infty}v_j$}{9/14}
\slideq{Polynôme trigonométrique réel}{10/14}
\slider{$\vect{\left\{\cos{nt},n\in\mathbb N\right\}\uplus\left\{\sin{nt},n\in\mathbb N^*\right\}}$}{10/14}
\slideq{Conservation de la continuité par passage à la limite uniforme}{11/14}
\slider{Si $\forall n\in\mathbb N$, $u_n$ est continue en $a$ et $u_n\cv[CVU]{n\to+\infty}u$ sur un voisinage de $a$, alors $u$ est continue en $a$}{11/14}
\slideq{Théorème de Weierstrass}{12/14}
\slider{Toute fonction continue sur un segment y est limite uniforme d'une suite de fonctions polynomiales}{12/14}
\slideq{$u_n\cv[CVS]{n\to+\infty}u$\linebreak Caractérisation par les limites}{13/14}
\slider{$\forall x\in\mathbb R,\;u_n\l x\r\cv{n\to+\infty}u\l x\r$}{13/14}
\slideq{Théorème de Weierstrass trigonométrique}{14/14}
\slider{Toute fonction complexe (ou réelle), continue et $2\pi$-périodique est limite uniforme sur $\mathbb R$ d'une suite de polynômes trigonométriques complexes (ou réels)}{14/14}
\end{document}